\subsection{Region One Construction}

\hskip 0.15 in 
The R1 chambers were designed and constructed through a collaboration 
of Carnegie Mellon University, the University of Pittsburgh, and the 
University of Massachusetts.  These detectors constitute the innermost of 
the three nested drift-chamber packages surrounding the target in the 
spectrometer~\cite{dscnim}.  The R1 detectors track all charged particles 
prior to their entry into the magnetic field of the CLAS torus.

R1 consists of six identical sectors integrated into a single unit.  Each
sector consists of 1296 hexagonal drift cells ranging in diameter from 15 to 
17~mm, with all wires located within 1~m of the primary beam.  Each sector 
consists of two superlayers, the stereo layer at smaller radius, and the axial 
layer at larger radius.  This arrangement, which is opposite to the outer two 
Regions, was necessitated by space constraints within the torus.  While each R2 
and R3 sector consists of 12 layers of sense wires, the limited space within the 
torus allowed for only 10 layers in each R1 sector.  The final R1 design included 
6 axial layers and 4 stereo layers.  Space constraints at the most backward 
angles spanned by R1 required a reduction in the number of axial wire layers 
from 6 to 3, as can be seen in Fig.~\ref{region1}.  Also the layer contour changes 
shape from circular to straight for polar angles greater than 90$^{\circ}$.

%%%%%%%%%%%%%%%%%%%%%% Figure : Region One %%%%%%%%%%%%%%%%%%%%%%%%%%%%%%%
\begin{figure}[htpb]   
\vspace{7.0cm}
\special{psfile=region1.eps hscale=70 vscale=70 hoffset=-65 voffset=-215} 
\caption{\small{Schematic side-view cross section of R1 highlighting the 
locations of the main torus, mini-torus, target, and detector supports.  
The shaded areas along the endplates correspond to the wire hole patterns.}}
\label{region1}
\end{figure}   
%%%%%%%%%%%%%%%%%%%%%%%%%%%%%%%%%%%%%%%%%%%%%%%%%%%%%%%%%%%%%%%%%%%%%%%%%%

Sense wires are shifted by $\pm$300~$\mu$m from the nominal position from one 
wire layer to the next.  This ``mini-stagger'' feature feature helps to resolve 
the track path for those tracks that run nearly parallel to a series of drift 
cells.  The size of the stagger was chosen such that the induced drift time 
differences could be resolved easily.  R2 and R3 did not stagger sense wires 
in this way.

Fig.~\ref{region1} shows a schematic of how the wire-supporting endplates are 
captured within the torus cryostat, as well as some of the detector support 
structures.  To protect the chambers from charged electromagnetic background 
emerging from the target, a small normal-conducting ``mini-torus'' magnet is 
located inside of R1 during standard electron-beam operation.  The integral 
magnetic field of the mini-torus is about 5$\%$ that of the main torus.  

The shapes of the outer main torus and the inner mini-torus limited the
amount of space available for the R1 chambers, along with their associated 
on-board electronics, cables, and support hardware.  Fig.~\ref{xsect} shows a 
projection of a portion of the detector near 90$^{\circ}$ to highlight how the 
wire-support hardware, circuit boards, and cables were arranged to fit within 
the shadow region of the torus cryostats.

%%%%%%%%%%%%%%%%%%% Figure : Cross Sectional View %%%%%%%%%%%%%%%%%%%%%%%
\begin{figure}[htpb]   
\vspace{9.5cm}
\special{psfile=xsect.eps hscale=47 vscale=47 hoffset=-15 voffset=-20} 
\caption{\small{Endplates and other components from neighboring R1 sectors 
seen in projection.  Feedthroughs, crimp pins, and circuit boards for 
pairs of sectors, along with the strut fixtures, had to fit within 
3.5~cm of space to be contained within the shadow of the torus magnet.}}
\label{xsect}
\end{figure}   
%%%%%%%%%%%%%%%%%%%%%%%%%%%%%%%%%%%%%%%%%%%%%%%%%%%%%%%%%%%%%%%%%%%%%%%%%

The 228-cm long wire-supporting endplates are made of 8-mm thick, 50-cm wide
Alcoa Alca-Plus cast tooling plate, a non-magnetic alloy reputed to have low 
internal stresses, and, therefore, thought to have minimal distortion after the 
hole pattern was drilled.  Holes for the wire feedthroughs and assembly 
hardware were drilled with a 4-mm diameter and an absolute position tolerance 
of 130~$\mu$m.  Measurements of the finished plates showed systematic positional 
distortions near 100~$\mu$m, some due to the manufacturing, and some of which 
were consistent with few-degree temperature differences between manufacturing 
and measuring~\cite{ereport}.

Single sectors are built out of endplate pairs that support most of the R1
mechanical loads.  During sector construction, the endplate spacing was 
controlled at 22 points along their inner and outer perimeters to an accuracy of 
25~$\mu$m using temporary 1.59-cm diameter aluminum posts.  Struts machined to 
precise angles and affixed to the edges of the endplates supported the posts, 
and later formed the mating surfaces for joining one sector to the next.  A 
ball-and-socket mechanism allowed accurate coupling of the posts to the sectors 
while later allowing smooth and simple removal of the posts upon integration of 
the finished sectors into a single unit.  The lengths of the posts and struts
defined the geometry of the sectors to 50~$\mu$m precision.

Deciding where to support the thin sector endplates, in view of the heavy
loads placed upon them, presented a complex mechanical problem that was
solved using two approaches.  First, finite-element modeling of the endplates
was used to understand the distribution of forces and deflections under the
loads of the wires and of gravity \cite{ansys}.  This allowed for proper
selection of the number, spacing, and position of the support posts.  Second, 
a full-scale prototype of the detector was built to test and improve the 
mechanical design, as well as to practice the assembly procedure.  Thick piano 
wires modeled groups of about 200 sense and field wires, and lead weights 
modeled the electronics and cables.  Fine wires were strung to monitor tension
changes in response to various steps in the assembly and installation procedure.

The endplate pairs each connect to thick aluminum back-plates, which in turn
connect to a large six-sided annular support ring that also serves to align
the six sectors.  After installation, three support legs anchor the support ring 
to the upstream portion of the cryostat, constraining the position of the chamber 
along the central beam axis.  At the downstream end of each sector, the 
sector-plates connect to a small six-sided cylindrical pipe that ultimately anchors 
onto a set of dowel pins on the cryostat.

Fig.~\ref{xsect} highlights the hardware pieces described in the text including 
the struts and wire feedthroughs.  The 4.8-mm length of the plastic feedthroughs 
provide separation between each trumpet at high voltage and the grounded endplate.  
Not shown are the custom-built polyethylene U-channels used to attach the 15-$\mu$m 
aluminized-nylon gas windows to the inner and outer edges of the endplates.  For
this purpose the glue seals were made with Scotchweld 2216 epoxy.  A 0.64-cm 
diameter stainless-steel gas input line runs longitudinally along the endplate 
to inject gas into the downstream end of each sector.  Another line is used to 
exhaust gas from the upstream end of the detector. 

Most of the challenges associated with the detector construction came about because 
the endplates are not rigid.  This lack of stiffness led to two important effects 
that required detailed consideration \cite{rat}.  First, the forces applied by the 
wires caused the endplates to deflect by up to 1~mm.  These deflections, which 
changed the tensions in some wires by over 50$\%$ of their strung value, were 
compensated for by pre-tensioning the sectors prior to stringing.  That is, 67 heavy 
steel wires were installed in order to deform the endplates prior to stringing, 
using effective mass loads between 2 and 14~kg.  For each of the six sectors, 3435 
field and guard wires were strung at a nominal tension of 100 g, and 1296 sense 
wires were strung at a nominal tension of 30~g, resulting in $\approx$400 kg of 
effective mass load on each endplate.  In contrast, the effective gravitational 
load of each endplate, with its corresponding circuit boards, electronics, and cables, 
comprised about 60~kg.  The pre-tension wires were released in stages and eventually 
removed during sector stringing.  The stereo wires exerted the equivalent of 
about 14~kg of shearing force which tended to distort the chamber sectors. However, 
tests showed that the structure was rigid enough so that wire tensions were 
not significantly affected.

A second effect, peculiar only to the R1 wires, resulted from gravitational 
forces acting on the long, thin endplates.  During construction, each sector was 
supported by its ends in an orientation such that the wires could be strung 
vertically.  However, in this orientation the gravitational forces on the
endplates, and hence the endplate distortions, were at a maximum.  The temporary 
posts maintained the relative spacing between the endplates to high precision, but 
gravity acted to deflect the center of each endplate downward by 6~mm.  Thus, the 
axial wires, which were exactly parallel to the posts, were unaffected by this effect, 
as both endplates had identical gravitational sags.  However, due to the slight angle 
of the stereo wires, their length was slightly increased or decreased, depending upon 
which direction the sector was rotated from its ``sag-free'' endplate orientation, or 
the orientation with the wires running horizontally.  This sagging of the endplates 
during sector stringing had to be taken into account in order to allow all wire tensions 
to fall within tolerance when the six sectors were integrated together.  By correctly 
adjusting the stringing tensions along the length of the sector, and proper selection 
of the number of pre-tension release steps, the final wire tensions in the sag-free 
orientation were kept to within 20$\%$ of their nominal values.









