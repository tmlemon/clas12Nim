\section{Chamber Operation and Performance Monitoring}

\subsection{Choice of Gas}
\hskip 0.15in

The main requirements for the chamber gas were that it have reasonably low 
multiple scattering, allow for reasonable gas gains, have short collection 
times in order to reduce the random background expected from M{\o}ller 
electrons and target-generated X-rays, and be inexpensive because of the 
large volume of the chambers. Also, safety considerations motivate the use of
a non-flammable gas mixture.  Additional concerns about small gas 
leaks and the proximity of many photomultiplier tubes argued against helium 
mixtures.  Ultimately a 90$\%$ argon - 10$\%$ CO$_2$ mixture was employed 
for several reasons: the gas has a fairly high saturated drift velocity 
($>$ 5~cm/$\mu$s), and it has an operating voltage plateau of several hundred 
volts before breakdown occurs.  The 90$\%$/10$\%$ mixture 
provides good efficiency, adequate resolution, and reasonable collection times.




\subsection{Selecting the Proper Operating Voltage}
\hskip 0.15in
In this section we discuss our operating voltages and how it was selected.

First, we discuss how we divide the total voltage between our sense, field 
and guard wires in order to mimic a cell layout with an infinite number of
layers, to achieve a situation in which all wires, regardless of layer, have
the same gain.  Then we discuss our choice of the total sense to field
wire difference in voltage; including the resulting gas gain and efficiency.


\subsubsection{Dividing the Total Voltage between Sense, Field and Guard Wires}
As noted previously, we ran our chambers with a mixed voltage scheme:
positive high voltage on the sense wires, negative voltage on the
field wires and positive voltage on the guard wires.
This mixed-voltage scheme has a couple of advantgages over a scheme in
which the field wires, for example, are held at ground potential:
\begin{itemize}
\item fewer field lines run from the sense wire to the endplate which
is grounded.  This reduces the likelihood of producing a `Malter effect'
(Ref.~\cite{Malter}) in which an accidental source of cathode emission
(due to an insulating contaminant on the endplate, for example) causes
a self-sustaining discharge
\item the sense to ground potential and the field to ground potentials 
are smaller; decreasing surface electric fields on the on-chamber
circuit boards
\end{itemize}

In addition, by carefully selecting the values of the sense, field and
guard wire voltages we can create potential distributions which mimic
an infinite grid of cells, where the gain on any wire is the same as
any other, regardless of whether is the first, last or middle layer.

This optimum condition is reached when the sense voltage is twice the
field voltage (and opposite sign).  This is because we have twice as many
field wires as sense wires and all field lines which originate on a
field wire land on a sense wire.  

The guard wire voltage was then chosen so that the total charge on all wires is zero.  
If we have a nearby ground plane due to the metallized gas bag then it will
have no net effect on the nearby wire planes if there is no net flux of
electric field through the bag which is the case if the enclosed net charge
is zero.

So, the ratio of voltages from Sense to Field to Guard wires is 1 : -1/2 : 5/14.

\subsubsection{Operating Voltage, Gas Gain and Layer Efficiency}
The gas gain varies exponentially with the total sense to field wire voltage
difference, with a doubling voltage of about 100, 110 or 120V respectively, for
R1, R2 and R3.  During our Fall 2019 run, we ran with sense - field wire voltage
differences of 2100, 2325 and 2475 V, respectively for R1, R2 and R3

We calculate that our total gas gain is approximately 2.7~$10^4$, 3.7~$10^4$, and 4.4~$10^4$,  
respectively, for R1, R2 and R3.


References: `A study of electron drift velocity in Ar-CO2 and AR-CO2-CF4 gas
mixtures', NIM A340 (1994) p.485-490

CLAS-Note 2009-27 1Calculation of Drift Chamber Gas Gain' Mestayer
