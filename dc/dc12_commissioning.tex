\section{Commissioning}

In this section we describe our procedures to do the following:

\begin{enumerate}
\item install electronics boards and ``on-chamber'' cables
\item gradually bring the chambers to full operating voltage (``burn-in'') 
and to test for any other problems
\item install and survey the chambers
\item diagnose and fix any early failures, and
\item perform efficiency scans over a range of HV and discriminator settings
\end{enumerate}

\subsection{Electronics Installation and Turn-on}
After the chambers were strung and went through a mechanical quality
check to insure that all wires are intact and properly tensioned, we
installed the on-chamber electronics boards.

There are two types of board: one is a High-Voltage Distribution Board (HVTB)
which brings RC-filtered high voltage to the wires.  Because there are three
types of wire: `sense', `field' and `guard' we supply three different voltages
to each board.  As discussed in the design section, we have a mixed high
voltage system: the sense (or anode) wires are operated at positive voltage,
typically about 2000 Volts; the field (or cathode) wires are operated at
about -1000 V and the guard wires at about +700 V.  This creates an
electric field which converges radially onto the sense wire with surface
field values of about 200 kV/cm.  Similarly each field wire has an
electric field directed radially outward from the surface with a strength
of about 50 kV/cm.  The surface field for the field wires is weaker 
because there are twice as many field wires as sense wires and their
diameter is larger (80 microns compared to 30).  The guard wires
close out the structure and their voltage is chosen to approximate an
infinite grid of wires; thus every sense and field wire in a superlayer
has the same field configuration unless there is an irregularity (like
a missing, broken wire) nearby.

On the other side of the chamber are our Signal Translator Boards (STB) 
which support an individual Single Inline Package (SIP) transimpedance
pre-amplifier for each sense wire.  This pre-amplifier takes the
small current pulse (microAmps) and translates it into a voltage 
pulse with a transimpedance of 2 mV/microAmp.  The signals (typically
10's to 100's of mV and 10's to 100's of nanosecond duration) are
transmitted down twisted-pair conducting wire to our downstream
Drift Chamber Readout Boards (DCRB) which further amplifies and
discriminates the voltage pulse and then converts the leading edge
to a digital time signal.
See figures (STB, HVTB) in the electroncis section for more details.

After stringing was complete we did the following:
\begin{enumerate}
\item ``daisy-chained'' the field wire crimp pins so that a single
HV cable could power two rows of field wires (32 wires)
\item physically positioned the boards so that their plated through
holes aligned directly above the sense wire crimp pins, and attached
the boards to the chamber with screws, and
\item electrically connected each sense wire crimp pin to each
plated through hole using a conductive rubber 'sleeve' which fit
over the crimp pin and also contacted the plated through hole on
its outer radius.
\end{enumerate}

Now the chamber was ready for ``burn-in'' and ``pre-testing''.

\subsection{``Burn-in and Pre-testing''}
When drift chambers are first turned on, they typically draw fairly high
`dark' currents, even at low voltages.  The standard procedure is to
slowly raise the high voltage, wait for a certain time period during
which the current subsides and raise the voltage again, and so on.
For our chambers, the typical time period was an hour and the typical
voltage step was 75 V which is approximately the `doubling voltage' of
our chambers (the voltage step which increases the gain by a factor
of two).

\subsection{Installation and Survey}
- chambers attached by ball and socket joints to rods which are attached
on the other end by ball and socket to the toroidal magnet frame



\subsection{Determining the Operating Values of the Discriminator Thresholds and High Voltage Settings}

We set the discriminator levels in the DCRB's to reduce the accidental hit rate (with no beam) due to electronic
noise to be less than about 1\%.  Since the electronic noise was generally proportional to wire length, we had less
electronic noise on the smaller R1 chambers.  Using this criteria, we set the thresholds to 30, 45 and 45 mV, respectively,
for R1, R2 and R3.  
Once we set the discriminator thresholds, we performed a High Voltage efficiency scan.  We raised the high voltage in
steps of 75 Volts and analysed the data.  We set the operating value for the high voltage at the point at which
the layer efficiency equaled or exceeded 97\%.

We determined the layer efficiency using the `excluded layer' method.  In one superlayer (of six layers) we found track
segments by our usual fitting method, but ignoring the data from a pre-selected layer (layer 3, for example).  We then
projected the track segment through that layer and determined whether or not the indicated wire (or an adjacent one) had a good hit.
Fig.~\ref{effcy-vs-voltage} shows the `plateau curve' of efficiency plotted versus voltage for one typical chamber 
and superlayer.  The layer efficiency at the chosen operating high voltage poiont was between 97\% and 98\%.

%%%%%%%%%%%%%%%%%%%%%%%%%%%%%%%%%%%%%%%%%%%%%%%%%%%%%%%%%%%%%%%%%%%%%%%%%%%
\begin{figure}[htbp]
\vspace{5cm}
\begin{picture}(50,50)
\put(-10,10)
{\hbox{\includegraphics[width=0.35\textwidth,natwidth=610,natheight=642]{img/trace-routing-schematic.jpg}}}
\end{picture}
\caption{\small{Layer efficiency plotted versus high voltage for a typical superlayer.}}
\label{effcy-vs-voltage}
\end{figure}
%%%%%%%%%%%%%%%%%%%%%%%%%%%%%%%%%%%%%%%%%%%%%%%%%%%%%%%%%%%%%%%%%%%%%%%%%%%


The single layer inefficiency is not uniform across the drift cell.  It is higher near the sense wire and also near the outer
edge of the cell.  A track passing close to a sense wire leave many ions in the cell, but the ion arrival times are stretched
out from near-zero to the maximum drift time, Tmax.  The result is that the pre-amplifier's output signal has a low voltage
amplitude but persists for a long time.  So, even though the collected charge is large the voltage put out by our trans-impedance
pre-amplifiers may not be large enough to exceed the voltage discriminator threshold of the DCRB,
For the case of tracks near the outer edge of the cell (so-called `corner-clippers') they simply leave a very small number
of ions in the cell and thus have a small signal.

In Fig.~\ref{effcy-vs-doca} we plot the layer efficiency as a function of the distance-of-closest-approach (DOCA) and
you can see the characteristic rise of the ineffiency near to and far from the sense wire.
%%%%%%%%%%%%%%%%%%%%%%%%%%%%%%%%%%%%%%%%%%%%%%%%%%%%%%%%%%%%%%%%%%%%%%%%%%%
\begin{figure}[htbp]
\vspace{5cm}
\begin{picture}(50,50)
\put(-10,10)
{\hbox{\includegraphics[width=0.35\textwidth,natwidth=610,natheight=642]{img/trace-routing-schematic.jpg}}}
\end{picture}
\caption{\small{Layer efficiency plotted versus DOCA.}}
\label{effcy-vs-doca}
\end{figure}
%%%%%%%%%%%%%%%%%%%%%%%%%%%%%%%%%%%%%%%%%%%%%%%%%%%%%%%%%%%%%%%%%%%%%%%%%%%
