\section{Chamber Construction}
\label{construction}
\subsection{Construction Overview}

The three chamber types (called ``regions'', with ``Region 1'' abbreviated as 
``R1'', etc.)  share the same basic design elements simply
scaled up in size by a factor of 1.5 for R2 relative to R1 and a factor
of 2 between R3 and R1.  
Each chamber is a solid trapezoid in shape, with  
a pair of wire-supporting endplates that bear both the load of the 
wire tensions and the weight of all associated hardware. A representative 
chamber is shown in Fig.~\ref{chamber-exploded}, with parts indicated.
This particular drawing is of a R1 chamber, but all chambers have the
same basic parts.

%%%%%%%%%%%%%%%%%%%%%% Figure : generic chamber sketch %%%%%%%%%%%%%%%%%%%%%%
\begin{figure}[htpb]   
\vspace{10cm}
\begin{picture}(35,35)
\put(-130,0)
{\hbox{\includegraphics[width=1.0\columnwidth,natwidth=610,natheight=642]{img/chamber-exploded.png}}}
\end{picture}
\caption{\small{Assembly of a typical drift-chamber
(here a R1 sector) highlighting the common components.}}
\label{chamber-exploded}
\end{figure}   
%%%%%%%%%%%%%%%%%%%%%%%%%%%%%%%%%%%%%%%%%%%%%%%%%%%%%%%%%%%%%%%%%%%%%%%%%%


The chamber bodies were constructed from the accurately machined plates
(2 endplates, a ``nose-plate'' and a ``back-plate'').
The endplates themselves were an assembly of a main plate with precision-drilled
holes to which we bolted and glued stiffener bars.  In the case of
R1 the main plate was aluminum and the stiffener bars were stainless steel;
for R2 the main plates were G10 and the bars stainless steel, and for R3
the main plates were themselves an assembly of two thin steel plates with a foam interior.
No additional stiffener bars were used for R3.

At the radially outward end of each chamber, a thick ``back-plate'' was 
employed to maintain the relative 
alignment of the endplates, to stiffen the chamber against bending moments, 
and to provide a place to attach gas seals and fittings. At the radially inward 
end of each chamber, the endplates were connected together via a small joining 
piece called the ``nose-plate''.  The hardware fabrication and placement 
was of critical importance to the dimensional accuracy of the chambers.



\subsection{Construction Materials}
\label{materials}

To insure that the drift chambers perform well for 10 years or more, care was taken to 
specify that all materials in contact with the gas volume were clean and ``chamber 
safe'' as defined in Ref.~\cite{kadyk}.  All construction was carried out in 
Class-10000 or better clean rooms.

The drift-chamber frames were made primarily of aluminum (R1), fiberglass (R2),
or steel-clad structural foam (R3).  The aluminum and steel endplates were 
manufactured with machine oils and were subsequently cleaned with  
Micro-laboratory detergent from the Cole-Parmer Instrument Company.  The 
fiberglass endplates were machined without any lubricating oils.  Immediately 
prior to chamber assembly all endplates were sequentially cleaned with 
detergent,
deionized water and alcohol, and then blown dry with pure nitrogen gas.  The 
wire feedthroughs, trumpets, and crimp pins were cleaned in an ultrasonic bath 
with detergent and then rinsed in a second ultrasonic bath with deionized water.

During construction three different types of epoxy were used in areas exposed 
to the chamber gas.  Shell Epon resin 826 mixed with Versamid 140, and 
Scotchweld varieties 210 and 2216 were employed.  These mixtures have been 
studied extensively and found not to outgas significantly~\cite{nasa}.

The on-chamber gas tubing employed is mainly stainless steel, with some
nylon tubing included for gas manifolds.  Special care was taken during
all steps of construction and testing to ensure that no oils or
silicones contacted any of the chamber materials.



\subsection{Chamber Body Construction}

The construction of the chamber bodies consisted of 3 main stages:
\begin{itemize}
\item receipt, inspection and cleaning of parts,
\item assembly of the main drilled plate and stiffener bars into a complete endplate,
followed by insertion and gluing of the feedthroughs into the pre-drilled holes 
in the main plate, and 
\item over-all assembly of the endplates and nose- and back-plates to make
a chamber ``box''.
\end{itemize}

\subsubsection{Inspection and Cleaning}

A visual and tactile inspection was performed on the endplates and 
structural frame upon delivery.  The endplates and structural 
frame underwent a gross cleaning by soaking in a low residue laboratory 
degreasing solution with hand scrubbing, followed by a high-pressure water 
rinse. The parts then received their final cleaning using an ultrasonic 
bath of a laboratory-grade detergent solution.  After two hours they were
removed from the bath and rinsed with de-ionized water and then sprayed 
with methanol to aid drying. Once dry, the parts were heat-sealed in nylon 
bags with a nitrogen atmosphere to await assembly.

Smaller parts such as feedthroughs and hardware were cleaned using a 
bench-top ultrasonic cleaner with laboratory detergent, followed by a 
de-ionized water rinse and a dousing of methanol to facilitate drying.  
In addition, the injection-molded parts were specified to be free of silicon 
mold releases.
 
\subsubsection{Endplate and Chamber Body Assembly}
The endplates were assembled on a table in the cleanroom.  The various parts,
the pre-drilled main plate and the various stiffener bars, were pinned
and glued into place.  
Once the endplate assembly was finished,
wire feedthrough insulators were inserted and glued into each hole. Special care was 
taken to use only the minimal amount of glue required to provide a solid gas 
seal and to prevent glue contamination inside the detector.  

Once the endplates were assembled, the chamber box was assembled from the
endplates, nose-plate and back-plates using a variety of special-purpose
fixtures.  With the box held in its final configuration, parts were bolted
and glued into place using the special glues mentioned in Subsection~\ref{materials}.



\subsection{Wire Choice}
\hskip 0.15in
Our ``thin endplate'' design required minimizing wire tensions and
thus diameter of the wire.  The real key to reducing wire tension is to
make the field wires (which are much larger than the sense wires) as 
thin as possible and to make them out of low-density metal.  

In general, designers have chosen very small diameter sense wires because they
require lower operating voltages.
The sense wire choice for all of our chambers, supplied by the Luma
Sweden Company, is 30-$\mu$m diameter gold-plated tungsten.  
The previous CLAS drift chambers used 20-$\mu$m diameter wire for the
sense wires.  We decided to use the thicker 30-$\mu$m wire for two 
reasons: first, it is significantly tougher making it easier to handle without
accidental kinking and less likely to break, and second, to operate at the same 
gas gain as chambers with the thinner wire requires a high-voltage setting 
(approximately 300 V higher) and this results in a higher field throughout the cell
and a more linear drift velocity function.  

Tungsten was chosen because of its durability, 
and the gold-plating of the wires, amounting to a thickness of 0.127~$\mu$m, 
ensures chemical inertness as well as a smooth surface finish.  The expected 
electronics gain and thresholds dictated that the gas gain be a few times 
10$^4$.  Under this condition, the electric field at the surface of the sense 
wires is $\approx$200~kV/cm.  

We chose 80-$\mu$m gold-plated Cu-Be wire for our field wire and
140-$\mu$m gold-plated Cu-Be wire for our guard wires; both supplied
by Little Falls Alloys.
Cu-Be wire is very tough, is easily plated, and resists ``flaking'' of the gold
plating. 
Minimizing the diameter for the field wire is important because it means that they 
could be strung at lower tension than a thicker wire for the same gravitational sag.  
This minimizes the forces on the endplates that we wanted to keep as 
thin as possible to maximize the solid-angle of the sensitive area of
the chambers.

The 80-$\mu$m diameter was chosen because it is the smallest possible choice
that will not initiate cathode emission at the surface.
At our operating conditions, the surface field at the field wire
surface is less than 50~kV/cm, and this will minimize 
conditions which could cause unwanted cathode emissions and a noisy chamber.  
We note that
our choice violated the ``20~kV/cm rule'', the conventional wisdom that
a surface field greater than 20 kV/cm on the cathode wire would lead to 
a noisy chamber.
Our own studies (Ref~\cite{cathode-emission}) showed that there was no cathode
emission below 50~kV/cm from any wire with good surface finish.  Each batch
of wire was tested with our test device (Ref~\cite{patent}) to ensure that at operating field 
values there was no emission.  

\subsubsection{Wire Tensions}
A basic principle of our drift chamber design is that each drift cell is a perfect hexagon.
We used this geometrical constraint to determine the hole placement in the endplates.
This geometrical design assumed that wires are straight lines.  Of course, real wires
sag across the wire span due to gravity.  To keep our perfect hexagonally shaped cells,
we tensioned our three types of wires (sense, field, and guard) such that they
{\bf sagged equally} under their gravitational loads.

Our 30~$\mu$m tungsten sense wires, 80~$\mu$m Cu-Be field wires and 140~$\mu$m Cu-Be guard
wires had linear densities of 0.014, 0.042, and 0.129 g/m, respectively.  To sag equally
under gravity, they were strung at 20, 62, and 190 g of tension, respectively.
In each chamber there were 12 rows of 112 sense wires, 28 rows of 112 field wires
and 4 rows of 112 guard wires for a total wire tension of 306 Kg.
This caused some bowing of our thin endplates.  This bowing and the sagging
of the wires themselves is discussed in the section on geometrical distortions,
Section ~\ref{geom_distortions}.


\subsection{Chamber Wire Stringing}
The sense, field, 
and guard wires in each 
sector were strung between pairs of wire ``feedthroughs''.  As the endplates of 
each chamber faced each other at a 60$^{\circ}$ angle, the wires had to bend 
30$^{\circ}$ at each endplate.  This was accomplished by means of a large radius 
stainless-steel insert at the tip of each feedthrough known as a ``trumpet''.

This trumpet was fitted into an injection-molded Noryl plastic feedthrough.  
The Noryl plastic was reinforced with glass beads to make them stiffer.
During early prototype testing, it was determined that the surface conductivity
of this glass-bead loaded composite was little affected by room humidities
as high as 60\%.  This is in contrast to similar plastic strengthened
with microscopic glass strands, which performed poorly in high-voltage stand-off
tests in humid conditions.

All wires were held in place using gold-plated copper crimp pins.  
Low-outgassing epoxy was employed to ensure a gas seal around the feedthroughs 
and crimp pins.  

Because all of the chambers have the same shape, differing only in
size and some materials, we strung them all using the same basic method.
They were gravity-strung using a similar methodology to that 
used when stringing the previous CLAS drift chambers.  The detector box 
assembly was mounted to a stringing fixture.  The frame was connected to 
the fixture using attachments that allowed the chambers to rotate along 
their center of mass.  This permitted safe and easy 
rotation of the detector by hand.  
The stringing fixture was mounted to the floor using 
concrete anchors, minimizing floor obstruction.

Under full wire tension, the endplates bow inward as much as 2~mm 
(see Section~\ref{geom_distortions} for a discussion of this issue).
Because of this bowing, it is necessary to pre-tension the chamber
so that the endplates are bowed into their final state at the 
beginning of the stringing process.

We pre-tensioned the chambers by over-tensioning the 140 $\mu$m guard
wires such that the total wire tension load was equal to the final, 
fully strung load.  The over-tensioning was done using an adjustable
spring attached to each guard wire. 
We then would string the field wires, starting at one end of the chamber.
After stringing a ``column'' of 14 field wires, we would reduce the
tension on the associated guard wire to its nominal value, and crimp
the guard wire.  In this way, the total wire tension on the endplate
remained approximately constant.


With nearly 130,000 wires to be threaded through the chambers and only 2 years 
for construction, it was important to minimize the time required to string 
each wire.  All chambers were strung with 
the wires running vertically.  The stringing technique involved attaching a 
small steel needle to the wire before threading it through the feedthrough in 
the upper endplate.  The wire was then despooled and gravity acted to bring the 
wire close to the feedthrough in the lower endplate.  A small magnet was then 
used to pull the needle and wire through the lower feedthrough.  After the upper 
crimp pin was attached and crimped, and the remaining slack in the wire was removed, the lower 
crimp pin was slid over the wire and stringing weights were attached to the wire to set the 
proper tension.  The lower crimp pin was then crimped, completing the process.

At the beginning of each shift, wires that had been strung on the previous days
were tested in two ways:
\begin{itemize}
\item a continuity test checked that the wire made a good electrical contact
from one feedthough to its partner on the other endplate, and
\item a tension measurement was performed using an ``oscillating wire'' technique.
A static magnetic field was established using large Helmholtz coils.  A sine
wave electric current was established on the wire being tested by a frequency-controlled 
AC power supply.  The AC current was varied in frequency.  For a 
few-second interval, the Lorentz force on the wire caused it to vibrate and
then during a few-second ``voltage-off'' period, the resulting induced voltage
was read out on an oscilloscope.  In this way, we determined the resonant frequency
of the wire.  If this frequency agreed within limits with a pre-calculated
table of nominal frequency, the wire passed the frequency test.
\end{itemize}
Wires that failed either test were removed and re-strung.

  Wires that wrapped around each other while being threaded through the chamber 
were a major contribution to stringing inefficiency. To avoid the wrapping
problem, a machine was built to spool the wire through the chambers quickly and smoothly.
Another important development was the design of a crimp pin 
that accepted both the tungsten and copper-beryllium wire types.  Using a 
thick-walled 
copper pin ensured a good crimp through a range of gap settings~\cite{sbc}.  This 
eliminated the need to use separate crimping tools, each requiring frequent 
calibrations.  This was possible because the wire position was determined by the 
radius of the trumpet at the end of the feedthrough inside of the chamber, and 
not by the concentricity of the pin, feedthrough, and endplate-hole diameters. 
As a result, the average time to string a wire was less than 4 minutes.

After all wires were strung,
a small amount of glue was applied to the glue well
in the feedthrough to firmly fix the crimp pin in place.  After this ``potting''
operation was done, the chambers were taken off of the stringing fixture and
placed on stands on the floor for final continuity checks.


\subsection{Region One Construction (Special Considerations)}

\hskip 0.15 in 
The R1 chambers were designed and constructed through a collaboration 
of Idaho State University and Jefferson Lab.  These 
chambers are located before particles enter the magnetic field of the torus
and are thus key to good angular resolution.  They also are subject to the
highest level of background radiation. 

As we have seen from the generic assembly sketch of a chamber, the R1
chambers have a similar shape to the R2 and R3 chambers, differing in
scale and in some material choices.
Most notably, the endplates were constructed of aluminum with stainless
steel stiffener bars.
Fig.~\ref{region1-endplate} shows an assembly drawing of an R1 endplate.


%%%%%%%%%%%%%%%%%%%%%% Figure : Region One %%%%%%%%%%%%%%%%%%%%%%%%%%%%%%%
\begin{figure}[htpb]   
\vspace{7.0cm}

\caption{\small{Assembly drawing of a R1 endplate.}}
\label{region1}
\end{figure}   
%%%%%%%%%%%%%%%%%%%%%%%%%%%%%%%%%%%%%%%%%%%%%%%%%%%%%%%%%%%%%%%%%%%%%%%%%%

The main challenges in the R1 construction and design came about because
of the small wire spacing (~8mm between sense and field wires).  This
increased the electrostatic attraction of neighboring wires if they are
not perfectly and symmetrically placed, and it also made the physical act
of stringing the wires more difficult.

Wires with opposite voltage are electrostatically attracted.  If perfectly
placed in a symmetric array the forces would cancel each other. 
However, the sense wires might be slightly misplaced and so they would feel
a force which, if the tension were below a critical value, would increase
and pull them further out, further increasing the force, and so on until 
the wire begins to oscillate and then spark.  For our electric field configuration
this critical tension was about 2 grams, a factor of 9 below our nominal
tension of 18 grams.




\subsection{Region Two Construction (Special Considerations)}

\hskip 0.15 in 
The R2 chambers, which were designed and constructed by Old Dominion University 
in collaboration with Jefferson Laboratory, are the middle of the three  
drift-chamber packages.  They track all charged particles in the magnetic field 
of the torus near the point of maximum sagitta.  The six identical R2 sectors 
are approximately equilateral triangular boxes with 3 m sides. 
They are located at a radius of $\approx$3~m from the nominal target location.  
  
The R2 chambers were designed with ultra-thin endplates which were thin enough
to be wholly within the ``shadow'' cast by the torus cryostat; in other words,
the full length of the wires is in the active fiducial volume of CLAS12. 
All chamber support hardware and electronics had to fit 
entirely within this shadow region.


Fig.~\ref{dc-corner} shows a slice through a chamber endplate (R2 in this case)
showing some of the wire positioning hardware and attachment of the electronics 
boards.
%%%%%%%%%%%%%%%%%%%%%% Figure : DC Sector Schematic %%%%%%%%%%%%%%%%%%%%%%
\begin{figure}[htpb]   
\vspace{8cm}
\special{psfile=img/r2_inserts.eps hscale=70 vscale=70 hoffset=0 voffset=50}  
\caption{\small{Schematic cross-sectional view of the R2 endplate showing
wire-positioning hardware.}}
\label{dc-corner}
\end{figure}   
%%%%%%%%%%%%%%%%%%%%%%%%%%%%%%%%%%%%%%%%%%%%%%%%%%%%%%%%%%%%%%%%%%%%%%%%%%


The R2 chambers have to operate 
in a magnetic field up to 2~T, and the chambers have to withstand any rapid 
changes in magnetic field, such as what might occur due to a magnet quench.
The R2 endplates are constructed from 2-cm thick Stesalit 4411W, a disordered 
epoxy-fiberglass composite commonly used in wire-chamber construction
\cite{stesalit}, and known not to cause aging problems~\cite{stesalitaging}.  
Using a nonconducting material eliminates any possible forces on the endplate 
due to eddy currents produced during a magnetic-field quench.  

The ``stesalit'' composite is not very stiff and, if not reinforced, would
bend excessively under the load of wire tension.  So, as in the case of
the R1 chambers, the R2 endplates were a composite structure with
stainless steel stifferners.  Fig.~\ref{dcr2-endplate} shows an assembly drawing of an R2 endplate.

%%%%%%%%%%%%%%%%%%%%%% Figure : generic chamber sketch %%%%%%%%%%%%%%%%%%%%%%
\begin{figure}[hbpt]   
\vspace{10cm}
\begin{picture}(35,35)
\put(0,0)
{\hbox{\includegraphics[width=1.0\columnwidth,natwidth=610,natheight=642]{img/dcr2-endplate.png}}}
\end{picture}
\caption{\small{A R2 endplate assembly.}}
\label{dcr2-endplate}
\end{figure}   
%%%%%%%%%%%%%%%%%%%%%%%%%%%%%%%%%%%%%%%%%%%%%%%%%%%%%%%%%%%%%%%%%%%%%%%%%%

It also allows 
the trumpets that position the wires to be essentially flush with the endplates, 
rather than having to insulate the trumpets from the conducting endplates as in 
the other two Regions (see Fig.~\ref{dc-corner}).  This reduced the thickness of 
the inactive region by 1 to 2~cm.




 



\subsection{Region Three Construction}

\hskip 0.15 in
The R3 chambers were designed and constructed at Jefferson Laboratory.  
They had the same general shape as the other chambers but were larger,
4 m on a side so the wires were as long as 4 m.
To reduce the gravitational sag of these very long wires we
strung them at 40 grams, twice the nominal tension of 20 grams.

%%%%%%%%%%%%%%%%%%% Figure : Region Three Cross Section %%%%%%%%%%%%%%%%%%
\begin{figure}[htpb]
\vspace{7.9cm}
\caption{\small{Assembly drawing of a R3 chambers showing the component
parts and highlighting the Carbon fiber tubes at the entrance face and
the carbon-foam composite plate at the exit, which supported the endplates
agains the wire tension.}}
\label{r3_cut}
\end{figure}
%%%%%%%%%%%%%%%%%%%%%%%%%%%%%%%%%%%%%%%%%%%%%%%%%%%%%%%%%%%%%%%%%%%%%%%%%%%

Because these are the final tracking chambers, multiple scattering
at the chamber entrance is not as important as multiple scattering that
occurs at a R1 or a R2 chamber, for example.  This allowed us to 
build a chamber in which the endplates were not supported only on
their ends.  At the entrance face we included 7 thin-walled Carbon
fiber tubes to span the gap and hold the endplates apart.  At the
exit face the endplates were coupled to a triangular carbon-foam-carbon
composite plate which similarly supported the wire tension.







