\section{Drift Chamber Tracking System Performance}
In this section, we document the drift chamber system's
efficiency at reconstructing charge particle tracks and
its spatial resolution of such tracks.




\subsection{Chamber Lifetime}
\label{choper}

\hskip 0.15 in 
In order to satisfy the statistical requirements of the experimental program, 
a critical design goal for CLAS is the ability to make routine measurements 
with electron beams giving luminosities of up to 
1$\times$10$^{35}$~cm$^{-2}$s$^{-1}$.  The tolerable luminosity in CLAS is 
generally set by the large flux of M{\o}ller electrons and low-energy photons 
produced from the targets by the multi-GeV incident electron beam.  This 
constraint is severe for the drift chambers since they are close to the 
target. Particularly for the R1 chambers, the large flux of 
particles limits the luminosity in two ways.  First, a large number of 
background-related hits in the chambers reduces the track reconstruction 
efficiency.  The working assumption is that the analysis software can tolerate 
no more that a 5$\%$ accidental chamber occupancy in order to successfully 
reconstruct charged-particle tracks.  Second, the effects of sustained high 
luminosities are unfavorable for long chamber lifetimes.  Aging correlates 
directly with the currents generated in the chambers.

We have not yet observed any sign of aging in the drift chambers.
For the previous CLAS chambers, we used the same gas mixture and ran at a
similar gain and the chambers lasted more than 10 years with no indication
of aging.  We expect the present
chambers to perform well for at least 10 years.


\subsection{Tracking Efficiency}

\hskip 0.15in
The probability of reconstructing a track due to a charged particle within
our fiducial volume is referred to as the ``tracking efficiency''.

The tracking inefficiency has three root causes:
\begin{enumerate}
\item {\bf intrinsic layer inefficiency}: the failure to record
a hit for a track crossing a layer, when all wires and electronics
are operating properly
\item {\bf malfunction-related inefficiency}: loss of hits and sometimes
whole track segments because of equipment malfunctions
\item {\bf background-related inefficiency}: out of time background
can interfere with the segment-finding algorithms when a background-related
track segment lies ``on top'' of a real, in-time, segment
\end{enumerate}





\subsubsection{Simulation of Inefficiencies}
In our generation and reconstruction of simulated events, we estimate the size of
the three types of inefficiency in the following manner:

\begin{enumerate}
\item {\bf simulation of intrinsic layer inefficiency}: this is a random process
and, as such, it is handled at event generation time by our GEMC simulation
program.  For each superlayer (1-6), we have defined a DOCA-dependent
layer inefficiency function, as determined from the data.  At hit-generation
time in GEMC a random number (between 0 and 1) is generated, and if it is
smaller than the layer inefficiency function, the hit is not digitized.
\item {\bf simulation of malfunction-related inefficiency}: the GEMC Monte
Carlo hits are generated as if there are no malfunctions of the wires.
During the Monte Carlo reconstruction, however, a status table for each
hit wire is queried and if the wire is in the ``bad status'' list, that
hit is not used in the tracking.  The malfunction-related inefficiency  is small.  
At the time of publication,
roughly $0.5\%$ of our wires are not operating properly.  
\item {\bf simulation of background-related inefficiency}: rather than try
to simulated out-of-time background due to all physics processes, we merge
``random-trigger'' events with low-luminosity runs and calculate the
efficiency of these merged events with un-merged low-luminosity events.
\end{enumerate}

We will not further discuss the {\bf malfunction-related inefficiency} or
the {\bf background-related inefficiency} further in this article.  See
our companion article on {\bf Track Reconstruction} for more details.

Now we present our results on measuring the {\bf intrinsic layer  inefficiency}.

\subsubsection{Intrinsic Layer Inefficiency}
The layer efficiency is the probability that a
good hit is recorded in a wire layer through which the track has passed, based on 
the evidence from all other layers in the superlayer.  This is called the 
``excluded-layer method''.  The layer efficiency is a measure of the intrinsic drift 
cell efficiency for the particular choice of gas mixture, high-voltage set point, and 
discriminator level.  

In the section on operations and monitoring we discussed our `high voltage plateau'.
We took data at successively higher values of high voltage and measured the
layer efficiency at each point.  We set the high-voltage operating points such that
the average layer efficiency is $>$98$\%$. 

The single layer inefficiency is not uniform across the drift cell.  It is higher near the sense wire and also near the outer
edge of the cell.  A track passing close to a sense wire leave many ions in the cell, but the ion arrival times are stretched
out from near-zero to the maximum drift time, Tmax.  The result is that the pre-amplifier's output signal has a low voltage
amplitude but persists for a long time.  So, even though the collected charge is large the voltage put out by our trans-impedance
pre-amplifiers may not be large enough to exceed the voltage discriminator threshold of the DCRB,
For the case of tracks near the outer edge of the cell (so-called `corner-clippers') they simply leave a very small number
of ions in the cell and thus have a small signal.

Fig.~\ref{hit-effcy-vs-doca} reveals that a large contribution to the $2\%$ inefficiency comes 
from tracks that pass close to the sense wire.  These tracks give rise to signals 
that have low pulse height and long duration, and thus may escape detection.
We fit this observed DOCA-dependent inefficiency to a functional form which is
used in our GEMC Monte Carlo hit digitization routine to randomly throw out
this percentage of hits.
%%%%%%%%%%%%%%%%%%%%%%%%%%%%%%%%%%%%%%%%%%%%%%%%%%%%%%%%%%%%%%%%%%%%%%%%%
\begin{figure}[htbp]
\vspace{5cm}
\begin{picture}(50,50)
\put(-10,10)
{\hbox{\includegraphics[width=0.35\textwidth,natwidth=610,natheight=642]{img/hit-effcy-vs-doca.png}}}
\end{picture}
\caption{\small{The observed ``layer efficiency'' as a function of DOCA.  Hits from tracks
which pass close to or far from the wire have the hits spread out in time, and the resulting
voltage pulse from the pre-amplifiers may fail to cross the discriminator threshold, resulting
in a ``lost hit''.}}
\label{hit-effcy-vs-doca}
\end{figure}
%%%%%%%%%%%%%%%%%%%%%%%%%%%%%%%%%%%%%%%%%%%%%%%%%%%%%%%%%%%%%%%%%%%%%%%%%%%

\subsection{Chamber Resolution}

\hskip 0.15in
The intrinsic chamber resolution, or single-wire resolution, is a measure
of the uncertainty between the DOCA of the track and the distance as
calculated from the time of the wire hit.  The excluded-layer fitting method 
has been used to estimate the single-wire resolution within a given superlayer.  
Tracks fitted to all hits except those on the excluded layer are projected to 
determine the intercept in the excluded layer.  The fit residual is the difference 
between the fitted DOCA of the track and the value of DOCA as calculated from the 
time of the hit in the excluded layer.  The variance of this residual distribution 
is the quadratic sum of the single-wire resolution and the track position uncertainty 
at the excluded layer.  This variance over-estimates the single-wire resolution.
Since there are six layers per superlayer and layer 3 is excluded from the fit,
this amounts to a $10 - 15\%$ over-estimate.

Fig.~\ref{resolution-vs-doca} shows the width of the track-hit residual distribution plotted versus DOCA for 
each of the different chamber Regions.  The single-wire resolution worsens near the 
wire and also at the outer edge of the cell.  This arises due to finite cluster sizes 
due to the Poisson distribution of ion-pair production along the path of the primary ion 
near the sense wire along with time walk effects and the divergent nature of the electric
field lines near the field wire.  

The bottom sub-figure is a histogram of the tracks' DOCA distribution.  It is approximately flat as
expected, since the particle luminosity is approximately constant over the small solid
angle of a single drift cell.  It shows a dip at zero, which is a real effect and is
due to a spreading out of the arrival time of ionized electrons when the track is
very close to the wire as explained in the preceding section on intrinsic layer inefficiency.
%%%%%%%%%%%%%%%%%%%%%%%%%%%%%%%%%%%%%%%%%%%%%%%%%%%%%%%%%%%%%%%%%%%%%%%%%
\begin{figure}[htbp]
\vspace{15cm}
\begin{picture}(50,50)
\put(30,10)
{\hbox{\includegraphics[width=1.\textwidth,natwidth=610,natheight=642]{img/resolution-vs-doca.png}}}
\end{picture}
\caption{\small{Top sub-plot: Resolution plotted versus the track DOCA (cm).  Bottom sub-plot:  The 1-d plot
of track DOCA (cm).}}
\label{resolution-vs-doca}
\end{figure}
%%%%%%%%%%%%%%%%%%%%%%%%%%%%%%%%%%%%%%%%%%%%%%%%%%%%%%%%%%%%%%%%%%%%%%%%%%%

A more quantitative look at the resolution as a function of doca is given in Fig.~\ref{resid-vs-doca},
in this case for one particular superlayer (sector 4, superlayer 5).
Once again we plot the 2-d distribution of the fit residual (TRKDOCA - DOCA) vesus TRKDOCA.
Overplotted are line plots of the results of slice Gaussian fits to the distribution showing
the sigma of the fit (top line) and the mean of the fit (bottom line near zero).
Here you can quantitatively see that the average single-wire resolution in the middle 
portion of the cell is about 350~$\mu$m, with a whole cell resolution of about 450~$\mu$m. 
Looking at similar fits for all of the chamber, we conclude that the whole-cell 
average is about 310, 315 or 380~$\mu$m for R1, R2, and R3, respectively.  
%%%%%%%%%%%%%%%%%%%%%%%%%%%%%%%%%%%%%%%%%%%%%%%%%%%%%%%%%%%%%%%%%%%%%%%%%
\begin{figure}[htbp]
\vspace{8cm}
\begin{picture}(50,50)
\put(30,10)
{\hbox{\includegraphics[width=.8\textwidth,natwidth=610,natheight=642]{img/resid-vs-doca.png}}}
\end{picture}
\caption{\small{A plot of residual vs. TRKDOCA with Gaussian slice fits done, showing the sigma and
mean as dark lines.}}
\label{resid-vs-doca}
\end{figure}
%%%%%%%%%%%%%%%%%%%%%%%%%%%%%%%%%%%%%%%%%%%%%%%%%%%%%%%%%%%%%%%%%%%%%%%%%%%


\subsection{Conclusions}

\hskip 0.15in
The toroidal geometry of the CLAS spectrometer necessitated a particle-tracking 
system of unconventional design.  Design challenges and solutions include the following:

\vskip 10pt
\noindent
- The necessity to conceal inactive areas of the drift chambers within the
shadow regions of the torus cryostat resulted in very thin endplates and low-profile
wire connection schemes and on-board preamplifiers.

\noindent
- The toroidal shape of the magnet and the desire to have measurements before, within, 
and after the high-field region, resulted in the design of a `rod and ball' mounting scheme
which minimized dead areas and facilitates maintenance.

\noindent
- The fabrication of chambers that support large static wire tensions, but have thin 
endplates necessitated three endplate designs: aluminum stiffened with steel bars (R1),
G10 stiffened with steel bars (R2) and Carbon fiber plates filled with foam and reinforce
with Carbon-fiber posts and close-out plate (R3). 

\noindent
-The need for precise tracking in a system with non-saturated drift velocity 
(necessitated by the requirements of large drift distances, non-flammable gas mixtures 
and low-gain operation) resulted in a semi-automated calibration and monitoring software 
package.

\vskip 10pt
The CLAS drift chamber system has been in routine operation since Spring, 2017. 
The system has reached its design goals of 
high-luminosity operation (1$\times$10$^{35}$~cm$^{-2}$s$^{-1}$) in a 
high-flux electromagnetic reaction environment, with very good track 
reconstruction efficiency over a large range of angles and 
magnetic fields.  The single-wire efficiency is greater than $98\%$ and the
single-wire resolution is about 330~$\mu$m averaged over all drift distances and
all 18 chambers, with each chamber showing a characteristic 250~$\mu$m resolution 
in the cell center.

\vskip 10 pt

{\large{\bf Acknowledgments}}

\vskip 10pt

The authors wish to thank the crews of wire stringers and technicians who 
participated during the sector construction at the University of Pittsburgh,
Old Dominion University, and Jefferson Laboratory, as well as the support of 
the technicians involved with installation of the detectors into CLAS.  The
authors also thank S. Corneliussen for his careful reading of the draft.  This
work was supported in part by DOE contract DE-AC05-84ER40150, DOE grants 
DE-FG02-87ER40315, DE-FG05-94ER40859, DE-FG02-96ER40960, DE-FG02-96ER40980, 
and NSF grant NSF-PHY-9412479.

\vskip 10pt

\noindent
$^{\dagger}$Present address : Syncsort Inc, Woodcliff Lake, NJ 07695 USA.

\noindent
$^{\ddagger}$Present address : Los Alamos National Laboratory, Los Alamos, NM
87545 USA.


