\section{Reconstruction Framework and Tools}

Experimental physics data analysis in a collaborative environment has historically involved
a computing model based on self-contained, monolithic software applications
running in batch-processing mode.
This model, if not organized properly, can result in inefficiencies in terms
of deployment, maintenance, response to program errors, update propagation,
scalability and fault-tolerance.
Over time, experimental configurations have become more diverse and compute capacity
has expanded at a rate consistent with Moore's Law. As a consequence,
compute applications have become much more complex,
with significant interaction between diverse program components.  This can
lead to computing systems so complex and intertwined that the programs become
difficult to maintain and extend. Modifying a feature
in one class often involves changes in several other classes, thereby increasing the needed development
time and effort. What starts small and simple, inevitably grows into a complicated scheme of interconnected
algorithms. At some point, to prevent further growth in complexity, the software has to be shielded from
contributors, preventing essential software evolution and optimization. The only way to bypass this highly
guarded system is software fragmentation, making the physics data processing validation process quite
arduous.

We have developed a reactive, data-stream processing framework called CLARA
\cite{clara-2011,clara-service,framework,clara-2016} that addresses these issues by simply deploying a true
“divide and conquer” principle to the software application, and insures continuous evolution and contribution of
software applications. Nuclear physics data processing applications have a very long lifetime, and the ability to
upgrade technologies is therefore essential. Hence, data processing applications must be organized in a way
that easily permits the discarding of aged software components and the inclusion of new ones without having to
redesign entire software packages at each change. The addition of new modules and removal of unsatisfactory
ones is a natural process of a software application evolution over time.  Software evolution and diversification
(e.g. using heterogeneous hardware structures, such as FPGAs, GPGPUs, etc.) is important and results in more
efficient and robust data processing applications.

Software applications designed within the CLAS12 CLARA data processing framework differ from ones developed adopting
a traditional approach in two major ways. First, software applications are composed of interlocking software building blocks
(bricks) called micro-services. Micro-services are linked together by data-stream pipes.  The technology
(e.g. high-level programming language, hardware deployment details) as well as the algorithmic solutions used to process
data (see Fig.~\ref{clara-overview}) are encapsulated. A micro-service has inputs, processes data and produces output data.
A micro-service reacts on a streaming data quantum in its input, processes it, and
passes processed data quantum to the next micro-service in the data-flow path. The second and critical difference is
that CLAS12 data processing applications execute according to the rules of data-flow instead of a more traditional
programming approach, where sequential series of instructions (lines of code) are written to perform a required algorithm.
In this respect CLAS12 data processing framework application design promotes a data-flow as the main concept
behind any data processing algorithm.   The flow of data between micro-services in the application determines
the execution order of micro-services within the application. These differences may seem minor at first, but
the impact is revolutionary because it allows the data paths between application building blocks to be the
application designer’s main focus. As a result, CLAS12 data processing application is more robust and agile, since
application building blocks can be improved individually and be replaced.

%%%%%%%%%%%%%%%%%%%%%%%%%%%%%%%%%%%%%%%%%%%%%%%%%%%%%%%%%%
\begin{figure*}
\centering
\includegraphics[width=0.8\textwidth]{pics/clara-overview.pdf}
\caption{Overview of the CLAS12 CLARA data processing framework that links the detector subsystem
event reconstruction applications to the data processing and data quality assurance monitoring applications
to the physics analysis applications.}
\label{fig:clara-overview}
\end{figure*}
%%%%%%%%%%%%%%%%%%%%%%%%%%%%%%%%%%%%%%%%%%%%%%%%%%%%%%%%%%%

\subsection{Common Tools}
\label{common-tools}

The offline software of the CLAS12 project aims at providing tools to
the collaboration that allow design, simulation, and data analysis to proceed
in an efficient, repeatable, and understandable way. As much as
possible, software-engineering-related details should be hidden from
collaborators, allowing them to concentrate on the physics.
To facilitate code development for the detector subsystems of CLAS12, and for detector-specific reconstruction output
analysis used for particle identification and overall event reconstruction,
the software was designed to provide libraries that are commonly used by all the reconstruction
packages.  These libraries, referred to as ``common tools'' provide means to avoid code replication and aid in software maintainability.

The common tools consist of various packages, each having a specific purpose and functionality. Below we discuss
the main packages used in the reconstruction software.

\subsubsection{Geometry}

Due to the complexity of the geometry of CLAS12 detector subsystems, an interface was developed
to provide classes and geometry software tools that are used to describe the geometry of all subsystem in an unified way.

A library of primitives was developed to provide geometrical objects needed to represent by all geometry detector subsystem
and for geometrical object transformations.  These primitives include lines, planes and various shapes (e.g.
cubes, trapezoids, etc.).  They provide the functionality needed
to represent the detector subsystem components in the reconstruction frames, to take into account geometrical distortions
and to translate and rotate detector components which is particularly relevant for alignment purposes.
Further more they provide methods to track particles through volumes, and return informations used to determine
track trajectories,
such as line to surface intersections, ray tracing through objects, distance of closest approach to a line or surface.

The CLAS12 geometry library insures consistency between the geometry implementation used in CLAS12 simulation, reconstruction and
event visualization packages. It is initialized through database, and provides output for GEANT4 simulations package
(GEMC), is used in reconstruction application and in the CLAS12 Event Display (CED) .

\subsubsection{Databases}

The Calibration Constant Data Base (CCDB) software package was developed at Jefferson Lab for the GlueX experiment
in HallD.  The CLAS12 reconstruction packages use the CCDB application programming interface to create and access
tables that contain detector geometry and calibration constants, as well as maps used for decoding raw data.
CCDB provides the functionality for storing and accessing
structured tables in MySQL-based and SQLite portable databases.

The CCDB package creates
tables that link constants to specific runs (using timestamps) and store different variations of constants depending on run
conditions. CLAS12 software tools employ an Application Programming Interface (API), that parses database tables and creates
structured maps of constants stored in  memory by detector sector, layer and component. This allows fast retrieving of the
constants.

The CLAS12 database access tools have been written to avoid bottlenecks that might result from multiple multi-threaded
services accessing the database to retrieve constants.  An interface has been designed to fetch the constants
on demand and cache them for further requests. In this approach each service will request the
constants it requires on one thread and each subsequent request by a new thread will be provided by the cached values.

\subsubsection{Plotting and Analysis Tools}

For ease of integration with the reconstruction software tools and packages, the plotting tools used for data
analysis, calibration and monitoring was written in the JAVA programming language.

The plotting software (called groot) developed at Jefferson Lab for CLAS12 is tailored to have a programming
interface similar to ROOT
(CERN data analysis package) and provides the functionality necessary for
fitting using the JAVA-based MINUIT library available from the JHEP repositories.

These same tools are used for analysis purposes. In addition, the analysis package
contains classes for four-vector manipulations and physics quantities extractions and
analysis methods (e.g. $Q^2, W$, boosts, etc.).

\subsubsection{Magnetic Field Package}
The magnetic field package (magfield) used by the CLAS12 reconstruction creates
binary field maps from engineering models of the CLAS12 torus and solenoid. It employs
a common self-described binary format, with a header containing meta-data describing
the pedigree of the field, its grid coordinate system, and the coordinate system
used by the field values. (For example, the CLAS12 torus has a cylindrical grid
but Cartesian field components.) The same magfield package provides the trilinear
interpolation of the field. Given that the field is often requested at a sequence
of points all contained within a single grid cell, magfield uses time-saving
software “probes” to cache nearest neighbors.

\subsubsection{Swimmer Package}
The swimmer package, in conjunction with the magfield package, is used by the CLAS12
reconstruction to integrate charged particles through the CLAS12 solenoid and torus.
It uses a fourth order (with 5th order corrections) adaptive step-size Runge-Kutta integrator
with single-step advancement achieved through a configurable Butcher tableau advancer.
There are a number of convenience methods for swimming to a plane, to the closest
point on a line, and to a specified value of a given coordinate such as z. For
forward swimming in CLAS12, the swimmer can reduce the dimensionality of the state
vector (and consequently run faster) by changing the independent variable from
the pathlength to the z coordinate.

