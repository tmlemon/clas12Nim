\subsection{Central Neutron Detector}

The Central Neutron Detector (CND)~\cite{cnd-nim} reconstruction service is designed to process raw ADC and TDC information and reconstruct clusters, defined as ensembles of counters matched in space and time. These objects constitute the output of the service that is passed to the Event Builder for particle formation and identification. The algorithm implemented in the service include three steps:

\begin{itemize}
\item{the reconstruction of the time and position of the hit in the paddle;}
\item{the reconstruction of the deposited energy;}
\item{the matching of CND hits with tracks coming from the interaction vertex.}
\end{itemize}

%The reconstruction makes use of the calibration constants of the CND summarized in
%Table~\ref{table_cnd_constants}.
%
%%%%%%%%%%%%%%%%%%%%%%%%%%%%%%%%%%%%%%%%%%%%%%%%%%%%%%%%%%%%
%\begin{table}[htbp]
%\begin{center}
%\begin{tabular}{|c|c|c|} \hline
%Constant Name & Number of Constants  & Units \\ \hline
%$t_{\rm{LR}}$ & 72 & ns\\ \hline
%v_{\rm{eff}}$ & 144 & cm/ns \\ \hline
%$u_{\rm{t}}$ & 72 & ns \\ \hline
%$t_{\rm{LR}_{\rm{ad}}}$ & 72 & ns \\ \hline
%$t_{\rm{off}}$ &72 & ns\\ \hline
%$A_{\rm{L}}$ & 144 & cm\\ \hline
%$MIP_{\rm{D}}$, $MIP_{\rm{I}}$ & 144 each & no units \\ \hline
%\end{tabular}
%\caption{The constants computed in the CND calibration.}
%\label{table_cnd_constants}
%\end{center}
%\end{table}
%%%%%%%%%%%%%%%%%%%%%%%%%%%%%%%%%%%%%%%%%%%%%%%%%%%%%%%%%%%%

%\subsubsection{Timing Calibration}
%
%The paddle in which the hit occurs must be determined before all other reconstruction steps. The raw hit
%times are obtained from the measured TDC channel using a slope constant of 0.0234~ns/channel for all
%channels.
%
%The left and right times of a hit in the left paddle (we label them as $t_{{\rm{L}}/{\rm{L}}}$ and
%$t_{\rm{R}/{\rm{L}}}$ where the first index corresponds to the paddle under exam, while the second indicates
%the paddle in which the primary hit happened) are given by:%
%
%\begin{equation}
%\label{eq_time_hit_lr}
%t_{{\rm{L}}/{\rm{L}}}=t_{\rm{off}}+t_{\rm{tof}}+\frac{z}{v_{\rm{eff}_L}}+t_{\rm{S}}+t_{\rm{off}_{\rm{L}}}+{\rm{TDC}}_{\rm{j}},
%\end{equation}
%
%\begin{equation}
%\label{eq_time_hit_lr1}
%t_{{\rm{R}/{\rm{L}}}}=t_{\rm{off}}+t_{\rm{tof}}-\frac{z}{v_{\rm{eff}_L}}+\frac{L}{v_{\rm{eff}_L}}
%+\frac{L}{v_{\rm{eff}_{\rm{R}}}}+u_{\rm{t}}+t_{\rm{S}}+t_{\rm{off}_{\rm{R}}}+{\rm{TDC}}_{\rm{j}},
%\end{equation}
%
%\noindent
%where $t_{\rm{tof}}$ is the time of flight, $z$ is the position of the hit measured from the upstream end of the
%paddle, $L$ is the length of the paddle, $t_{\rm{S}}$ is the start time of the event, $t_{\rm{off}_{\rm{L}}}$ and
%$t_{\rm{off}_{\rm{R}}}$ are time offsets associated to the left and right coupled paddles, and ${\rm{TDC}}_{\rm{j}}$
%is the TDC clock jitter. Similarly if the hit happened in the right paddle one can write:
%
%\begin{equation}
%  t_{{\rm{L}}/{\rm{R}}}=t_{\rm{off}}+t_{\rm{tof}}-\frac{z}{v_{\rm{eff}_{\rm{R}}}}+\frac{L}{v_{\rm{eff}_L}}
%  +\frac{L}{v_{\rm{eff}_{\rm{R}}}}+u_{\rm{t}}+t_{\rm{S}}+t_{\rm{off}_{\rm{L}}}+{\rm{TDC}}_{\rm{j}},
%\end{equation}
%
%\begin{equation}
%t_{R/R}=t_{\rm{off}}+t_{\rm{tof}}+\frac{z}{v_{\rm{eff}_{\rm{R}}}}+t_{\rm{S}}+t_{\rm{off}_{\rm{R}}}+{\rm{TDC}}_{\rm{j}}.
%\end{equation}
%
%Defining $\Delta$ and $\Delta'$ as:
%
%\begin{equation}
%\Delta=\frac{L}{v_{\rm{eff}_L}}-\frac{L}{v_{\rm{eff}_{\rm{R}}}},
%\end{equation}
%
%\begin{equation}
%\Delta'=t_{\rm{L}/{\rm{X}}}-t_{\rm{R}/{\rm{X}}}+t_{\rm{off}_{\rm{R}}}-t_{\rm{off}_{\rm{L}}},
%\end{equation}
%
%\noindent
%where the index X can be R or L, one can compute $\Delta-\Delta'$ for both cases (hit in the left paddle or hit
%in the right paddle). If the hit is in the left paddle:
%
%\begin{equation}
%\Delta'-\Delta= \frac{2z}{v_{\rm{eff}_L}} - \frac{2L}{v_{\rm{eff}_L}} -u_{\rm{t}} <0.
%\end{equation}

%\noindent
%If the hit is in the right paddle:
%
%\begin{equation}
%\Delta'-\Delta= \frac{2L}{v_{\rm{eff}_{\rm{R}}}}-\frac{2z}{v_{\rm{eff}_{\rm{R}}}} +u_{\rm{t}} >0.
%\end{equation}
%
%\noindent
%If $\Delta'<\Delta$, the paddle in which the hit happened is the left one, otherwise it is the right one.

\subsubsection{Hit Position and Time Reconstruction}

Hit position and time are reconstructed from the TDCs measured for the two PMTs at the ends of the given scintillator paddle. The algorithm is conceptually similar to the one applied to the TOF systems but includes some modification to account for the specific structure and geometry of the detector. The relevant equations are detailed in the following.

{\color{red} REWRITE Starting from $t_{\rm{L}}$ and $t_{\rm{R}}$, defined as the reconstructed hit time from the left and right PMTs,
and subtracting the time offsets obtained from the calibrations, the start time and the time jitter, one can
define the propagation times $t_{\rm{L}_{\rm{prop}}}$ and $t_{\rm{R}_{\rm{prop}}}$ as:

\begin{equation}
t_{\rm{L}_{\rm{prop}}}=t_{\rm{tof}}+\frac{z}{v_{\rm{eff}_{\rm{L}}}},
\end{equation}

\begin{equation}
t_{\rm{R}_{\rm{prop}}}=t_{\rm{tof}}-\frac{z}{v_{\rm{eff}_{\rm{L}}}}+\frac{L}{v_{\rm{eff}_{\rm{L}}}}
+\frac{L}{v_{\rm{eff}_{\rm{R}}}}+u_{\rm{t}}.
\end{equation}

The position of the hit is then obtained as the difference of the propagation times:

\begin{equation}
z=\frac{v_{\rm{eff}_{\rm{L}}}}{2} \left(t_{\rm{L}_{\rm{prop}}}-t_{\rm{R}_{\rm{prop}}}
+ L \cdot \left(\frac{1}{v_{\rm{eff}_{\rm{L}}}}+\frac{1}{v_{\rm{eff}_{\rm{R}}}}\right)  +u_{\rm{t}}\right).
\end{equation}

The $x$ and $y$ coordinates of the hit are obtained from the radius and the azimuthal angle of the hit, which
are, in turn, determined by knowing the layer, sector, and component (left or right) of the hit.  Finally, the time
of flight of the particle that produced the hit is obtained as:

\begin{equation}
t_{\rm{tof}}= \frac{1}{2}\left(t_{\rm{L}_{\rm{prop}}}+t_{\rm{R}_{\rm{prop}}}- L \cdot \left(\frac{1}{v_{\rm{eff}_{\rm{L}}}}
+\frac{1}{v_{\rm{eff}_{\rm{R}}}}\right)  -u_{\rm{t}}\right).
\end{equation}}

\subsubsection{Energy Reconstruction}
{\color{red} REWRITE 
For hits in the left paddle, the two associated ADCs can be written as:

\begin{equation}
\label{eq_rec_3}
ADC_{\rm{L}}=\frac{E_{\rm{L}}}{E_0}\cdot MIP_{\rm{D}}\cdot e^{\frac{-z}{A_{\rm{L}}}},
\end{equation}

\begin{equation}
\label{eq_rec_4}
ADC_{\rm{R}}=\frac{E_{\rm{R}}}{E_0}\cdot MIP_{\rm{I}}\cdot e^{\frac{-(L-z)}{A_{\rm{L}}}},
\end{equation}

\noindent
where $E_{L/R}$ is half the energy deposited by the particle in the left/right paddle and $E_0$ is given by
Eq.~\ref{eq_def_e0}.

\begin{equation}\label{eq_def_e0}
E_0=\frac{h\cdot 1.956}{2}~\rm{MeV},
\end{equation}

\noindent
where $h$ is the thickness of each scintillator. The above equations are valid for hits in the left paddles, while
for hits in the right paddles, the applicable equations are obtained by switching the $L/R$ indices. From
Eqs.~\ref{eq_rec_3} and \ref{eq_rec_4} follows the relations:

\begin{equation}
E_{\rm{L}}=\frac{ADC_{\rm{L}} \cdot E_0}{MIP_{\rm{D}}}\cdot e^{\frac{z}{A_{\rm{L}}}},
\end{equation}

\begin{equation}
E_{\rm{R}}=\frac{ADC_{\rm{R}} \cdot E_0}{MIP_{\rm{I}}}\cdot e^{\frac{L-z}{A_{\rm{R}}}}.
\end{equation}

\noindent
The total deposited energy is given by the sum of $E_{\rm{L}}$ and $E_{\rm{R}}$:

\begin{equation}
E_{\rm{dep}}=E_{\rm{L}}+E_{\rm{R}}.
\end{equation}

}

\subsubsection{Clustering algorithm}
%\subsubsection{Hit/Track Matching}
%Tracks from charged particles crossing the CLAS12 Central Vertex Tracker (CVT) are associated to hits in the CND. This allows, for each CND hit matched with a CVT track, to calculate the position of the hit from the extrapolated track, the pathlength between the track vertex and the hit and the path travelled in the hit paddle. This information is used in the calibration, as well as to veto charged particles when looking for neutrons in the CND. CVT tracks are extrapolated to radii corresponding to the entry point, middle point and exit point of the track in the paddle. These points are defined as the intersections between the helix of the track and cylinders of radii corresponding to the distances between the center of the CD and the three CND layers. A CVT track and a CND hit are matched if the hit coordinates ($x$, $y$, and $z$) and the extrapolated coordinates ($x_m$, $y_m$, and $z_m$) verify the relations:

%\begin{equation}
%\mid x-x_m \mid < \sigma_x ,~~~~\mid y-y_m \mid < \sigma_y , ~~~~z_m % \in [-\sigma_z,L+\sigma_z],
%\end{equation}

%\noindent
%where $\sigma_z=1.5$~cm, $ L$ is the length of a paddle, and $\sigma_x$ and $\sigma_y$ are given by:

%\begin{equation}
%\sigma_x= \sqrt{x^2\frac{\sigma_R^2}{R^2}+y^2\sigma_\phi^2},~~~~
%\sigma_y= \sqrt{y^2\frac{\sigma_R^2}{R^2}+x^2\sigma_\phi^2},
%\end{equation}

%\noindent
%where $R$ is the radius of the hit, $\sigma_R$ is half the thickness of a paddle (1.5~cm) and $\sigma_\phi$ is the azimuthal resolution of each paddle (3.75$^\circ$). The path travelled by the particle in the paddle is approximated as the distance between the entry and exit points. The path length between the vertex and the hit is given by the helix parameters.
