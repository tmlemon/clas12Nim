\section{Monitoring and Calibration Suites}
\label{sec:calibration}

\subsection{Framework}

A calibration framework was developed to implement visualization software tools needed for all detector
systems. Standard views were developed using the Java {\it Swing} application to visualize detector components
and to provide call-back mechanisms necessary to display detector-component specific information.  These
software tools provide functionality for data fitting, plotting, and displaying using a Graphical User Interface
(GUI) environment.

The calibration framework makes use of the other CLAS12 libraries (the geometry and plotting packages, as well
as database utilities) and provides a uniform GUI for all calibration applications. The framework provides a data
processing interface and a calibration constant database interface used for online and offline data analysis.

A common data streaming interface is implemented with software level abstraction that allows the calibration and
monitoring codes to run on variety of supported data formats used in CLAS12, including reading data in real-time
from the CLAS12 DAQ system~\cite{daq-nim}.

\subsection{Calibration and Monitoring Suites}

The software programs used for the CLAS12 detector subsystem monitoring, as well as the energy and time
calibrations,  are Java-based suites that employ the framework discussed in Section~\ref{common-tools}. The
software tools provided by the framework facilitate the development of detector-specific suites.
Figure~\ref{suites} shows representative views of the CLAS12 subsystem calibration suites.

\begin{figure*}
\centering
\includegraphics[width=0.9\textwidth]{pics/suites.png}
\caption{Representative subsystem calibration GUIs for the Electromagnetic Calorimeter (ECAL)~\cite{ecal-nim}
  (upper left), the Drift Chambers (DC)~\cite{dc-nim} (upper right), the Forward Time-of-Flight (FTOF)
  \cite{ftof-nim} (lower left),  and the Forward Tagger (FT)~\cite{ft-nim} (lower right).}
\label{suites}
\end{figure*}

The calibration applications take as input raw or reconstructed data files (from either beam data or Monte Carlo
simulations) in either EVIO or HIPO data formats.  They display and fit the various quantities and histograms
relevant to the extraction of the calibration constants.  The calibration analysis parameters are saved into ASCII
files with the same structure as the tables defined in CCDB.

\section{CLAS12 Event Display}
\label{sec:ced}

The CLAS12 Event Display ({\it ced}) is a diagnostic graphical application for displaying CLAS12 events. The
primary element of {\it ced} is the view, i.e. a graphical representation of CLAS12 in its entirety or a subset of
detector packages. For a given event, the primary purpose is to display the detector components that have recorded
a signal, and, if available, the reconstructed tracks, to provide a visualization of the particle passage through the
detector. In addition, {\it ced} can display information about the event such as the data banks, or information about
the detector, such as the magnetic fields. Available views are both 2- and 3-dimensional with the possibility of
disabling the latter for a faster execution. An illustration of views in {\it ced} is shown in Fig.~\ref{fig:dcTracks},
where a section of CLAS12 is displayed in a cut-view with specific focus on the Forward Detector. The colored areas
in the space around the detectors indicate regions where a significant magnetic field intensity is present from either
the solenoid or torus; reconstructed tracks are shown by the orange lines. Similarly, Fig.~\ref{fig:ced} shows views
of the Central Detector and of the Forward Tagger. Fig.~\ref{fig:ced}(left) shows two tracks originating from the
target as reconstructed from the fit of the available central tracker hits in correlation with signals in the outer
detectors. Here, the color scale is representative of the recorded signal intensity. Figure~\ref{fig:ced} shows a
front view of the Forward Tagger calorimeter for an event where three clusters were recorded. {\it ced} is
designed to be operated offline, reading either raw EVIO or HIPO events from a file, or online, reading events
from the CLAS12 DAQ system~\cite{daq-nim} to allow for real-time monitoring of the detector during data taking.

\begin{figure*}
\centering
\includegraphics[width=0.95\textwidth]{pics/dcTrack3.png}
\caption{Views from {\it ced} of charged particle tracks in the Drift Chambers showing cut-views to highlight
  different pairs of sectors of the six sectors of the CLAS12 Forward Detector. The colored detector elements
  are the registered hits and the orange lines are the result of track reconstruction using the hits in the Drift
  Chambers. The colored areas about the detectors represent the regions of magnetic field from the torus and
  the solenoid. In these views the beam is incident from the left and the target is located in the middle of the
  solenoid (for which just the downstream half is displayed here).}
\label{fig:dcTracks}
\end{figure*}

\begin{figure*}
\centering
\includegraphics[width=0.4\textwidth]{pics/ced_central.png}
\includegraphics[width=0.39\textwidth]{pics/ced_ft.png}
\caption{Views from {\it ced} of the Central Detector (left) and the Forward Tagger (right) from a view looking
  down the beamline. In the Central Detector view (left), two tracks originating from the target are shown as
  reconstructed from the fit of the available central tracker hits in correlation with signals in the outer detectors
  (Central Time-of-Flight (CTOF) and Central Neutron Detector (CND)). Here the color scale is representative of
  the recorded signal intensity. The right figure shows a front view of the Forward Tagger calorimeter for an event
  where three clusters were recorded.}
\label{fig:ced}
\end{figure*}
