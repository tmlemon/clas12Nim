\subsection{Threshold Cherenkov Counters}

The goal of the High and Low Threshold Cherenkov Counter (HTCC, LTCC) reconstruction algorithms is to
calculate the signal strength, time, and position from the
raw ADC signals coming from the PMTs through the corresponding FADC board. The algorithm takes into account
the properties of the HTCC and LTCC geometries, namely, the possibility for the signal from the single track to split into
up to four mirrors.
Hence, up to four separated signals (or hits) are produced. The final signal reconstruction is done in three
steps: decoding, hit reconstruction, and cluster reconstruction.

At the decoding stage, the signal is converted from the hardware notation (crate, slot, channel) into the CLAS12
notation (sector, layer, component).  For each signal, strength ($nphe_{hit}$) and timing ($T_{hir}$) is determined
from the position of the threshold crossing, fit to the shape of the signal and measured value of the pedestal.

At the hit reconstruction stage, individual signals in terms of the ADC channels are converted into number of photoelectrons ($nphe_{hit}$) for each hit using gain constants derived from the detector calibration and stored in CCDB:

\begin{equation}
nphe_{hit} = \frac{ADC}{gain}
\end{equation}

Geometry information on the PMT location is used to associate the angular coordinates ($\theta_{hit}$, $\phi_{hit}$) to the hit.

In order to reconstruct the real signal strength ($nphe_c$),  split signals (hits) have to be combined into a single
cluster. The algorithm starts by selecting the hit with the largest $nphe_{hit}$, which is used as seed for the cluster. Adjacent hits within a certain time window are then searched iteratively and, if found, added to the cluster. The total signal strength is determined as the sum of the individual signals, and the signal time is determined as the average between the individual signal times, weighted by the corresponding number of photoelectrons. The cluster
angular coordinates are determined as the average between the individual hits forming the cluster. The following equations summarize the definitions of the cluster coordinates:

\begin{eqnarray*}
nphe_c &= \frac{\sum_{i=1}^{N}{nphe_{hi}}}{N}\\
T_c &= \frac{\sum_{i=1}^{N}{N*T_{hi}}}{\sum_{i=1}^{N}{nphe_{hi}}}\\
\theta_c &=\frac{\sum_{i=1}^{N}{\theta_{hi}}}{N}\\
\phi_c &= \frac{\sum_{i=1}^{N}{\phi_{hi}}}{N}
\end{eqnarray*}

The clustering algorithm is run iteratively until the list of hits is
exhausted.

In the HTCC, the cluster coordinates, required for the matching of the hit with the reconstructed track in the Event Builder, are
reconstructed by projecting  ($\theta_c$, $\phi_c$) of the cluster on the surface of the ellipsoidal mirror of the detector. In the LTCC, an estimated cluster position is calculated based on a parameterization extracted from Monte Carlo simulations..  The track that passes the closest to the cluster position is then chosen as the match for this cluster.
