\subsection{Threshold Cherenkov Counters}

The goal of the High and Low Threshold Cherenkov Counter (HTCC, LTCC) reconstruction algorithms
calculate the signal strength, time, and position from the
raw ADC signals coming from the PMTS through the corresponding FADC board. The algorithm takes into account
the properties of the HTCC and LTCC geometries, namely, the possibility for the signal from the single track to split into
up to four mirrors.
Hence, up to four separated signals (or hits) are produced. The final signal reconstruction is done in three
steps: decoding, hit reconstruction, and cluster reconstruction.

At the decoding stage, the signal is converted from the hardware notation (crate, slot, channel) into the CLAS12
notation (sector, layer, component).  For each signal, strength ($nphe_{hi}$) and timing ($T_{hi}$) is determined
from the position of the threshold crossing, fit to the shape of the signal and measured value of the pedestal.

At the hit reconstruction stage, individual signals in terms of the ADC channels is converted into the number of
the photoelectrons ($nphe_{hi}$) for each hit using the gain constants in the $ccdb$:

\begin{equation}
nphe_{hi} = \frac{ADC}{gain}
\end{equation}

The correspondence between the  hit coordinate in terms (sector, layer, component) and specific PMT, observing
the mirror,  provides the information on the angular hit coordinates ($\theta_{hi}$, $\phi_{hi}$).

In order to reconstruct the real signal strength ($nphe_c$), we need to combine split signals (hits)into the single
cluster. We start by selecting the strongest hit and use it as a starting point of the cluster. Then we look for the
adjacent hits within a certain time window.  If such hits are found, they will be added to the growing cluster. If
not, the cluster growth stops.

The total signal strength is determined as the sum of the individual signals, and the signal time is determined as
the average between the individual signals, weighted by the corresponding number of the photoelectrons. The hit
angular coordinate is determined as the average between the individual hits forming the cluster.  Hits, attributed
to the established clusters, are removed from the further consideration, and algorithm starts to look for the next
cluster until list of hits is exhausted.

\begin{eqnarray*}
nphe_c &= \frac{\sum_{i=1}^{N}{nphe_{hi}}}{N}\\
T_c &= \frac{\sum_{i=1}^{N}{N*T_{hi}}}{\sum_{i=1}^{N}{nphe_{hi}}}\\
\theta_c &=\frac{\sum_{i=1}^{N}{\theta_{hi}}}{N}\\
\phi_c &= \frac{\sum_{i=1}^{N}{\phi_{hi}}}{N}
\end{eqnarray*}

Hits, attributed to the established clusters, are removed from the further consideration, and algorithm starts to
look for the next cluster until list of hits is exhausted.

In the HTCC the cluster coordinates, required for the matching of the hit with the reconstructed track in the Event Builder, is
reconstructed by projecting  ($\theta$, $\phi$) of the cluster on the surface of the ellipsoidal mirror.     In the LTCC an estimated cluster
position is calculated leveraging the Monte-Carlo simulation.  The track that passes the closest to the tentative true cluster position is then chosen as the true match for this cluster.
