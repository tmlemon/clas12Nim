\section{Performance}

Data taken during the first CLAS12 experiments, on hydrogen and deuterium targets, and with various beam energies (7.5 and 10.6 GeV), were analyzed to verify the performance of the CND. 

The timing performance for the three layers of the CND are illustrated in Fig.~\ref{fig_performance_deltat_layers}, showing the vertex time difference $v_t$ for selected negative tracks, integrated over all sectors defined as
\begin{equation}\label{eq_vtp_definition}
v_t=t_{\rm{CND}}-t_{\rm{S}}-\frac{path}{c \cdot \beta} ,
\end{equation}
where $t_{\rm{S}}$ is the event start time determined by the FTOF, $path$ is the path length from the event vertex to the CND and
\begin{equation}\label{eq_beta_definition}
\beta=\frac{p}{\sqrt{p^2+m^2}} .
\end{equation}
% e
The distribution of $v_t$ is centered at 0, and from its width, obtained with a Gaussian fit ($\sigma_{v_t}\simeq 243$ ps in average) one can deduce the average timing resolution for each PMT of the CND, convoluted with the CVT resolution, using the formula:
\begin{equation}\label{eq_resolution}
\sigma_t=\frac{\sqrt{\sigma_{v_t}^2-\sigma_{t_{S}}^2}}{\sqrt{2}}=157\, \rm{ps}, 
\end{equation}
assuming $\sigma_{t_{S}}=80$ ps \cite{ftof-nim}. This is compatible with the detector specifications (150 ps) and the result of the measurements in cosmic rays during the detector assembly phase \cite{Niccolai:2018qzm} (148 ps).

\begin{figure}[htb]  
\begin{center}
\includegraphics[width=0.45\textwidth]{Figure/canVTPlot.pdf}
\caption {Difference between the vertex time computed combining CND and CVT information and the start time computed by the FTOF, for negative tracks, for the three layers of the CND, integrated over all paddles. }
\label{fig_performance_deltat_layers}
\end{center}
\end{figure}

\begin{figure}[htb]  
\begin{center}
\includegraphics[width=0.45\textwidth]{Figure/VTsigma.pdf}
\caption {Timing resolution for each PMT of the CND, convoluted with CVT resolution.}
\label{fig_performance_vt_sigma_allpaddles}
\end{center}
\end{figure}

The position reconstruction performances of the CND can be checked in Fig.~\ref{fig_performance_deltaz}, which displays the difference between the $z$ coordinate computed by the CND and by the CVT, for negative tracks, and for the three layers of the CND, integrated over all paddles. Its Gaussian width is around 3 cm, corresponding roughly to $4^{\circ}$ in polar-angle resolution. This corresponds to the convolution of the angular resolutions of CND and CVT. 

\begin{figure}[htb]  
\begin{center}
\includegraphics[width=0.45\textwidth]{Figure/canZ.pdf}
\caption {Difference between the $z$ coordinate computed by the CND and the CVT, for negative tracks and for the three CND layers integrated over all paddles. }
\label{fig_performance_deltaz}
\end{center}
\end{figure}

Figure~\ref{fig_performance_edep} shows the energy deposited divided by path length, for selected MIPs. It peaks at around the expected value of 1.956 MeV/cm.

\begin{figure}[htb]  
\begin{center}
\includegraphics[width=0.45\textwidth]{Figure/canENE.pdf}
\caption {$dE/dx$ for MIPs in the three layers of the CND, integrated over all sectors. The blue line indicates the nominal value for the expected energy deposit of a MIP in a centimeter of plastic scintillator.}
\label{fig_performance_edep}
\end{center}
\end{figure}

\subsection{Neutron detection efficiency}
The exclusive reaction $e p \rightarrow e n \pi^+$ was analyzed to evaluate the neutron detection efficiency of the CND. The data used were taken with a 7.5-GeV electron beam inpinging on a liquid-hydrogen target. Events with an electron and a $\pi^+$ in the CLAS12 Forward Detector were selected. The missing mass of the $e \pi^+ X$ system is plotted versus $\beta_X$ in Fig.~\ref{fig_performance_selection}. The missing particle is required to be in the CLAS12 Central Detector (CD) ($\theta>40^\circ$). The effect of this selection is shown in Fig.~\ref{fig_performance_selection}. We apply an additional cut on $\beta$ of the missing neutron ($0.2<\beta_X<0.8$). From this set of $e p \rightarrow e (n) \pi^+$ events, events with a neutron identified by the CND (CND cluster with $E_{dep}>2.5$ MeV, no associated tracks, $\beta<0.8$) were selected. If multiple neutron candidates were detected by the CND, the neutron with the smallest momentum separation from the missing neutron was kept. A cut on $\beta>0.2$ was applied to remove out-of-time hits that could be mistaken as low $\beta$ neutrons. Finally, the detected neutron and the missing neutron azimuthal angle difference is constrained to be less than $20^\circ$.

The efficiency was measured in bins of missing neutron polar angle and as a function of missing momentum. For each bin in polar angle and momentum, the efficiency is defined as the ratio of events with a detected neutron to the number of missing neutron events. The result is shown in Figure \ref{fig_performance_efficiency}. The detection efficiency extracted from this method is consistent with simulation predictions and with the design specifications of the CND.

\begin{figure}[htb]  
\begin{center}
\includegraphics[width=0.5\textwidth]{Figure/SelectionPlots.pdf}
\caption {Missing mass $M_X$ versus $\beta_X$ of the $e p \rightarrow e\pi^+X$ reaction. The outer plot shows the effect of selecting events where the missing particle $X$ is emitted in the CLAS12 Central Detector.}
\label{fig_performance_selection}
\end{center}
\end{figure}

\begin{figure}[htb]  
\begin{center}
\includegraphics[width=0.5\textwidth]{Figure/newEfficiency.pdf}
\caption {Neutron detection efficiency of the CND as a function of momentum for three bins in polar angle.}
\label{fig_performance_efficiency}
\end{center}
\end{figure}

