\subsection{Validation of the Photoproduction Trigger}

As described in Section \ref{sec:photoproduction_trigger}, the CLAS12 photoproduction trigger requires a coincidence between one electron measured in the Forward Tagger detector and two hadrons measured in the CLAS12 detector in either the forward or central part. The validation procedure aims to verify if, for a given event foreseeing one final-state electron in the FT acceptance and two or more hadrons within the CLAS12 acceptance, whether the Trigger System would recognize it properly, resulting in event readout. In order to validate the system with beam during commissioning, the following strategy was adopted. First, the electron detection by the FT was validated using Random Trigger runs. After this, the detection of single hadrons in CLAS12 was studied in special runs, where the only trigger source was the FT. Finally, the coincidence between the two systems was assessed. Full details are described below.

\subsubsection{Validation of Electron Detection in FT}

A scattered electron in the FT is identified as an electromagnetic shower in the FT Calorimeter (FT-Cal) within a defined energy range, in time coincidence and geometrically matched to a hit in both layers of the FT Hodoscope (FT-Hodo). The map providing the matching between the cluster seed position in the FT-Cal and the tile position in the FT-Hodo was first derived from the nominal detector geometry, and then confirmed by Monte Carlo simulations.

The identification of the scattered electron in the FT was validated through a similar procedure as the one adopted for the CLAS12 electron trigger discussed in Section \ref{electron_trigger_validation} based on Random Trigger runs. The recorded events were processed through the standard CLAS12 reconstruction software and filtered, keeping only those with a reconstructed electron in the FT system. Since event readout was triggered by a random pulser, events with the reconstructed electron signal close to the margins of the readout window were also rejected. For these events, the electromagnetic clusters found by the reconstruction software (``offline'' clusters) were compared to those reported by the Trigger System and stored in the trigger data banks.

The efficiency of the FT-Cal clustering algorithm in the Trigger System was evaluated by comparing all ``offline'' clusters to those matched - in space and time - to the ``online'' clusters \footnote{The energy of ``offline'' clusters is properly corrected to account for electromagnetic shower leakage from the back of the FT-Cal, while ``online'' clusters do not have these corrections. Therefore, for a given electron in the FT-Cal, there is a systematic difference between the two energies. This effect is properly taken into account when setting the energy range for electron detection in the Trigger System, and does not affect the corresponding trigger efficiency.}. The efficiency was computed as:
\begin{equation}
\varepsilon=\frac{N_{trigger}}{N_{all}} \; ,
\end{equation}
where $N_{all}$ and $N_{trigger}$ are, respectively, the total number of ``offline'' clusters and the number of ``offline'' clusters matched to an ``online'' cluster. The result is shown in Fig.~\ref{fig:FT_ClusterEfficiency}, reporting the FT trigger efficiency for electromagnetic clusters as a function of the corresponding corrected energy. The efficiency is higher than 99.5$\%$ over the full energy range of interest. This small inefficiency is mainly due to the fact that the clustering algorithm in the Trigger System works on a 3x3 matrix of crystals, whereas this limitation doesn't hold in the offline reconstruction.
The efficiency of the FT-Cal/FT-Hodo matching algorithm was evaluated in a similar way, repeating the previous calculation but considering only electromagnetic clusters associated with one hit in each FT-Hodo layer. 
The result is reported in Fig.~\ref{fig:FT_ClusterEfficiencyHODO}: the efficiency of the matching algorithm is higher than 96.5$\%$  over the full energy range of interest. The inefficiency is mainly due to the difference in the FT-Cal / FT-Hodo geometrical matching map between the Trigger System and the offline reconstruction. The latter map is broader, hence there are some clusters associated with FT-Hodo hits in the offline reconstruction that are not matched in the online system. 

\begin{figure}[!htb]
 \centering
{\includegraphics[width=.5\textwidth]{img/FT_ClusterEfficiency.pdf}}
 \caption{FT-Cal trigger efficiency as a function of the measured electromagnetic cluster energy.}
 \label{fig:FT_ClusterEfficiency}
\end{figure}

\begin{figure}[!htb]
 \centering
{\includegraphics[width=.5\textwidth]{img/FT_ClusterEfficiencyHODO.pdf}}
 \caption{FT-Cal+FT-Hodo trigger efficiency as a function of the measured electromagnetic cluster energy.}
 \label{fig:FT_ClusterEfficiencyHODO}
\end{figure}

\subsubsection{Validation of Charged Hadron Detection in CLAS12 Forward Detector}
\label{valid_hadron_forward_detector}

The Trigger System recognizes a charged hadron in the CLAS12 Forward Detector as a hit in the FTOF (\cite{ftof-ref}) (panel-1b) in time coincidence and geometrically matched to a hit in the U-strips of the Preshower Calorimeter \cite{ec-ref} associated with a cluster with energy larger than a programmable threshold. The map providing the geometrical matching between the FTOF counter and the PCAL U-strip was first derived from the nominal detector geometry, and then confirmed by Monte Carlo simulations. To reduce the rate of random coincidences, the Trigger System also requires the presence of a segment in 5 out of 6 Drift Chamber superlayers in a given CLAS12 sector. The charged hadron identification algorithm was validated in special data-taking runs in which the Forward Tagger was the only enabled event readout source. In these runs, the Trigger System was configured to report in the output trigger bank the presence of a charged hadron in any CLAS12 sector, as defined in Section~\ref{sec:trigger_in_datastream}. 

The recorded events were processed through the standard reconstruction software and filtered, keeping only those with a well reconstructed forward charged track measured in the CLAS12. The track was required to be within the nominal acceptance of the CLAS12 PCAL, and a momentum threshold of 300~MeV was applied. The Trigger System efficiency was evaluated by comparing all reconstructed tracks to those recognized by the Trigger System. During commissioning, the efficiency was evaluated as a function of different observables, such as the energy deposited in the FTOF counters and in the PCAL, and the topology of the geometrical matching window. The trigger parameters were individually tuned to maximize the trigger efficiency. In the final configuration, an energy threshold of 2~MeV and 10~MeV for the FTOF counters and PCAL clusters, respectively, was selected. The result is reported in Fig.~\ref{fig:FD_TrackEfficiency}, showing the CLAS12 Forward Detector trigger efficiency for charged hadrons as a function of the track momentum. The efficiency is larger than 99$\%$ in the full momentum range, with the inefficiency dominated by threshold effects for the PCAL clusters.

\begin{figure}[!htb]
 \centering
{\includegraphics[width=.5\textwidth]{img/FD_TrackEfficiency.pdf}}
 \caption{CLAS12 Forward Detector trigger efficiency as a function of the measured track momentum.}
 \label{fig:FD_TrackEfficiency}
\end{figure}

\subsubsection{Validation of Charged Hadron Detection in CLAS12 Central Detector}

The Trigger System recognizes a charged hadron in the CLAS12 Central Detector as a cluster in the CTOF (\cite{ctof-ref}). To maximize the trigger efficiency for low momentum tracks, no coincidence with other detectors, such as the CND (\cite{cnd-ref}), was implemented. As for the CLAS12 Forward Detector case, the CLAS12 Central Detector charged hadron identification algorithm was validated in special data-taking runs in which the FT was the only enabled event readout source. 

The recorded events were processed through the standard reconstruction software and filtered, keeping only those with a well reconstructed charged track measured in the CLAS12 Central Detector. The track was required to be associated with a CTOF hit, with energy deposition larger than the corresponding threshold of 1 MeV. The result is reported in Fig.~\ref{fig:CD_TrackEfficiency}, showing the CLAS12 Central Detector trigger efficiency for charged hadrdons as a function of the track momentum. At low momenta (${\sim}$1~GeV), where most of the tracks are concentrated, the trigger efficiency is ${\sim}$99$\%$

\begin{figure}[!htb]
 \centering
{\includegraphics[width=.5\textwidth]{img/CD_TrackEfficiency.pdf}}
 \caption{CLAS12 Central Detector trigger efficiency as a function of the measured track momentum.}
 \label{fig:CD_TrackEfficiency}
\end{figure}

\subsubsection{Full Trigger Efficiency}
As described in Section~\ref{sec:photoproduction_trigger}, the two main CLAS12 trigger configurations for photoproduction measurements are: (i) the coincidence between one charged cluster in the FT-Cal with two tracks in different CLAS12 Forward Detector sectors and (ii) the coincidence between one charged cluster in the FT-Cal, one track in any CLAS12 Forward Detector sector, and one track in the CLAS12 Central Detector. The corresponding trigger efficiency was evaluated using data from special data-taking runs in which the FT was the only enabled event readout source. The recorded events were processed through the standard reconstruction software and filtered, keeping only those with two forward tracks or one forward track and one central track. The efficiency was evaluated as the ratio between the number of events with the trigger bit set and the total number of events. The obtained efficiency is $\sim$97.5$\%$ for both configurations. 
