
\section{Validation of stage1 triggers using simulations}
Almost all of stage one trigger components, before deploying to the production firmware, were tested on GEANT4 simulated data.

\section{Validation of electron trigger based on ``Beam ON" data}
The ultimate validation of the trigger is done using the so called ''Random Trigger" (RT) runs.
RT runs are special runs, where event readout is initiated not by the trigger logic, but by an external random generator, that
can be tuned on the desired frequency. Most of events in RT runs will not contain any tracks, or other useful information, however,
small fraction of events will have real reconstructed particles which were reconstructed because accidentally detector's response
signal to the particle felled in the readout window that was initiated by the random generator.
Int the event readout in addition to various detector signals, the trigger decisions are stored as well (see section {\color{Red} XX, somewhere above
it should be described, how trigger decisions are made, and what is the clock cycle for trig decisions }).

We want to use these accidental ``Good" events, and check whether corresponding trigger bit is set by trigger logic ({\color{Red} I assume
trigger bits will be described above}).
%%%%%%%%%%%%%%%%%%%%%%%%%%%%%%%%%%%%%%%%% F I G U R E %%%%%%%%%%%%%%%%%%%%%%%%%%%%%%%%%%%%%%%%%%
\begin{figure}[!htb]
 \centering
 \subfloat[]{\includegraphics[width=0.24\textwidth]{img/PCal_Fiducials_4878.png}}
 \subfloat[]{\includegraphics[width=0.24\textwidth]{img/ECin_Fiducials_4878.png}}
 \caption{Distribution of cluster coordinates of PCal (left) and EC\_{in} (right).
 scatter plot in Red shows all events, while blue scatter plot show events where cluster
 is in the fiducial region of the calorimeter (about 15 cm away from the edges).}
\end{figure}
%%%%%%%%%%%%%%%%%%%%%%%%%%%%%%%%%%%%%%%%% F I G U R E %%%%%%%%%%%%%%%%%%%%%%%%%%%%%%%%%%%%%%%%%%

The technique of the trigger validation is the following,
Analysing RT runs, we select 
