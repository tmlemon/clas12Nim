\section{Hardware Implementation}

The CLAS12 Trigger System is implemented using High Speed Serial (VXS) techniques for a complete fully pipelined multi-crate trigger system that takes advantage of the elegant high speed VXS serial extensions for VME.  This trigger system includes three stages starting with the front end VXS crate Trigger Processor (VTP), a sector-level Sub-System Processor (SSP) and a global VTP processor and Trigger Supervisor that manages the timing, synchronization and front end event readout.  
Within a front end crate, trigger information is gathered from each 16 Channel, 12bit Flash ADC module at 4nS intervals via the VXS backplane, to a Trigger Processor (VTP).  Each Trigger Processor is capable of handling these 500MBps VXS links from the 16 FADC modules, and then performs real time crate level trigger algorithms.  The crate trigger processor transmits the Level 1 trigger information through multiple Gigabit transceivers that are combined into a fiber link.  The VTP uses a multi-fiber link to increase the aggregate trigger data transfer rate to the global trigger to 10Gbps.
The front end crate trigger data is transmitted on the VXS backplane, and on the multi-fiber link using the Aurora protocol from Xilinx.  The front end modules use Virtex-V devices with Gigabit Transceivers operating at 2.5Gbps. The VXS Trigger Processor collects these serial streams with a Virtex7 device and works with a Zynq7 processor to manage the network interface and on-board Linux operating system.
The entire trigger system is synchronous and operates at 250Mhz with the Trigger Supervisor managing not only the front end event readout, but also the distribution of the critical timing clocks, synchronization signals, and the global trigger signals to each front end readout crate.  These signals are distributed to the front end crates on a separate fiber link and each crate is synchronized using a unique encoding scheme to guarantee that each front end crate is synchronous with a fixed latency, independent of the distance between each crate.  The overall trigger signal latency is <=8uS, and the CLAS12 experiments at Jefferson Lab require up to 20kHz trigger rate with can be easily handled since hardware has an ability to run up to 200KHz trigger rate.


\subsection{VTP board}

\begin{figure}[hbt]
	\centering
	\includegraphics[width=1.0\columnwidth,keepaspectratio]{img/vtp_board.png}
	\caption{VXS Trigger Processor (VTP)}
	\label{fig:vtp_board}
\end{figure}

The VXS Trigger Processor (VTP) Fig.~\ref{fig:vtp_board} is a VXS switch card. It is used to implement trigger logic on the front-end crates (Stage 1) and global trigger crate (Stage 3). There are 80 full-duplex serial links each capable of running at up to 8.5Gbps that can be used for transporting trigger data. The links are bonded in groups of 4 for a total of 20 channels: 16 VXS payload slot interfaces (copper) and 4 QSFP interfaces (optical).

\paragraph{Front-end (Stage 1) Crate Processing}
The VTP in the front-end crate collects data from the VXS payload FADC250 and DCRB modules (and optionally from some of the QSFP links) where it aligns data in time for all links, and presents it to the detector specific trigger logic, which resides in a the XC7V550T FPGA. The trigger logic processes the data and produces and output trigger stream that is sent to the stage 2 trigger crate (and optionally to other stage 1 VTP modules) using up to 4 QSFP optical links. The QSFP optical links allow stage 1 trigger logic to use information from multiple stage 1 crates, which is required for some detectors that span multiple VXS crates (e.g. DC and FT subsystems). The QSFP optical links also allow multiple links to go to stage 2 when more bandwidth is needed (e.g. HTCC and CTOF).

\paragraph{Global Trigger (Stage 3) Crate Processing}
In the global trigger crate the VTP collects data from the VXS payload SSP modules. The SSP modules supply a stream of trigger bits for each sector (HTCC, FTOF, ECAL, PCAL, and DC) and also a stream of trigger bits for the central detectors (CTOF, CND, and FT). These sector and central trigger bit streams have already have performed timing, multiplicity, and geometry coincidence between detectors within the sector or central detectors. The stage 3 VTP allows defining the final (``global'') trigger bits (up to 32) that can be defined by different combinations for sectors, sector trigger bits, and central trigger bits. The 32 global trigger bit decisions are evaluated at 250MHz so that no additional jitter is introduced by this stage. These bits are sent to the TS using the high density LVDS front-panel output using a twisted pair ribbon cable.

\paragraph{Event Builder}
A Zynq FPGA is used on the VTP to run the standard CLAS12 CODA ROC component, which allow the VTP to be configured and readout the same as other VME/Intel based CODA components. Event data can be generated by the VTP which contain trigger decisions for both stage 1 and 3 components, which is used for understand the trigger efficiency. Additionally there is a large buffer (4GByte with 200Gbps bandwidth) and 40Gbps Ethernet interface that is intended for future upgrades of the front-end crate readout system, which would use the VTP and 40Gbps Ethernet for event readout rather than the VME interface. Fig.~\ref{fig:vtp_block_daq} shows the interfaces between the FPGAs, memory, network, VXS, and fiber modules.

\begin{figure}[hbt]
	\centering
	\includegraphics[width=1.0\columnwidth,keepaspectratio]{img/vtp_block_daq.png}
	\caption{VTP Block Diagram}
	\label{fig:vtp_block_daq}
\end{figure}


\subsection{SSP board}

\begin{figure}[hbt]
	\centering
	\includegraphics[width=1.0\columnwidth,keepaspectratio]{img/ssp_board.png}
	\caption{SubSystem Processor (SSP)}
	\label{fig:ssp_board}
\end{figure}

The SubSystem Processor (SSP) Fig.~\ref{fig:ssp_board} is a VXS payload card used to collect data from multiple front-end (Stage 1) crates. The SSP performs the stage 2 trigger processing by creating sector and central trigger bit decisions. Up to 16 SSP modules can be housed in a single VXS crate, but only 7 are currently needed: 6 for sector detectors, and 1 for central detectors. There are 36 full-duplex serial links each capable of running at up to 6.5Gbps that can be used for transporting trigger data. The links are bonded in groups of 4 for a total of 8 channels: 1 VXS switch slot interface (copper) and 8 QSFP interfaces (optical). All VXS and QSFP lanes run at 5Gbps (or 20Gbps per channel).

\paragraph{Stage 2 Trigger Processing}
Stage 1 optical data arrives at the SSP where it is aligned and processed through various algorithms to make the sector and central trigger bit decisions. There are 8 sector and central trigger bits (expandable to 32) that evaluated at 250MHz so that no jitter is introduced by this stage. These bits are sent to the stage 3 VTP using the VXS switch serial interface. As well as VTP, SSP has event builder allowing to readout trigger information and insert it into data stream.
