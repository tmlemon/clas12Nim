\section{Electronics}

The acquisition system is designed to readout 6000 channels of the forward station and 18000 channels of the barrel station. With the physics background as high as 20~MHz, the strip hit rates are about 60~kHz and 20~kHz in the forward detectors and in the barrel detectors respectively. The readout system is compliant with the CLAS12 requirements of the 20~kHz maximum trigger rate and provides sufficiently long data pipeline to cope with up to 16~$\mu$s trigger decision latency. Timing precision of a few ns is sufficient to limit the number of ghost hits compatible with the timing of trigger signals. Charge measurement with the 10-bit dynamic range is enough to cover the full span of the Micromegas detector signals and to discriminate accurately MIPs from noise.

\subsection{Readout system Architecture}
The extremely tight design of the CLAS12 Central Detector leaves a very narrow space between the MVT and its neighbor subsystems, as well as between the Micromegas detectors themselves.  In addition to the stringent space, the operational conditions of the tracker are harsh in terms of radiation and high 5T magnetic field. The low material budget is an obvious concern. Consequently, a readout architecture based on the off-detector front-end electronics has been adopted. Lightweight micro-coaxial cable assemblies with low 40~pF/m linear capacitance carry bare unamplified signals to the front-end units (FEU) housed in crates some 1.5--2 m upstream of the detectors. 

\begin{figure}[htb]
 \includegraphics[width=1.0\columnwidth,keepaspectratio]{images/electronics_fig1.png}
 \caption{MVT/FTT readout system.}
 \label{fig:mm-e_1}
\end{figure}

The front-end electronics are responsible for the amplification and shaping of the detector signals, for holding the latter in a pipeline waiting for trigger process to yield, for the digitization and compression of the selected event data and for their delivery to the back-end electronics. The back-end is responsible for data concentration event by event. It provides an interface with the CLAS12 event building system. It also ensures a fixed latency path between the CLAS12 trigger system and the FEUs. It receives the system clock and trigger from the CLAS12 trigger supervisor and synchronously conveys them to the FEUs over bidirectional optical links.

\subsection{The 64-channel Dream ASIC}
Depending on the type and size of the CLAS12 Micromegas detectors, the strip capacitances vary from 60 to 120~pf. The total capacitance seen by the front-end electronics input is even higher, up to 200~pF due to the contribution from the detector micro-coaxial cables. To achieve a comfortable signal to noise ratio (SNR) well above 10, the equivalent noise charge (ENC) of the detection chain should be $\sim2500~e^-$ for the $140-200$~pF range of the total input capacitance. By the time of development, none of the existing HEP ASICs could deliver the required performance while, in addition, sustaining up to 20~kHz readout rate and providing 16$~\mu$s deep trigger pipeline.

A new 64-channel ASIC, called Dream (for Dead-timeless Readout Electronics ASIC for Micromegas), has been developed \cite{DRM}. 

\begin{figure}[htb]
 \includegraphics[width=1.0\columnwidth,keepaspectratio]{images/electronics_fig2.png}
 \caption{Block diagram of the Dream ASIC.}
 \label{fig:mm-e_2}
\end{figure}

Each channel includes a charge sensitive amplifier (CSA) adapted to a wide spread of detector capacitances (up to 1nF) and four selectable charge measurement ranges (from 50 to 600fC), a shaper with programmable peaking times (from 75~ns to 1~$\mu$s) and 512-cell deep Switched Capacitor Array (SCA) used as the a trigger pipeline memory and a de-randomization buffer.

The input signals are continuously sampled and stored in the SCA at up to 50~MHz rate. Upon reception of the trigger signal, a programmable number of samples of all channels, corresponding in time to the event, is read-out serially through a differential analog buffer capable to drive an external ADC at up to 28~MHz frequency. The sampling is not stopped during the readout process allowing nearly dead-timeless operation.
Other features, such as ability to operate with both signal polarities, possibility to inject input signals directly in SCA memory bypassing the filter and/or CSA, integrated per-channel discriminators (useful to form trigger primitives), make the chip extremely versatile. The integrated circuit is manufactured in the AMS CMOS 0.35~$\mu$m technology and is encapsulated in the 128-pin LQFP square package with 1~mm side and 0.4~mm pitch.

\subsection{The 512-channel front-end unit FEU}
The FEU is a mixed analog-digital electronics board. The analog section comprises eight input connectors, protection circuits, Dream ASICs and an 8-channel flash ADC (Figure Figure~\ref{fig:mm-e_3}). The protection circuits are optional.  They are installed on the FEUs to protect the Dreams from the sparks of the standard Micromegas detectors. When working with the resistive detectors, the input channels of the Dream ASICs can be directly connected to the detector strips, improving signal to noise ratio. The protected or non-protected type of the FEUs are determined during their manufacturing.

\begin{figure}[htb]
 \includegraphics[width=1.0\columnwidth,keepaspectratio]{images/electronics_fig3.png}
 \caption{512-channel front-end unit.}
 \label{fig:mm-e_3}
\end{figure}

As described above, the pre-amplification, shaping, and trigger pipeline functionality is implemented in the Dream chips. The analog samples from the eight Dreams are digitized by an 8-channel 40~MHz 12-bit flash ADC AD9222 from Analog Device~\cite{ADC}. The eight serial streams of digital data are delivered to the onboard FPGA.
The digital section of the board comprises an xc6vlx75t-2-ff748 FPGA from the Xilinx Virtex-6 device family~\cite{XIL}, its configuration memory, a 2~Mbyte synchronous SRAM, small form-factor pluggable (SFP) transceivers, an onboard clock synthesizer and an auxiliary trigger interface circuitry.

The FPGA controls the Dream integrated circuits and the ADC, producing the sampling and readout clocks, as well as various required control signals. One of the SFP cages is populated with an optical transceiver module. It is used to establish a synchronous communication channel with the back-end electronics over a 2.5~Gbit/s link. In the downstream direction, the link encodes the 125~MHz system clock, trigger signals and fast synchronous commands.

Upon accepting the trigger signal, the FPGA reads the corresponding samples from the Dreams and optionally applies the following digital data processing steps. First, after serial to parallel conversion, the pedestals are equalized. Next, for each sample, the coherent noise affecting Dream inputs is estimated and subtracted on a chip-by-chip basis. This greatly improves the noise immunity of the MVT readout system. Finally, the per-channel zero suppression is performed. The retained samples describe the signal development in the channel. Fitting their values with a known function allows an accurate estimation of deposited charge and of signal timing. For each accepted trigger, the FPGA forms an event fragment from the retained channel data and delivers it to the back-end electronics via the optical channel. The optical channel is also used for run control parameters setting and monitoring.

The FEU is a 6U (266~mm) high, 220~mm deep and 5HP (25.4~mm) wide module. The thickness of its 12-layer PCB is 1.6~mm. It can be powered from either 4.3V or 5V source and consumes slightly less than 20~W when all eight Dream-s operate in their most power-hungry mode. The FEUs have been operated in up to 1.5~T magnetic field without any perceptible change of their power consumption or functionality.

\subsection{The back-end unit}
The back-end of the data acquisition system of the MVT is based on the Jefferson Lab standard VME/VXS hardware including a trigger interface (TI), a signal distribution (SD) and a sub-system processor (SSP) board and a crate controller single board computer (SBC)~\cite{nim:daq}. The flow of the trigger, data and control messages is presented on Figure~\ref{fig:mm-e_4}. 

\begin{figure}[htb]
 \includegraphics[width=1.0\columnwidth,keepaspectratio]{images/electronics_fig4.png}
 \caption{MVT back-end unit and dataflow.}
 \label{fig:mm-e_4}
\end{figure}

The TI receives a low jitter 250~MHz system clock and fixed latency trigger signals from the CLAS12 trigger supervisor. It also delivers to the trigger supervisor the status information (e.g. busy) of the MVT readout system. The physical layer interface is based on a parallel optic technology. The clock and trigger signals are delivered to the SD board over the VXS backplane. The SD board conveys properly delayed and aligned clock and trigger signals to the SSP boards. It also gathers their status information, combines and sends it to the TI board. These communications happen over the VXS backplane.

The SSP board was primarily designed to be a part of the hardware level trigger logic of the JLAB experiments. Given the massive resources it provides (notably a Virtex-5 TX150T Xilinx FPGA, 32 multi-gigabit transceivers (GTX) routed to the front panel, 4 Gbyte DDR2 memory) it was considered for the readout of the MVT front-end electronics. The SSP firmware has been modified to fit the needs of the MVT back-end unit (BEU).

An SSP can distribute the global system clock, trigger and synchronous commands to up to 32 FEUs. In practice, there are two back-end units each serving 24 front-ends. The trigger pulses and fast run control commands are broadcast synchronously to all FEUs over the synchronized fixed latency 2.5~Gbit/s links. The protocol between the FEUs and the BEU sets an 8~ns resolution on successive triggers and synchronous commands (125~MHz clock). However, the dispersion of their arrival times on the FEUs is well under 1~ns.

On each trigger, the SSPs timestamp the event with the synchronous 125~MHz clock and assign it the event counter value. The 48-bit timestamp along with the 60-bit event ID is used for local event building. This process implies gathering from all FEUs the event data packets belonging to the same event (matching timestamps and event IDs). Multi-event buffers, with programmable number of events, are constructed in the external DDR2 memory. Upon the request from the crate controller SBC, the content of the buffers are transferred to its memory over the VME64 backplane using the 2SST protocol. Transmission rates of $\sim$200~Mbyte/s are routinely achieved.

The SBC executes the data collection and the run control tasks within the CODA software framework~\cite{CODA}. It completes the data integrity checks performed in the SSP firmware, disentangles multi-event buffers, forms MVT events concatenating the FEU/SSP data with the corresponding TI data and sends them to the CLAS12 event builder over a 10~GB/s Ethernet link.

The MVT readout electronics is continuously surveyed by the CLAS12 detector monitoring system using the EPICS framework.

