\section{Performance}

The response of the LTCC to electrons and pions has been studied using experimental data from the spring 2019 period,
the first time that the LTCC sectors 3 and 5 were filled with the C$_4$F$_{10}$ gas.
At the time of this writing the data is not yet fully calibrated.

\subsection{LTCC Response to Electrons}\label{sec:elecResponse}

The electrons are selected using the reconstruction algorithms~\cite{recon-nim} that identifies electron tracks.
The LTCC response is calculated by checking whether the electrons produced a signal in the detector or not.
The electron momenta has been selected in the expected pion response range, between 3.5 and 8~GeV.
The criteria for event selections are:

\begin{itemize}
   \item  Electrons identified using the reconstruction event builder algorithm;
    \item Electrons must be within the fiducial volume of the LTCC.
\end{itemize}


The electron momentum spectrum before and after the requirement of an associated LTCC signal is shown
in \F{electronEfficiency}, along with a 0$^{th}$-order polynomial fit. The average efficiency of electrons is
94$\%$, below the expected efficiency of 99$\%$. This may be due to several reasons:

\begin{itemize}
\item the electron selection was not refined by using the calorimeters (due to the uncalibrated detector status)
\item possible impurities in the gas, which was not purifed by the recovery system (unavailable at that time)
\item the data analyzed was not calibrated at the time of this writing
\item the mirror overlaps and re-positinioning mentioned in \ref{sec:possibleInefficiency}.
\end{itemize}

\begin{figure}
	\centering
	\includegraphics[width=0.98\columnwidth,keepaspectratio]{img/electronMomenta.png}
	\includegraphics[width=0.98\columnwidth,keepaspectratio]{img/electronEfficiency.png}
	\caption{Top: the number of reconstructed electrons vs. momentum (GeV) before and
          after the requirement of an associated LTCC signal. Bottom: the LTCC efficiency to electrons is the
          ratio of the two distributions above. A 0$^{th}$-order polynomial fit gives an average of 94$\%$
          efficiency.}
	\label{fig:electronEfficiency}
\end{figure}

\subsection{LTCC Response to Pions}

The calculate the response of the LTCC to pions, reconstructed positively charged pion are selected
and a check is done on whether they produced a good signal in the LTCC detector. The positive pion selection
considers all positively charged particles that pass a neutron missing mass cut for the reaction
$ep \to e'\pi^+n$. The criteria for event selection are:

\begin{itemize}
    \item The electron selection described in Section~\ref{sec:elecResponse};
    \item Positive pion candidates are identified using the reconstruction event builder algorithm;
    \item The positive pion candidates must be within the LTCC fiducial volume.
    \item A neutron missing mass cut is applied between 0.9 and 1.05~GeV (see \F{neutronMM}).
\end{itemize}

The missing mass cut is shown in \F{neutronMM}. The positive pion candidates that satisfy the cuts are shown
in black and those with a good LTCC cluster associated with the track are shown in red.

\begin{figure}
	\centering
	\includegraphics[width=0.98\columnwidth,keepaspectratio]{img/neutronMM.png}
	\caption{The missing mass from the reaction $ep \to e'\pi^+X$, where the peak at the neutron mass
          between 0.95 and 1.05~GeV is selected for the efficiency analysis. The positive pion candidates that
          satisfy the cuts are shown in black. The pions associated with an LTCC signal are shown in red.}
	\label{fig:neutronMM}
\end{figure}

\begin{figure}
	\centering
	\includegraphics[width=0.98\columnwidth,keepaspectratio]{img/pionMomentum.png}
	\caption{The pion momentum distribution before and after the requirement of an associated LTCC signal. }
	\label{fig:pionMomentum}
\end{figure}

The momentum distribution of the pions is shown in \F{pionMomentum} for the all pions and for the pions
with an associated signal in the LTCC. The ratio, normalized by the electron efficiency of 94$\%$ found above
to account for other system inefficiencies, defines the LTCC pion detection efficiency as a function of
momentum. Figure~\ref{fig:pionEfficiency} shows the LTCC efficiency for pions normalized to that for electrons
to account for detector inefficiencies unrelated to the LTCC. The LTCC response starts around 50$\%$ near
the expected signal threshold, and rises with momentum as expected, given that the rise of the number of emitted
photons with momentum. A plateau of 88$\%$ is reached after
a momentum of 5~GeV. This is within range of an expectation of efficiency above 90$\%$. 

\begin{figure}
	\centering
	\includegraphics[width=0.98\columnwidth,keepaspectratio]{img/pionEfficiency.png}
	\caption{The LTCC pion efficiency as a function of momentum. This is the ratio of the curves in \F{pionMomentum}
				normalized by the average electron efficiency of 94$\%$ shown in \F{electronEfficiency}.}
	\label{fig:pionEfficiency}
\end{figure}
