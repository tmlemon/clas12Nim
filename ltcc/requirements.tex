\section{Requirements}


The LTCC requirements to satisfy an adequate pion/kaon discrimination include:


\begin{itemize}
	\item Maximizing the coverage in each of the six sectors up to an angle of 45$^0$;
	\item Minimizing the radiation length in the active area of CLAS12;
	\item Fitting the LTCC box in the available space between the Drift Chambers and the Forward Time-Of-Flight;
	\item Providing a reasonable response to pions.
\end{itemize}

The azimuthal and polar coverage was already achieved in the CLAS 6 GeV era and was not modified by the refurbishment.
The radiation length of the detector was minimized by placing the light collecting cones and photomultiplier tubes
in the regions obscured by the torus magnet coils. In the active area the window radiation length is 0.02$\%$.

The distance between the target and the LTCC was increased by about 2 m in CLAS12. This brought some of the passive
elements into the active area of the detectors behind the LTCC, namely the support structure of the mirrors, the Winstone
Cones, the PMT magnet shields, and the dectector walls.
The remaining requirements are addressed by the refurbishment and addressed in this paper.
