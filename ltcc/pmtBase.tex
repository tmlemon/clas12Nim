\subsection{Photo-multipliers dividers}

In the original readout electronics the LTCC single output from each PMT was amplified by a factor of 10
and then split in two to feed the ADC and TDC boards. This amplification and splitting was performed
by a dedicated electronic module (UVA 132) \cite{Adams:2001kk}. This was replaced by a pulse amplifier entirely powered by
the current flowing through the basevoltage divider, developed in 2002 at Jefferson Lab \cite{Popov:2003mj}.
The new electronics provided a factor of 10 PMT signal boost while preserving the fast PMT pulse shape. It also significantly improved signal amplitude and
signal to noise ratio: the high voltage needed to detect the single photo-electron signal (SPE) was reduced from in average by 360 V, see \F{pmtHVImprovement}.

\begin{figure}
	\centering
	\includegraphics[width=0.95\columnwidth,keepaspectratio]{img/pmtHVImprovement.png}
	\caption{Comparison of sector 5 gain matched PMT high voltages that provide a SPE peak at about ADC channel 200.
            The PMTs with the modified bases could produce the same response function of the original base but at about 360 HV less voltage (1666 V vs 1292 V)}
	\label{fig:pmtHVImprovement}
\end{figure}

The design was adapted to use the LTCC XP4500B base and a prototype, shown in \F{pmtWithDivider}, was built to provide a x10
amplification and a split signal directly from the PMT base.

\begin{figure}
	\centering
	\includegraphics[width=0.95\columnwidth,keepaspectratio]{img/pmtWithDivider.png}
	\caption{The prototype module installed in the XP4500B PMT base. The bottom of the base has been modified to contain the HV
				BNC input and two output signals BNC. }
	\label{fig:pmtWithDivider}
\end{figure}

In \F{dividerTests} (top) a comparison of the two signals shows the similarity between the two outputs.
During testing of the modified PMT bases, the output has been processed by a Flash ADC and acquired with a data
acquisition software using the PMT itself as trigger.
The corresponding SPE spectrum has been analyzed. The shape of the SPE signal is very similar to the original signal
coming from the external dedicated splitter and amplifier through the ADC electronics, see  \F{dividerTests} (bottom).

\begin{figure}
	\centering
	\includegraphics[width=0.87\columnwidth,keepaspectratio]{img/doubleSignal.png}
	\includegraphics[width=0.97\columnwidth,keepaspectratio]{img/fadcOutput.png}
	\caption{Top: qualitetively comparison of the two outputs from the base modifications as seen on an oscilloscope.
				The analysis of the adc spectra histograms confirmed quantitevily that the signals are identical. Bottom:
				The single photo-electron ADC spectrum of one of the output compared with the PMT output in the original
            configuration of a dedicated external splitter and amplifier.
    }
	\label{fig:dividerTests}
\end{figure}

180 bases were assembled at Jefferson Lab and installed on the PMT dividers. Both signals from all the modified bases were tested.
The response of the PMTs, amplified by a factor of 10, was verified to be identical to the original output.


