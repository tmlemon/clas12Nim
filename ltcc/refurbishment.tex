\section{LTCC Refurbishment}


The re-scoping of the LTCC to discriminate pions instead of electrons requires a dramatic increase in the
number of photo-electrons detected. Four areas have been considered to achieve this:

\begin{enumerate}
	\item Increase optic reflectivity
	\item Increase PMTs response to UV light
	\item Increase gas volume
	\item Address the gas leaks and other hardware issues.
\end{enumerate}


GEMC Simulations of pions quantify the Cherenkov light number of reflections on the cylindrical, elliptical,
parabolic and winston cones surfaces as:

\begin{itemize}
	\item 2 reflections only: $30\%$ of the times
	\item 3 reflections $40\%$ of the times
	\item 4 or more reflections $30\%$ of the times
\end{itemize}

The results have been confirmed with empirical observations during the mirrors aligmnent.

The original average measured reflectivity of the mirrors and winston cones
shows a degraded reflectivity compared to the values at installation, probably due to  17 years of aging, see Fig.~\ref{fig:reflectivityGain}.

A study of refurbishment of the mirrors and winston cones surfaces
involved re-coating samples of surfaces and mirrors using several companies.
All scenarios resulted in a dramatic increase of reflectivity. The original versus refurbished projected
reflectivity is shown in Fig.~\ref{fig:reflectivityGain}.

\begin{figure}[h]
	\centering
	\includegraphics[width=1.0\columnwidth,keepaspectratio]{img/reflectivityGain.png}
	\caption{A typical reflectivity measurement of one LTCC mirror (black diamond) shows a 65% reflectivity versus what can be achieved
		      by a new coating (red dots). }
	\label{fig:reflectivityGain}
\end{figure}

When the number of reflections are taken into account, the refurbishment shows
a factor of 2.4 gain in the visible and a factor of 3 in the UV:

\begin{itemize}
	\item Visible:
	\begin{itemize}
		\item original integrated reflectivity: $27.5\%$ (1 reflection $65\%$)
		\item refurbished integrated reflectivity: $68.1\%$  (1 reflection $88\%$ )
	\end{itemize}
	\item Ultra violet:
	\begin{itemize}
		\item original integrated reflectivity: $9.1\%$ (1 reflection $45\%$)
		\item refurbished integrated reflectivity: $34.3\%$  (1 reflection $75\%$ )
	\end{itemize}
\end{itemize}

The $C_4F_{10}$ reflectivity is shown in Fig.~\ref{fig:c4f10RefrIndex}. A detailed study of the detector response to electrons and pions used the Frank–Tamm to calculate
the Cherenkov radiation yield as a function of wavelengh and momenta, see for example Fig.~\ref{fig:photonYieldStudy}.

\begin{figure}[h]
	\centering
	\includegraphics[width=1.0\columnwidth,keepaspectratio]{img/c4f10RefrIndex.png}
	\caption{The gas refraction index as a function of wavelength.}
	\label{fig:c4f10RefrIndex}
\end{figure}

\begin{figure}[h]
	\centering
	\includegraphics[width=0.95\columnwidth,keepaspectratio]{img/photonYieldStudy.png}
\caption{Cherenkov light yield as a function of pion momentum and photon wavelength. The study took into account: the mirror reflectivity; the number of reflections on the mirror and Winston Cones;
the gas refraction index and the gas reflectivity as a function of wavelength; the PMT quantum efficiency; different improvements to optics and PMTs response increase. }
	\label{fig:photonYieldStudy}
\end{figure}

The results have been summarize in Fig.~\ref{fig:refurbishmentGains} for two main scenarios: mirrors re-coating and PMT's quantum efficiency improvements.
By performing both of these improvements the study shows that the LTCC response to pions will be the same as it was for electrons above 4 Gev/c.

\begin{figure}[h]
	\centering
	\includegraphics[width=0.95\columnwidth,keepaspectratio]{img/refurbishmentGains.png}
\caption{Ratio of LTCC response to pions to old response to electrons. Black diamonds: with no refurbishment, the goal of detecting pion cannot be reached.
          More than a factor of two can be gained by recoating the optics (blue squares). By also increasing the PMTs response to UV light (green stars), the
          LTCC can reach the old performance above pion momenta of 4 Gev / c.}
	\label{fig:refurbishmentGains}
\end{figure}


The design changes of the LTCC are summarized below:

\begin{enumerate}
\item Resize the box to fit in the reduced space between the Drift Chambers and the Time-Of-Flight:
	\begin{enumerate}
		\item Cut Sides
		\item Relocate segments 16-17-18
		\item Redesign, replace back-wall frame
	\end{enumerate}

	\item Decrease inefficient regions.
	\item Increase PMT gain, split output to provide both FADC and TDC signals.
	\item Minimize gas leaks
	\item Better mirrors support
	\item Increase light yield:
	\begin{enumerate}
		\item Re-coat all mirrors
		\item Re-coat WCs
		\item Wavelength shifting re-coating of the PMTs
		\item Add “nose” to increase gas volume
	\end{enumerate}
\end{enumerate}

\subsection{The Box}


- Box Cut, Nose, Mirrors re-location
- Back-wall, Connectors
- Mirror Support (Spine)




- Use Hermetical signals, HV connectors
- Seal the box from the inside
- Window: glue + sealant


\subsection{Mirrors Re-coating}

As seen in Fig.~\ref{fig:optics}, each segment is composed of four optical surfaces: one elliptical mirror,
one hyperbolic mirror, one cylindrical mirror and the Winston cone.

Several mirrors's reflectivity was measured and show significant degradation from the original
desired reflectivty of $90\%$ in the visible spectrum, see Fig.~\ref{fig:reflectivityBefore}.

A refurbishment of the mirrors was crucial to enhance the detector response to the pion emitted Cherenkov light.
Due to the material, assembly and dimension of the different type of mirrors, two different techniques were applied.

\begin{figure}[h]
\centering
	\includegraphics[width=1.0\columnwidth,keepaspectratio]{img/mirrorsReflectivityBefore.png}
	\caption{Average number of reflections calculated from simulations studies.}
	\label{fig:reflectivityBefore}
\end{figure}


\subsubsection{Re-coating of Cylindical mirrors}

The cylindrical mirrors range from 6 to 12 inches in length. Each mirror is made by a single piece of aluminum or plastic.
Due to the small size, it can fit in most vacuum chambers used to coat mirrors by evaporation of aluminum with magnesium fluoride
($AlMg_2$). After successful testing a re-coating of $AlMg_2$ onto the existing substrate, the work of recoating 216 cylindrical mirrors
was awared to ECI~\cite{ECI}. The re-coated mirrors shows a $\sim 90\%$ reflectivity in the visible spectrum and an exceptional $\sim 80\%$
reflectivity in the UV spectrum, see Fig.~\ref{fig:reflectivityAfter} for typical curves.


\subsubsection{Re-coating of Elliptical and Hyperbolic mirrors}

The elliptical and hyperbolic mirrors are composed by a kevlar support structure with a lexan substrate. The mounting hardware
material, that allowed for pitch, roll and yaw alignment of the mirrors, included wood and aluminum that was glued to the support structure.

Several companies attempted to re-coat these mirrors but failed due to the outgassing of the various material. Furthermore many of the mirrors
are $> 50$'' in length, longer than most vacuum chambers. In summary, the $AlMg_2$ could not be re-deposited directly on the mirrors.

A different approach consisted of coating thin (25 microns) lexan strips and glue the strips on the mirror's substrate. While promising, this
presented the challenge of protecting the coated lexan strip from shipping and handling and from the gluing procedure to the mirrors.

A working chain was setup to:

\begin{enumerate}
	\item coating the lexan strip
	\item protection with a film
	\item shipping
	\item gluing to mirrors
	\item testing reflectivity
\end{enumerate}

An example of unwrapping the film off the lexan strip is shown in Fig.~\ref{fig:filmOnStrip}. Several companies produced various test. In the end the
job was awarded to ECI~\cite{ECI}. The typical reflectivity of the refurbished mirrors is shown in Fig.~\ref{fig:reflectivityAfter}.

\begin{figure}[h]
\centering
	\includegraphics[width=0.98\columnwidth,keepaspectratio]{img/filmOnStrip.png}
	\caption{For the elliptical and hyperbolic mirrors lexan strips were coated with $AlMg_2$ then glued on top of the mirrors. The photo shows
            the process of removign the protecting film from one such strip.}
	\label{fig:filmOnStrip}
\end{figure}


\subsubsection{Elliptical Mirrors Gaps}

The segment with long elliptical mirrors presented several gaps between them, some a few cm long. To make sure that no light is lost in these gaps,
additional 120 micron thick lexan extension coated with $AlMg_2$ were glued to cover the gaps. The strips were manufactured by ECI~\cite{ECI}.


\subsubsection{Mirrors re-coating summary and results}

In Fig.~\ref{fig:reflectivityAfter} spectrum of 

\begin{figure}[h]
\centering
	\includegraphics[width=1.0\columnwidth,keepaspectratio]{img/mirrorsReflectivityAfter.png}
	\caption{Average number of reflections calculated from simulations studies.}
	\label{fig:reflectivityAfter}
\end{figure}




\subsection{Mirrors Alignment}



\begin{figure}[h]
\centering
	\includegraphics[width=1.0\columnwidth, height=0.5\textheight]{img/mirrorAlignmentSimulation.png}
	\caption{Average number of reflections calculated from simulations studies.}
	\label{fig:alignmentSimulation}
\end{figure}



Add mirror pieces to cover gaps


\subsection{Winston cones refurbishment}

Winston Cones (WC) are used to collect light onto the PMTs. In the LTCC there are three kind of WC:

\begin{enumerate}

\item Small
	\begin{enumerate}
		\item Height: 18cm
		\item Parallel Plate distance: 14cm
		\item Radius at the top: 20cm
		\item Radius at the bottom: 11cm
		\item Material: copper (electro-formed)
	\end{enumerate}

	\item Medium
	\begin{enumerate}
		\item Height: 22cm
		\item Parallel Plate distance: 15cm
		\item Radius at the top: 20cm
		\item Radius at the bottom: 11cm
		\item Material: 0.2” plastic (vacuum pressed)
	\end{enumerate}

	\item Large
	\begin{enumerate}
		\item Height: 30cm
		\item Parallel Plate distance: 18cm
		\item Radius at the top: 22cm
		\item Radius at the bottom: 11cm
		\item Material: copper (electro-formed)
	\end{enumerate}
\end{enumerate}

The reflectivity of the WC showed the same degradation as 
A setup on an optical bench allowed to measure the reflectivity for all the WCs at wavelengths between 200 and 400 nm was designed to accept incident light
shallow angles of 10-15 degrees (typical incident angle based on simulation studies), see \F{wcSetup}. A typical reflectivity of a poor WC is shown in \F{wcStatusBefore} (top).
All 216 WC were measured, and the results are shown in \F{wcStatusBefore} (bottom). This allowed to catalog the quality of the WC, as the

\begin{figure}
	\centering
	\includegraphics[width=0.95\columnwidth,keepaspectratio]{img/wcSetup.png}
	\caption{Setup to measure the WC reflectivity. The wavelength of light from a deuterium lamp was measured using a mono-chromator and splitted in two
            light beams, each with calibrated intensity. One of the light beam impinged on the WC at a typical angle of 12 degrees, while the other was directed at the reference PMT.
				}
	\label{fig:wcSetup}
\end{figure}


\begin{figure}
	\centering
	\includegraphics[width=1.0\columnwidth,keepaspectratio]{img/wcStatusBefore.png}
	\includegraphics[width=1.0\columnwidth,keepaspectratio]{img/winstoConeSample2Reflectivity.png}
\caption{Top: typical reflectivity of a poor WC.  }
	\label{fig:wcStatusBefore}
\end{figure}


\begin{figure}
	\centering
	\includegraphics[width=1.0\columnwidth,keepaspectratio]{img/wcStatusAfter.png}
	\caption{Top view of the back-wall of the LTCC. A stainless steel bar encapsulate a sandwich wall of aluminum and foam. On the left and right side
			of the frame a new patch panel allow for 3 hermetical connectors (1 HV, 2 signals) from each PTM. }
	\label{fig:wcStatusAfter}
\end{figure}

\begin{figure}
	\centering
	\includegraphics[width=1.0\columnwidth,keepaspectratio]{img/winstoConeSample1Reflectivity.png}
	\caption{Top view of the back-wall of the LTCC. A stainless steel bar encapsulate a sandwich wall of aluminum and foam. On the left and right side
			of the frame a new patch panel allow for 3 hermetical connectors (1 HV, 2 signals) from each PTM. }
	\label{fig:wcReflectivitySamples}
\end{figure}



\subsection{Photo-multipliers surface coating}

\paragraph{p-terphenyl recoating}
\paragraph{base modification: 2 x10 output}



\subsection{The LTCC Windows}

The LTCC windows that cover the upstream and downstream open frame of the box are a composite of
Tedlar/Mylar/Tedlar, see \F{windowDesign}. The Tedlar
material provides light tightness, while the Mylar adds the material strength necessary to withstand the gas pressure.

\begin{figure}
	\centering
	\includegraphics[width=1.0\columnwidth, keepaspectratio]{img/windowDesign.png}
	\includegraphics[width=1.0\columnwidth, keepaspectratio]{img/blank.png}
	\includegraphics[width=1.0\columnwidth, keepaspectratio]{img/windowSeaming.png}
	\caption{Top: the design of the LTCC Tedlar/Mylar/Tedlar window sandwich. The pyramid design allowed for the seaming shown at the bottom.
			 Bottom: the seaming design involves gluing Mylar to Mylar to ensure that the window stress is transmitted entirely to the Mylar. }
	\label{fig:windowDesign}
\end{figure}

The window was fabricated in two steps:

\begin{enumerate}
	\item lamination of Tedlar/Mylar/Tedlar rolls 1.6~m  wide
	\item seaming of the laminated strips into a square 4.8~m $\times$ 4.8~m window
\end{enumerate}

The lamination of the composite material, with dimensions outlined in \F{windowDesign} (top) was performed
at Madico \cite{madico}, where a sheet 400~m long was produced.

At Jefferson Lab rectangles were cut out of the laminated sheet, each 1.6~m wide and 4.8~m long.
To form a final 4.8~m $\times$ 4.8~m single LTCC window, three of the rectangles were seamed together
using G/Flex 655. The seam was load tested to withstand a pressure 10 times higher than that expected from
the gas flow and gas weight, see \F{windowTest}.

\begin{figure}
	\centering
	\includegraphics[width=1.0\columnwidth, height=1.0\columnwidth]{img/windowTest.png}
	\caption{A window sample (white piece in the photograph) was tested with up to 388~lbs of load.
          No significant damage was observed until rupture, which occurred at a load of 388~lbs, corresponding
          to about 15,000~psi of stress on the window, about a factor of 10 higher than the stress during normal
          operations of the detector.}
	\label{fig:windowTest}
\end{figure}

\subsubsection{Window Installation and Gas Leak Tests}

The installation of the window onto the box was achieved through gluing the window on the box sides using
G/Flex 655. The width of the window
attached with glue was 12~cm, to provide sufficient gluing area.
A photograph of the downstream window after installation is shown in \F{downstreamWindow}.

\begin{figure}
	\centering
	\includegraphics[width=1.0\columnwidth,keepaspectratio]{img/downstreamWindow.png}
	\caption{The downstream window of one LTCC sector during curing of the glue. The yellow strips protect the
          window seaming.}
	\label{fig:downstreamWindow}
\end{figure}

After curing of both the upstream and downstream windows, the LTCC box was filled with nitrogen gas to a pressure of
2~in of water.
Freon gas was pumped into the box and leaks were detected using a refrigerant leak detector. After the leaks were
sealed, the box was pressurized
for a 48 hour period to test the overall box gas tightness. This procedure was repeated after every movement of the LTCC boxes, as small
shifts of the frame walls had the potential to introduce additional leaks.



