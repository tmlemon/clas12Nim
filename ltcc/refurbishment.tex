\section{LTCC Refurbishment}


The re-scoping of the LTCC to discriminate pions instead of electrons required a dramatic increase in the
number of photo-electrons detected. Four areas were considered to achieve this:

\begin{enumerate}
	\item Increase optical reflectivity
	\item Increase PMT response to UV light
	\item Increase gas volume
	\item Address the gas leaks and other hardware issues.
\end{enumerate}


Monte-Carlo simulations of pions in the LTCC quantified the number of Cherenkov light  reflections on the cylindrical, elliptical,
parabolic, and Winston cones surfaces as:

\begin{itemize}
	\item 2 reflections only: 30$\%$ of the time
	\item 3 reflections: 40$\%$ of the time
	\item 4 or more reflections: 30$\%$ of the time
\end{itemize}

\noindent
These results were confirmed with empirical observations during the mirror alignment.

A study of the refurbishment of the mirror and Winston cone surfaces
involved re-coating samples of the surfaces and mirrors using several companies.
The original average measured reflectivity of the mirrors and Winston cones
showed a degraded reflectivity compared to the values at installation and to
the new samples, likely due to  17 years of aging, see \F{reflectivityGain}.


\begin{figure}
	\centering
	\includegraphics[width=0.98\columnwidth, height=0.7\columnwidth]{img/reflectivityGain.png}
	\caption{A typical reflectivity measurement vs wavelength of one original LTCC mirror (diamonds)
			shows a 65\% reflectivity versus what can be achieved
			by a new coating (open circles), about 90\% reflectivity. The increase is even more significant at smaller wavelengths }
	\label{fig:reflectivityGain}
\end{figure}

In the study a few scenarios using the $C_4F_{10}$ index of refraction (shown in \F{c4f10RefrIndex})
and considering the various reflection probabilities outlined above  were considered:

\begin{figure}
	\centering
	\includegraphics[width=0.98\columnwidth, height=0.65\columnwidth]{img/c4f10RefrIndex.png}
	\caption{The $C_4F_{10}$ gas index of refraction as a function of wavelength.}
	\label{fig:c4f10RefrIndex}
\end{figure}

\begin{itemize}
	\item using old or new mirror reflectivity (see \F{reflectivityGain});
	\item using a completely transparent gas or the $C_4F_{10}$ with its measured transparency;
	\item using the actual or the enhanced PMT quantum efficiency function;
	\item using or not the additional $C_4F_{10}$ gas volume;
	\item comparing the pion response to the original CLAS 6 GeV era electron response.
\end{itemize}

The results of these studies are discussed below.

The reflection probability vs wavelength distributions for each scenario were input into
the Frank\textendash Tamm formula \cite{Frank:1937fk} below to calculate
the Cherenkov radiation yield as a function of wavelength and momenta:

%~\ref{eq:cerenkov}.

\begin{equation} \label{eq:cerenkov}
	\frac{d^2n}{dxd\lambda} = \frac{2\pi z^2\alpha}{\lambda^2}\sin^2{\theta_C(v)},
\end{equation}

\noindent
where $\alpha$ is the fine structure constant, $\theta_C$ is the Cherenkov cone half-angle, $z$ is the particle charge and $v$ is its speed.

One particular combination among the described scenarios is given as an example \F{photonYieldStudy} for pions,
the measured non-refurbished mirror reflectivity, a 100$\%$ transparent gas, and a PMT with ideal refurbished quantum efficiency.

\begin{figure}
	\centering
	\includegraphics[width=0.98\columnwidth, height=0.75\columnwidth]{img/photonYieldStudy.png}
	\caption{Cherenkov light yield as a function of pion momentum and photon wavelength using the Frank\textendash Tamm formula
             for one of the scenarios considered: mirrors and Winston cones with low reflectivity (LTCC original mirror, not refurbished),
             a perfectly transparent gas and an ideally refurbished PMT.}
	\label{fig:photonYieldStudy}
\end{figure}


When the number of reflections is taken into account, the refurbishment shows
a factor of 2.5 gain in the visible light wavelengths and a factor of 3.8 in the UV:

\begin{itemize}
	\item Visible:
	\begin{itemize}
		\item original integrated reflectivity: 27.5$\%$
		\item refurbished integrated reflectivity: 68.1$\%$
	\end{itemize}
	\item Ultra-violet:
	\begin{itemize}
		\item original integrated reflectivity: 9.1$\%$
		\item refurbished integrated reflectivity: 34.3$\%$
	\end{itemize}
\end{itemize}


The results have been summarized in \F{refurbishmentGains} for the two main scenarios: mirror re-coating and PMT quantum efficiency improvements.
By performing both of these improvements, the study shows that the LTCC response to pions will be the same as it was for electrons above 4 GeV
in the original LTCC for CLAS.

\begin{figure}
	\centering
	\includegraphics[width=0.99\columnwidth, height=0.7\columnwidth]{img/refurbishmentGains.png}
	\caption{Ratio of LTCC response to pions to the old response to electrons. Diamonds: with no refurbishment, the goal of
		     detecting pions cannot be reached. More than a factor of two can be gained by re-coating the mirrors and Winston cones
             (squares). In addition, by increasing the PMT response to UV light (stars), the LTCC can reach the old
			 performance above pion momenta of 4 GeV.}
	\label{fig:refurbishmentGains}
\end{figure}

The design changes of the LTCC are summarized below:

\begin{enumerate}
\item Resize the box to fit in the reduced space between the Drift Chambers and the Forward Time-Of-Flight:
	\begin{enumerate}
		\item Cut Sides
		\item Relocate the three mirror sets closest to the back-wall
		\item Redesign, replace back-wall frame
	\end{enumerate}

	\item Decrease inefficient regions
	\item Increase PMT gain, split output to provide both FADC and TDC signals
	\item Minimize gas leaks
	\item Better mirror supports
	\item Increase light yield:
	\begin{enumerate}
		\item Re-coat all mirrors
		\item Re-coat Winston cones
		\item Wavelength shifting re-coating of the PMTs
		\item Modify small angle nose to increase gas volume
	\end{enumerate}
\end{enumerate}

\subsection{The Box}


- Box Cut, Nose, Mirrors re-location
- Back-wall, Connectors
- Mirror Support (Spine)




- Use Hermetical signals, HV connectors
- Seal the box from the inside
- Window: glue + sealant

\subsection{Mirrors Re-coating}

As seen in Fig.~\ref{fig:optics}, each segment is composed of four optical surfaces: one elliptical mirror,
one hyperbolic mirror, one cylindrical mirror and the Winston cone.

Several mirrors's reflectivity was measured and show significant degradation from the original
desired reflectivty of $90\%$ in the visible spectrum, see Fig.~\ref{fig:reflectivityBefore}.

A refurbishment of the mirrors was crucial to enhance the detector response to the pion emitted Cherenkov light.
Due to the material, assembly and dimension of the different type of mirrors, two different techniques were applied.

\begin{figure}[h]
\centering
	\includegraphics[width=1.0\columnwidth,keepaspectratio]{img/mirrorsReflectivityBefore.png}
	\caption{Average number of reflections calculated from simulations studies.}
	\label{fig:reflectivityBefore}
\end{figure}


\subsubsection{Re-coating of Cylindical mirrors}

The cylindrical mirrors range from 6 to 12 inches in length. Each mirror is made by a single piece of aluminum or plastic.
Due to the small size, it can fit in most vacuum chambers used to coat mirrors by evaporation of aluminum with magnesium fluoride
($AlMg_2$). After successful testing a re-coating of $AlMg_2$ onto the existing substrate, the work of recoating 216 cylindrical mirrors
was awared to ECI~\cite{ECI}. The re-coated mirrors shows a $\sim 90\%$ reflectivity in the visible spectrum and an exceptional $\sim 80\%$
reflectivity in the UV spectrum, see Fig.~\ref{fig:reflectivityAfter} for typical curves.


\subsubsection{Re-coating of Elliptical and Hyperbolic mirrors}

The elliptical and hyperbolic mirrors are composed by a kevlar support structure with a lexan substrate. The mounting hardware
material, that allowed for pitch, roll and yaw alignment of the mirrors, included wood and aluminum that was glued to the support structure.

Several companies attempted to re-coat these mirrors but failed due to the outgassing of the various material. Furthermore many of the mirrors
are $> 50$'' in length, longer than most vacuum chambers. In summary, the $AlMg_2$ could not be re-deposited directly on the mirrors.

A different approach consisted of coating thin (25 microns) lexan strips and glue the strips on the mirror's substrate. While promising, this
presented the challenge of protecting the coated lexan strip from shipping and handling and from the gluing procedure to the mirrors.

A working chain was setup to:

\begin{enumerate}
	\item coating the lexan strip
	\item protection with a film
	\item shipping
	\item gluing to mirrors
	\item testing reflectivity
\end{enumerate}

An example of unwrapping the film off the lexan strip is shown in Fig.~\ref{fig:filmOnStrip}. Several companies produced various test. In the end the
job was awarded to ECI~\cite{ECI}. The typical reflectivity of the refurbished mirrors is shown in Fig.~\ref{fig:reflectivityAfter}.

\begin{figure}[h]
\centering
	\includegraphics[width=0.98\columnwidth,keepaspectratio]{img/filmOnStrip.png}
	\caption{For the elliptical and hyperbolic mirrors lexan strips were coated with $AlMg_2$ then glued on top of the mirrors. The photo shows
            the process of removign the protecting film from one such strip.}
	\label{fig:filmOnStrip}
\end{figure}


\subsubsection{Elliptical Mirrors Gaps}

The segment with long elliptical mirrors presented several gaps between them, some a few cm long. To make sure that no light is lost in these gaps,
additional 120 micron thick lexan extension coated with $AlMg_2$ were glued to cover the gaps. The strips were manufactured by ECI~\cite{ECI}.


\subsubsection{Mirrors re-coating summary and results}

In Fig.~\ref{fig:reflectivityAfter} spectrum of 

\begin{figure}[h]
\centering
	\includegraphics[width=1.0\columnwidth,keepaspectratio]{img/mirrorsReflectivityAfter.png}
	\caption{Average number of reflections calculated from simulations studies.}
	\label{fig:reflectivityAfter}
\end{figure}




\subsection{Mirrors Alignment}



\begin{figure}[h]
\centering
	\includegraphics[width=1.0\columnwidth, height=0.5\textheight]{img/mirrorAlignmentSimulation.png}
	\caption{Average number of reflections calculated from simulations studies.}
	\label{fig:alignmentSimulation}
\end{figure}



Add mirror pieces to cover gaps

\subsection{Winston cones refurbishment}

Winston Cones (WC) are used to collect light onto the PMTs. In the LTCC there are three kind of WC:

\begin{enumerate}

\item Small
	\begin{enumerate}
		\item Height: 18cm
		\item Parallel Plate distance: 14cm
		\item Radius at the top: 20cm
		\item Radius at the bottom: 11cm
		\item Material: copper (electro-formed)
	\end{enumerate}

	\item Medium
	\begin{enumerate}
		\item Height: 22cm
		\item Parallel Plate distance: 15cm
		\item Radius at the top: 20cm
		\item Radius at the bottom: 11cm
		\item Material: 0.2” plastic (vacuum pressed)
	\end{enumerate}

	\item Large
	\begin{enumerate}
		\item Height: 30cm
		\item Parallel Plate distance: 18cm
		\item Radius at the top: 22cm
		\item Radius at the bottom: 11cm
		\item Material: copper (electro-formed)
	\end{enumerate}
\end{enumerate}

The reflectivity of the WC showed the same degradation as 
A setup on an optical bench allowed to measure the reflectivity for all the WCs at wavelengths between 200 and 400 nm was designed to accept incident light
shallow angles of 10-15 degrees (typical incident angle based on simulation studies), see \F{wcSetup}. A typical reflectivity of a poor WC is shown in \F{wcStatusBefore} (top).
All 216 WC were measured, and the results are shown in \F{wcStatusBefore} (bottom). This allowed to catalog the quality of the WC, as the

\begin{figure}
	\centering
	\includegraphics[width=0.95\columnwidth,keepaspectratio]{img/wcSetup.png}
	\caption{Setup to measure the WC reflectivity. The wavelength of light from a deuterium lamp was measured using a mono-chromator and splitted in two
            light beams, each with calibrated intensity. One of the light beam impinged on the WC at a typical angle of 12 degrees, while the other was directed at the reference PMT.
				}
	\label{fig:wcSetup}
\end{figure}


\begin{figure}
	\centering
	\includegraphics[width=1.0\columnwidth,keepaspectratio]{img/wcStatusBefore.png}
	\includegraphics[width=1.0\columnwidth,keepaspectratio]{img/winstoConeSample2Reflectivity.png}
\caption{Top: typical reflectivity of a poor WC.  }
	\label{fig:wcStatusBefore}
\end{figure}


\begin{figure}
	\centering
	\includegraphics[width=1.0\columnwidth,keepaspectratio]{img/wcStatusAfter.png}
	\caption{Top view of the back-wall of the LTCC. A stainless steel bar encapsulate a sandwich wall of aluminum and foam. On the left and right side
			of the frame a new patch panel allow for 3 hermetical connectors (1 HV, 2 signals) from each PTM. }
	\label{fig:wcStatusAfter}
\end{figure}

\begin{figure}
	\centering
	\includegraphics[width=1.0\columnwidth,keepaspectratio]{img/winstoConeSample1Reflectivity.png}
	\caption{Top view of the back-wall of the LTCC. A stainless steel bar encapsulate a sandwich wall of aluminum and foam. On the left and right side
			of the frame a new patch panel allow for 3 hermetical connectors (1 HV, 2 signals) from each PTM. }
	\label{fig:wcReflectivitySamples}
\end{figure}


\subsection{Photo-multipliers surface coating}

\paragraph{p-terphenyl recoating}
\paragraph{base modification: 2 x10 output}


\subsection{Photo-multipliers dividers}

In the original readout electronics the LTCC single output from each PMT was amplified by a factor of 10
and then split in two to feed the ADC and TDC boards. This amplification and splitting was performed
by a dedicated electronic module (UVA 132) \cite{Adams:2001kk}. This was replaced by a pulse amplifier entirely powered by
the current flowing through the basevoltage divider, developed in 2002 at Jefferson Lab \cite{Popov:2003mj}.
The new electronics provided a factor of 10 PMT signal boost while preserving the fast PMT pulse shape. It also significantly improved signal amplitude and
signal to noise ratio: the high voltage needed to detect the single photo-electron signal (SPE) was reduced from in average by 360 V, see \F{pmtHVImprovement}.

\begin{figure}
	\centering
	\includegraphics[width=0.95\columnwidth,keepaspectratio]{img/pmtHVImprovement.png}
	\caption{Comparison of sector 5 gain matched PMT high voltages that provide a SPE peak at about ADC channel 200.
            The PMTs with the modified bases could produce the same response function of the original base but at about 360 HV less voltage (1666 V vs 1292 V)}
	\label{fig:pmtHVImprovement}
\end{figure}

The design was adapted to use the LTCC XP4500B base and a prototype, shown in \F{pmtWithDivider}, was built to provide a x10
amplification and a split signal directly from the PMT base.

\begin{figure}
	\centering
	\includegraphics[width=0.95\columnwidth,keepaspectratio]{img/pmtWithDivider.png}
	\caption{The prototype module installed in the XP4500B PMT base. The bottom of the base has been modified to contain the HV
				BNC input and two output signals BNC. }
	\label{fig:pmtWithDivider}
\end{figure}

In \F{dividerTests} (top) a comparison of the two signals shows the similarity between the two outputs.
During testing of the modified PMT bases, the output has been processed by a Flash ADC and acquired with a data
acquisition software using the PMT itself as trigger.
The corresponding SPE spectrum has been analyzed. The shape of the SPE signal is very similar to the original signal
coming from the external dedicated splitter and amplifier through the ADC electronics, see  \F{dividerTests} (bottom).

\begin{figure}
	\centering
	\includegraphics[width=0.87\columnwidth,keepaspectratio]{img/doubleSignal.png}
	\includegraphics[width=0.97\columnwidth,keepaspectratio]{img/fadcOutput.png}
	\caption{Top: qualitetively comparison of the two outputs from the base modifications as seen on an oscilloscope.
				The analysis of the adc spectra histograms confirmed quantitevily that the signals are identical. Bottom:
				The single photo-electron ADC spectrum of one of the output compared with the PMT output in the original
            configuration of a dedicated external splitter and amplifier.
    }
	\label{fig:dividerTests}
\end{figure}

180 bases were assembled at Jefferson Lab and installed on the PMT dividers. Both signals from all the modified bases were tested.
The response of the PMTs, amplified by a factor of 10, was verified to be identical to the original output.



\subsection{The LTCC Windows}

The LTCC windows that cover the upstream and downstream open frame of the box are a composite of
Tedlar/Mylar/Tedlar, see \F{windowDesign}. The Tedlar
material provides light tightness, while the Mylar adds the material strength necessary to withstand the gas pressure.

\begin{figure}
	\centering
	\includegraphics[width=1.0\columnwidth, keepaspectratio]{img/windowDesign.png}
	\includegraphics[width=1.0\columnwidth, keepaspectratio]{img/blank.png}
	\includegraphics[width=1.0\columnwidth, keepaspectratio]{img/windowSeaming.png}
	\caption{Top: the design of the LTCC Tedlar/Mylar/Tedlar window sandwich. The pyramid design allowed for the seaming shown at the bottom.
			 Bottom: the seaming design involves gluing Mylar to Mylar to ensure that the window stress is transmitted entirely to the Mylar. }
	\label{fig:windowDesign}
\end{figure}

The window was fabricated in two steps:

\begin{enumerate}
	\item lamination of Tedlar/Mylar/Tedlar rolls 1.6~m  wide
	\item seaming of the laminated strips into a square 4.8~m $\times$ 4.8~m window
\end{enumerate}

The lamination of the composite material, with dimensions outlined in \F{windowDesign} (top) was performed
at Madico \cite{madico}, where a sheet 400~m long was produced.

At Jefferson Lab rectangles were cut out of the laminated sheet, each 1.6~m wide and 4.8~m long.
To form a final 4.8~m $\times$ 4.8~m single LTCC window, three of the rectangles were seamed together
using G/Flex 655. The seam was load tested to withstand a pressure 10 times higher than that expected from
the gas flow and gas weight, see \F{windowTest}.

\begin{figure}
	\centering
	\includegraphics[width=1.0\columnwidth, height=1.0\columnwidth]{img/windowTest.png}
	\caption{A window sample (white piece in the photograph) was tested with up to 388~lbs of load.
          No significant damage was observed until rupture, which occurred at a load of 388~lbs, corresponding
          to about 15,000~psi of stress on the window, about a factor of 10 higher than the stress during normal
          operations of the detector.}
	\label{fig:windowTest}
\end{figure}

\subsubsection{Window Installation and Gas Leak Tests}

The installation of the window onto the box was achieved through gluing the window on the box sides using
G/Flex 655. The width of the window
attached with glue was 12~cm, to provide sufficient gluing area.
A photograph of the downstream window after installation is shown in \F{downstreamWindow}.

\begin{figure}
	\centering
	\includegraphics[width=1.0\columnwidth,keepaspectratio]{img/downstreamWindow.png}
	\caption{The downstream window of one LTCC sector during curing of the glue. The yellow strips protect the
          window seaming.}
	\label{fig:downstreamWindow}
\end{figure}

After curing of both the upstream and downstream windows, the LTCC box was filled with nitrogen gas to a pressure of
2~in of water.
Freon gas was pumped into the box and leaks were detected using a refrigerant leak detector. After the leaks were
sealed, the box was pressurized
for a 48 hour period to test the overall box gas tightness. This procedure was repeated after every movement of the LTCC boxes, as small
shifts of the frame walls had the potential to introduce additional leaks.



