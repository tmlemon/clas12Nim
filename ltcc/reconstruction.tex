\section{Reconstruction}

The aim of the LTCC is to differentiate between pions and kaons. The lighter pions
leave a signal in the detector, while the heavier kaons pass through the detector without
leaving a signal. To accomplish this, it is necessary to associate the hits in the detector with their
corresponding tracks.
On average, the Cherenkov light from a charged particle will hit between 1 and 3
adjacent PMTs.
The task of the reconstruction program is therefore to (a) cluster together the hits that
belong to a single track and (b) provide the positional information needed to match the
cluster with the correct reconstructed track. The reconstruction program is implemented as an engine in the
CLAS12 event reconstruction framework~\cite{recon-nim}.

\subsection{Clustering Algorithm}

The Cherenkov cone associated with a charged particle is in all cases contained within a
single sector of the LTCC. This allows for a relatively straightforward clustering
algorithm:
\begin{enumerate}
	\item Scan for the highest multiplicity hits, identified as the cluster center;
    \item Grow the cluster by adding all hits adjacent to the cluster center within this
      sector;
    \item Repeat the procedure until all hits have been assigned to a cluster.
\end{enumerate}

\subsection{Track Matching}

The \textit{true} cluster center can be defined as the position where the charged particle
(and its Cherenkov cone) crossed the elliptical mirror of the LTCC.
Due to the geometry of the LTCC, this position does not uniquely correspond to a single
PMT, as the angle with which a particle crosses the elliptical mirrors depends on the
particle momentum, position, charge, and the current in the torus.
The implication of this is that, based solely on the LTCC hits, the \textit{true} cluster
position cannot be uniquely constrained.
This is illustrated in \F{trackmatching}.

\begin{figure}
  \centering
  \includegraphics[width=0.99\columnwidth,keepaspectratio]{img/LTCC-event-1.png}
  \includegraphics[width=0.99\columnwidth,keepaspectratio]{img/LTCC-event-2.png}
  \caption{Two different simulated LTCC hits for particles passing through the same
  elliptical mirror.
  Based on particle kinematics, either the PMT in the same segment (top) or a neighboring
  segment (bottom) is hit.}
  \label{fig:trackmatching}
\end{figure}

The track matching is performed in a stage of the reconstruction where the tracking
information is available. In order to perform the track matching, the estimated
\textit{true} cluster position is recalculated for each track, leveraging the Monte-Carlo
simulation of the LTCC to correctly associate a tentative \textit{true} cluster position
with the measured hits.
The track that passes the closest to the tentative \textit{true} cluster position is
then chosen as the true match for this cluster.
