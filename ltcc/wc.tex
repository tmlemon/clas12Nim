\subsection{Winston Cone Refurbishment}

Winston cones (WCs) are used to collect light onto the PMTs. In the LTCC there are three kind of WCs:

\begin{enumerate}

\item Small
	\begin{enumerate}
		\item Height: 18 cm
		\item Radius at the top: 20 cm
		\item Radius at the bottom: 11 cm
		\item Material: 0.25 cm thick copper (electro-formed)
	\end{enumerate}

	\item Medium
	\begin{enumerate}
		\item Height: 22 cm
		\item Radius at the top: 20 cm
		\item Radius at the bottom: 11 cm
		\item Material: 0.5 cm thick plastic (vacuum pressed)
	\end{enumerate}

	\item Large
	\begin{enumerate}
		\item Height: 30 cm
		\item Radius at the top: 22 cm
		\item Radius at the bottom: 11 cm
		\item Material: 0.25 cm thick copper (electro-formed)
	\end{enumerate}
\end{enumerate}

\begin{figure}
	\centering
	\includegraphics[width=0.98\columnwidth,keepaspectratio]{img/wcSetup.png}
	\caption{Setup to measure the WC reflectivity. The wavelength of light from a deuterium lamp was measured
          using a monochromator and split in two light beams, each with calibrated intensity. One of the light beams
          impinged on the WC at a typical angle of 12\mdeg, while the other was directed at the reference PMT. The
          reflectivity was measured in different spots for a sample of WCs.  The results proved to be independent of
          the spot location.}
	\label{fig:wcSetup}
\end{figure}

The reflectivity of the WCs showed quantitatively the same degradation as the mirrors. However, due to their shape,
re-coating of the WC is more costly than the mirrors and the LTCC budget allowed refurbishing only 160 out of the
216 total WCs.
A setup on an optical bench to measure the reflectivity for all of the WCs at wavelengths between 200 and 400~nm
was designed to
accept incident shallow angles of 10\mdeg-15\mdeg (typical incidence angle based on simulation studies), see \F{wcSetup}.
The typical reflectivity of a poor WC vs. wavelength is shown in \F{wcStatusBefore} (top).
All 216 WCs were studied, and the average reflectivity results are shown in \F{wcStatusBefore} (bottom).
These data allowed cataloging of the quality of the WCs to select the worst ones to refurbish.
The cones were put in a vacuum chamber and AlMgF$_2$ was deposited on top of the existing coating.
The typical reflectivity of a representative WC after re-coating is shown in \F{wcStatusAfter} (top).
About 30 cones needed the additional treatment of removing the existing aluminum coating to improve the new AlMgF$_2$ deposition.
Even then, about half of these cones did not show improvement.
The results of the WC refurbishment are summarized in \F{wcStatusAfter} (bottom).

\begin{figure}
	\centering
	\includegraphics[width=0.98\columnwidth,keepaspectratio]{img/winstoConeSample2Reflectivity.png}
	\includegraphics[width=0.98\columnwidth,keepaspectratio]{img/wcStatusBefore.png}
	\caption{Top: typical reflectivity vs. wavelength (nm) of a ``very poor'' WC. The reflectivity is below 30\% for most
          wavelengths between 200 and 400~nm. The reflectivity proved to be independent of the particular spot on the
          WC surface. Bottom: the average reflectivity $r$ between 200 and 400~nm for all WCs. The shaded gray
          boxes represents WCs with poor reflectivity ($r < 50\%$).}
	\label{fig:wcStatusBefore}
\end{figure}

\begin{figure}
	\centering
	\includegraphics[width=0.98\columnwidth,keepaspectratio]{img/winstoConeSample1Reflectivity.png}
	\includegraphics[width=0.98\columnwidth,keepaspectratio]{img/wcStatusAfter.png}
	\caption{Top: typical reflectivity vs. wavelength (nm) of a ``very poor'' WC after refurbishment. The
          reflectivity quickly rises to $65\%$ at a wavelength of about 340~nm. Bottom: average WC reflectivity $r$
          between 200 and 400~nm for all WCs. The shaded gray boxes represents WCs with poor reflectivity
          ($r < 50\%$). This picture should be compared to \F{wcStatusBefore}. Most re-coated WCs show improved
          reflectivity.}
	\label{fig:wcStatusAfter}
\end{figure}


