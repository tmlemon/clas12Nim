\subsection{Box Modifications}



\paragraph{Box Cut and support relocation }

In the CLAS12 design upgrade the space between the Drift Chambers and the Forward Time-Of-Flight was reduced.
In order to accommodate the LTCC in the new space the original aluminum frame has been modified with a cut, see Fig.~\ref{fig:boxCut}.

Four mirror set, segment 15, 16, 17 and 18 had the mounting structure on the part that was removed.
To re-position these segments new threaded holes has been drilled in the frame, and the old holes have been plugged.

\begin{figure}[hbt]
	\centering
	\includegraphics[width=1.0\columnwidth,keepaspectratio]{img/boxCut.png}
	\caption{The top part of the LTCC frame had to be removed. Left: design before removal. Right: after removal}
	\label{fig:boxCut}
\end{figure}


\paragraph{Nose Addition}

In the original design the upstream window followed the spherical curvature of the frame sidewalls. To help the goal of increasing
the pion number of Cherenkov photons, a nose addition (see Fig.~\ref{fig:boxCut}) has been designed and built to increase the gas volume.
The nose dimensions have been optimized to provide the best support while at the same time maximize the gas volume increase. The gas increase
of the final configuration is shown in Fig.~\ref{fig:noseVolume}.

\begin{figure}[hbt]
	\centering
	\includegraphics[width=1.0\columnwidth,keepaspectratio]{img/noseVolume.png}
	\caption{The top part of the LTCC frame had to be removed. Left: design before removal. Right: after removal}
	\label{fig:noseVolume}
\end{figure}


\paragraph{Back-wall and Connectors}
In the original implementation both the high voltage and signals connectors that link the PMTs inside the hermetical box and the electronics were not
hermetical. Epoxy has been used to minimize the leaks from these connectors.
During the refurbishment the patch panel had been re-design to accommodate hermetical connectors. This involved rebuilding the back wall entirely.

The new back wall design is shown in Fig.~\ref{fig:backWall}. The wall is supported by stainless steel bars that enclose a panel made of foam enclosed by
thin aluminum sheets to minimize the radiation length.
The new patch panels provides 3 connectors for each PMTs: one for High Voltage and two for an identical signal coming from the modified PMT base.
The new connectors are hermetical to eliminate gas leaks in the patch panel.

\begin{figure}[hbt]
	\centering
	\includegraphics[width=1.0\columnwidth,keepaspectratio]{img/backWall.png}
	\caption{The top part of the LTCC frame had to be removed. Left: design before removal. Right: after removal}
	\label{fig:backWall}
\end{figure}


\paragraph{Mirror Support Spine}



- Seal the box from the inside
- Window: glue + sealant




