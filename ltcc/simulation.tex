\section{Simulation}

A realistic model of the LTCC has been developed, describing the location and material composition
of the support box, the mirrors, PMTs, Winston cones, magnetic shields, the C$_4$F$_{10}$ gas, and the
optical properties of the gas and mirrors and WCs as a function of wavelength, see Ref.~\cite{sim-nim}.
Here we describe additional details of the mirror geometry implementation relevant to the technique
used for the alignment.

The optics of the mirrors were tuned to compromise the two different topology of angles of incidence
(positive and negative particles) by using photons originating from the target (placed at the center of the CLAS12 coordinate system),
see \ref{sec:mirrorAlignment}.
This is reflected in the mirrors mathematical outline implemented in the simulation and the choice of the
focal points of the elliptical and hyperbolic mirrors \F{mirrorMath}.

\begin{figure}
	\centering
	\includegraphics[width=0.98\columnwidth,keepaspectratio]{img/mirrorMath1.png}
	\includegraphics[width=0.98\columnwidth,keepaspectratio]{img/mirrorMath2.png}
	\caption{The geometry of the LTCC mirrors is defined by functions describing an ellipse and a hyperbola.
             Top: the ellipsoidal curve (solid line), with its center (square) and its focal points
             (circle at the origin, representing the target, and the bottom star).
             The two hyperbola curves, defined by the two focal points (stars, one coinciding with the ellipse
			 focal point and the other at the desired location of the PMT) are the dashed lines. The
			 hyperbola at the bottom is the one that defines the hyperbolic mirror.
             Bottom: zoomed in details of the setup above. The squares represent the left and right mirror edges.
             These parameters are stored in databases and loaded in the software modeling the LTCC mirrors
             in the GEMC simulation \cite{sim-nim}.}
	\label{fig:mirrorMath}
\end{figure}

The elliptical mirrors are built by a sequence boolean operation of ellipsoidal
and cubic shaped volumes and the hyperbolic mirrors are made by Geant4 ``polycones'' (volumes with cylindrical symmetry
with varying radius along one axis) that approximate the shape of the mirrors using a number of facets
varying from 10 (smallest mirror) to 30 (longest mirrors). 


\subsection{Run Period Variations}

At the start of CLAS12 beam operations, there was not sufficient C$_4$F$_{10}$ gas to fill all sectors, so some
LTCC sectors were removed from the Forward Carriage. As they were installed or removed, any gas leaks were
found and fixed.
These CLAS12 configuration changes are imported in GEMC as database variations of the simulation setup.
The default simulations only include sector 2 (S2), S3, S5, and S6 as the RICH detector replaces the LTCC
boxes in S1 and S4. The variations are listed in Table~\ref{tab:simVariations}.

\begin{table}
	\begin{center}
		\begin{tabular}{| l | c |}
			\hline \hline
			Run Period       & Sectors Installed and Gas \\
			\hline
			Default          & S2, S3, S5, S6, all C$_4$F$_{10}$    \\
			RGA Spring 2018  & S2, S3, S6 (N$_2$), S5 (C$_4$F$_{10}$)  \\
			RGA Fall 2018    & S3 (C$_4$F$_{10}$), S5 (N$_2$)          \\
			RGB Spring 2018  & S3 (C$_4$F$_{10}$), S5 (C$_4$F$_{10}$) \\
			\hline \hline
		\end{tabular}
	\end{center}
	\caption{LTCC simulation variations for different CLAS12 run periods. Shown are which sectors are present
          and the gas in each sector}
	\label{tab:simVariations}
\end{table}
