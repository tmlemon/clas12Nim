\subsection{The LTCC Windows}


\begin{figure}[!ht]
	\centering
	\includegraphics[width=0.98\columnwidth, keepaspectratio]{img/windowDesign.png}
	\includegraphics[width=0.98\columnwidth, keepaspectratio]{img/blank.png}
	\includegraphics[width=0.98\columnwidth, keepaspectratio]{img/windowSeaming.png}
	\caption{Top: the design of the LTCC Tedlar/Mylar/Tedlar window sandwich. The pyramid design allowed for the
          seaming shown at the bottom. Bottom: the seaming design involves gluing Mylar to Mylar to ensure that the window
          stress is transmitted entirely to the Mylar.}
	\label{fig:windowDesign}
\end{figure}

The LTCC windows that cover the upstream and downstream open frame of the box are a composite of
Tedlar/Mylar/Tedlar, see \F{windowDesign}. The Tedlar material provides light tightness, while the Mylar adds the
material strength necessary to withstand the gas pressure.

The window was fabricated in two steps:

\begin{enumerate}
	\item lamination of 1.6~m wide Tedlar/Mylar/Tedlar rolls
	\item seaming of the laminated strips into a 4.8~m $\times$ 4.8~m window
\end{enumerate}

The lamination of the composite material, with dimensions outlined in \F{windowDesign} (top), was performed at
Madico~\cite{madico}, where a sheet 400~m long was produced.

At Jefferson Lab rectangles were cut out of the laminated sheet, each 1.6~m wide and 4.8~m long. To form a final
4.8~m $\times$ 4.8~m single LTCC window, three of the rectangles were seamed together using G/Flex 655 epoxy.
The seam was load tested to withstand a pressure 10 times higher than that expected from the gas flow and gas weight.

\subsubsection{Window Installation and Gas Leak Tests}

The installation of the window onto the box was achieved through gluing the window on the box sides using G/Flex 655
epoxy. The width of the window attached with glue was 12~cm, to provide sufficient gluing area. A photograph of the
downstream window after installation is shown in \F{downstreamWindow}.

\begin{figure}
	\centering
	\includegraphics[width=1.0\columnwidth,keepaspectratio]{img/downstreamWindow.png}
	\caption{The downstream window of one LTCC sector during curing of the epoxy. The yellow strips protect the
          window seaming.}
	\label{fig:downstreamWindow}
\end{figure}

After curing of both the upstream and downstream windows, the LTCC box was filled with nitrogen gas to a pressure of
2~in of water. Freon gas was pumped into the box and leaks were detected using a refrigerant leak detector. After the
leaks were sealed, the box was pressurized for a 48 hour period to test the overall box gas tightness. This procedure
was repeated after every movement of the LTCC boxes, as small shifts of the frame walls had the potential to introduce
additional leaks.
