%
%---Beginning of Document---
%
% Use 1). latex ftof-nim.tex, 2). dvipdfm ftof-nim
%
%\RequirePackage{lineno}
\documentclass{elsart}
\usepackage[dvips]{color,graphics}
\usepackage[pdftex]{graphicx}
%\setcounter{footnote}{0}
\setcounter{tocdepth}{4}
%
\usepackage{lineno} 
\linenumbers

\begin{document}
\begin{frontmatter}

\title{The Forward Time-of-Flight System for CLAS12}

\author[JLab]{D.S. Carman\thanksref{corresponding}},
\author[USC]{R.W. Gothe},
\author[Glasgow]{L. Clark},
\author[INFN]{R. De Vita},
\author[JLab]{B. Miller}, and
\author[JLab]{C. Wiggins}

\address[JLab]{Thomas Jefferson National Accelerator Facility, Newport News, VA 23606, USA}
\address[USC]{University of South Carolina, Columbia, SC 29208, USA}
\address[Glasgow]{University of Glasgow, Glasgow G12 8QQ, United Kingdom}
\address[INFN]{INFN, Sezione di Genova, 16146 Genova, Italy}
\thanks[corresponding]{Corresponding author. Address: 12000 Jefferson Ave., Newport News, VA; 
e-mail: carman@jlab.org.}

\date{\today}

%\maketitle

\begin{abstract}
The Forward Time-of-Flight system for the large-acceptance CLAS12 spectrometer in Hall~B at the
Thomas Jefferson National Accelerator Facility is described. The system is positioned at distances
in the range from 6.2~m to 7.2~m from the beam-target interaction point and spans laboratory polar
angles from $5^\circ \to 35^\circ$ and nearly the full azimuth. The system consists of 540 individual
scintillation counters with double-ended readout that range in length from 17~cm to 426~cm of
discrete widths of 6~cm, 15~cm, and 22~cm, and of discrete thicknesses of 5~cm and 6~cm. The
effective counter time resolution for passing charged particles varies from 50~ps for the shortest
counters at small angles to 200~ps for the longest counters at large angles. The detectors are part
of the forward-angle particle identification system for CLAS12 during offline event reconstruction
and are a component of the online data acquisition trigger to select final state event topologies with
forward-going charged particles.
\end{abstract}

\end{frontmatter}

PACS:29.40.Mc \\
Keywords: CLAS12, time of flight, plastic scintillator, particle identification
\newpage

%\newpage
%\tableofcontents

%\vfil
%\eject

\section{Overview of CLAS12}

The Thomas Jefferson National Accelerator Facility (JLab) recently completed a project to 
double the maximum energy of its electron accelerator from 6~GeV to 12~GeV. The experimental 
equipment in Hall~B forms the large-acceptance CLAS12 spectrometer that is designed to operate 
with beam energies up to 11~GeV at a beam-target luminosity of up to $10^{35}$~cm$^{-2}$s$^{-1}$
to allow for precision measurements of exclusive reactions with polarized beam and both unpolarized
and polarized targets. This spectrometer is based on two superconducting magnets, a solenoid in the
central region around the target and a toroid at forward angles. See Ref.~\cite{clas12-nim} for more
complete information on CLAS12.

The CLAS12 torus magnet has a six-fold symmetry that divides the forward acceptance in the polar angle
range from 5$^\circ$ to 35$^\circ$ into six 60$^\circ$-wide sectors. It produces a field primarily in
the azimuthal direction. The torus $\int \!B d\ell$ field strength at its nominal full current is 2.8~T
at 5$^\circ$ and 0.5~T at 35$^\circ$. A set of three multi-layer drift chambers in each sector (before
the field, within the field, and after the field) and a forward micromegas vertex tracker are used for
charged particle tracking to measure momenta. Downstream of the torus each sector is instrumented with
a Cherenkov counter for $\pi/K$ separation (four sectors are instrumented with low threshold gas Cherenkov
counters, one sector is instrumented with a ring-imaging Cherenkov counter, and the final sector will eventually
be instrumented with a second ring-imaging Cherenkov counter), three planar layers of scintillation counters
for charged particle time measurements called the Forward Time-of-Flight (FTOF) system, and an
electromagnetic calorimeter system for electron and neutral particle identification. Just upstream of the torus
is a large-volume high-threshold gas Cherenkov counter for electron identification and a tagging system to
detect electrons and photons at scattering angles below $5^\circ$.

The CLAS12 solenoid spans the central angular range from 35$^\circ$ to 125$^\circ$ and has a uniform
5~T central field at its nominal full current. The solenoid serves to focus the low-energy M{\o}ller
background down the beam pipe to the beam dump away from the acceptance of the spectrometer. The
detectors mounted within the solenoid include a thick scintillation counter for neutron identification, a
barrel of thin scintillation counters for charged particle timing measurements, and a set of tracking
detectors around the target. Upstream of the solenoid, covering angles from 135$^\circ$ to 145$^\circ$,
a wall of scintillation counters is installed for additional neutron detection coverage.

Figure~\ref{clas12-model} shows a model representation of CLAS12 to highlight its overall layout and scale.
The CLAS12 spectrometer was installed and instrumented in Hall~B in the period from 2012 to 2018. The
CLAS12 spectrometer took the place of the original CLAS spectrometer~\cite{clas-nim} that operated in
Hall~B in the period from 1997 to 2012 when it was decommissioned.

%%%%%%%%%%%%%%%%%%%%%%%%%%%%%%%%%%%%%%%%%%%%%%%%%%%%%%%%%
\begin{figure}[t]
\vspace{6.0cm}
\begin{picture}(50,50) 
\put(20,250)
{\hbox{\includegraphics[width=0.70\textwidth,natwidth=610,natheight=642,angle=-90]{pics/ftof_clas12.pdf}}}
\end{picture} 
\caption{(Color Online) Model representation of the CLAS12 spectrometer in Hall~B at Jefferson Laboratory.
The electron beam is incident from the left side of this figure. The CLAS12 detector is roughly 20~m in
scale along the beam axis. For further details on the individual subsystems that make up the CLAS12
spectrometer see Ref.~\cite{clas12-nim}.}
\label{clas12-model}
\end{figure}
%%%%%%%%%%%%%%%%%%%%%%%%%%%%%%%%%%%%%%%%%%%%%%%%%%%%%%%%%

This paper focuses on the CLAS12 FTOF detector system and is organized as follows:
Section~\ref{clas12-fd-pid} reviews the scheme for particle identification in the CLAS12 Forward
Detector, Section~\ref{sec:overview} provides a high-level overview of the FTOF system and its
overall design requirements, Section~\ref{sec:design} provides a technical description of the system
design, and Section~\ref{sec:performance} highlights the performance of the system through both
bench testing with cosmic rays, as well as during the 2017 commissioning run and 2018 first production
running with electrons. Finally, Section~\ref{sec:summary} provides a summary of the FTOF system for
CLAS12.

\section{CLAS12 Forward Detector Particle Identification}
\label{clas12-fd-pid}

Particle identification in the CLAS12 Forward Detector relies on input from each of the different
Forward Detector subsystems. A reconstructed track in the drift chambers (DC) identifies the
presence of a charged particle and is used as a veto for forward-going neutral particles. The curvature
of the particle tracks in the magnetic field of the torus provides the electric charge and momentum.
The other detector subsystems are used to identify the particle type. These subsystems include the
different Cherenkov counters (HTCC, LTCC, and RICH), the electromagnetic calorimeters (ECAL), and
the FTOF. These systems are used as part of the overall CLAS12 particle identification scheme to
separate the different particle species as a function of momentum. Figure~\ref{clas12-pid} highlights
the different detector subsystems used for charged particle identification of forward-going charged
particles in CLAS12 and the range of momenta for which they are responsible for the separation of the
different particle species.

%%%%%%%%%%%%%%%%%%%%%%%%%%%%%%%%%%%%%%%%%%%%%%%%%%%%%%%%%
\begin{figure}[htbp]
\vspace{6.0cm}
\begin{picture}(50,50) 
\put(25,-35)
{\hbox{\includegraphics[width=0.70\textwidth,natwidth=610,natheight=642]{pics/clas12-pid-chart.pdf}}}
\end{picture} 
\caption{(Color Online) CLAS12 Forward Detector subsystems used for particle identification and the
momentum range where they are designed to be effective in separating different particle species.}
\label{clas12-pid}
\end{figure}
%%%%%%%%%%%%%%%%%%%%%%%%%%%%%%%%%%%%%%%%%%%%%%%%%%%%%%%%%

CLAS12 consists of three different Cherenkov counters in the Forward Detector. A CO$_2$-filled
High Threshold Cherenkov Counter (HTCC) is positioned just downstream of the solenoid magnet. The
HTCC is used in conjunction with the electromagnetic calorimeters (ECAL) to identify scattered electrons
over their full momentum range. The HTCC threshold for pions is $\sim$5~GeV and is used for $e/\pi$
discrimination up to this threshold. The HTCC is also used for $\pi/K$ and $\pi/p$ separation from
$\sim$5-9~GeV. $C_4F_{10}$-filled Low Threshold Cherenkov Counters (LTCC) are included in four
of the six CLAS12 sectors just upstream of the FTOF. This system is used for $\pi/K$ and $\pi/p$
separation for particle momenta in the range from $\sim$3-9~GeV. Finally, an aerogel-based Ring Imaging
Cherenkov (RICH) detector is included in one sector of CLAS12 and is used for $\pi/K$, $\pi/p$, and $K/p$
separation in the range of momenta from $\sim$3-8~GeV (with a second RICH unit for the final sector now
under construction).

However, the FTOF is the primary system for particle identification in CLAS12 for forward-going charged
particles. The main requirement for this system is that its counters provide excellent timing resolution for
particle identification. The FTOF  was designed to measure the flight time of charged particles emerging
from the target with an average time resolution of 80~ps. Given this nominal time resolution for the
counters, the momentum threshold for particle identification can be defined. For our purposes, thresholds
are quoted at the 4$\sigma$ level for FTOF, which amounts to the momenta where particle identification
can occur with up to an order of magnitude difference in the relative yields of the different species. The
time resolution is illustrated by computing the flight time differences between different charged particle
species, pions, kaons, and protons, for tracks normally incident on the detector. Figure~\ref{tdiff} shows
the computed time differences as a function of momentum. Where the 4$\sigma$ line crosses the computed
time difference curves defines the momentum limit for particle identification for each particle species.
These limits are quoted as 2.8~GeV for $\pi/K$ separation, 4.8~GeV for $K/p$ separation, and 5.4~GeV
for $\pi/p$ separation.  

%%%%%%%%%%%%%%%%%%%%%%%%%%%%%%%%%%%%%%%%%%%%%%%%%%%%%%%%%
\begin{figure}[htbp]
\vspace{5.0cm}
\begin{picture}(50,50) 
\put(35,-105)
{\hbox{\includegraphics[width=1.0\textwidth,natwidth=610,natheight=642]{pics/tdiff_sep18.pdf}}}
\end{picture} 
\caption{(Color Online) Time differences (ns) between protons and pions, between protons and kaons,
and between kaons and pions (as indicated) for a 7~m path length from the target to the FTOF system
plotted vs. particle momentum (GeV).  The horizontal line indicates a time difference four times larger
than the average FTOF counter design resolution of $\sigma_{TOF} \approx$ 80~ps. The vertical lines
that meet each curve represent the momentum where 4$\sigma$ particle species separation is quoted.}
\label{tdiff}
\end{figure}
%%%%%%%%%%%%%%%%%%%%%%%%%%%%%%%%%%%%%%%%%%%%%%%%%%%%%%%%%

Figure~\ref{pth-kin} illustrates the momentum versus polar angle coverage in CLAS12 from beam
data of a 10.6~GeV electron beam incident upon a liquid-hydrogen target. The plots show the kinematic
phase space for scattered electrons (left) and charged pions (right) for the semi-inclusive reactions
$ep \to e'\pi^{\pm}X$ ($X$ represents all other possible reaction products). For these reactions the
typical charged hadron track momenta accepted by FTOF are in the range from 1~GeV to 6~GeV.

%%%%%%%%%%%%%%%%%%%%%%%%%%%%%%%%%%%%%%%%%%%%%%%%%%%%%%%%%
\begin{figure}[ht]
\vspace{5.0cm}
\begin{picture}(50,50) 
\put(-5,-55)
{\hbox{\includegraphics[width=0.55\textwidth,natwidth=610,natheight=642]{pics/pthe.pdf}}}
\put(190,-55)
{\hbox{\includegraphics[width=0.55\textwidth,natwidth=610,natheight=642]{pics/pthpi.pdf}}}
\end{picture} 
\caption{(Color Online) Plot of momentum vs. lab polar angle from beam data for a 10.6~GeV electron
  beam incident upon a liquid-hydrogen target in CLAS12 for scattered electrons (left) and $\pi^{\pm}$
  (right). The discontinuity at $\theta=35^\circ$ is due to the small acceptance gap between the
  Forward and Central Detectors. The typical momentum of charged hadron tracks in the Forward
  Detector in these kinematics is between 1~GeV and 6~GeV.}
\label{pth-kin}
\end{figure}
%%%%%%%%%%%%%%%%%%%%%%%%%%%%%%%%%%%%%%%%%%%%%%%%%%%%%%%%%

\section{Overview of the FTOF System}
\label{sec:overview}

The Forward Time-of-Flight System (FTOF) is a major component of the CLAS12 Forward Detector
used to measure the time-of-flight of charged particles emerging from interactions in the target.
The requirements for FTOF include excellent time resolution for charged particle identification and
good segmentation to minimize count rates and to provide for flexible trigger options. The system
specifications call for an average time resolution of $\sigma_{TOF}$=80~ps at the more forward angles
of CLAS12 and 150~ps at angles larger than 35$^\circ$. The system must also be capable of operating
in a high-rate environment where the maximum count rate for each FTOF scintillator at an operating
luminosity of $1 \times 10^{35}$~cm$^{-2}$s$^{-1}$ is less than 1~MHz.

In each of the six sectors of the CLAS12 Forward Detector, the FTOF system is comprised of three
arrays of counters, referred to as panels, named panel-1a, panel-1b, and panel-2. Each panel consists
of a set of rectangular scintillators with a photomultiplier tube (PMT) on each end. Panel-1 refers to
the counters located at forward angles (5$^\circ$ to 35$^\circ$) (where two panels are employed to
meet the 80~ps average time resolution requirement) and panel-2 refers to the sets of counters at
larger angles ($35^\circ$ to 45$^\circ$). The positioning and attachment of the FTOF detector arrays
on their Forward Carriage supports are shown in Fig.~\ref{fwd_car}.

%%%%%%%%%%%%%%%%%%%%%%%%%%%%%%%%%%%%%%%%%%%%%%%%%%%%%%%%%
\begin{figure}[htbp]
\vspace{5.2cm}
\begin{picture}(50,50) 
\put(105,-35)
{\hbox{\includegraphics[width=0.50\textwidth,natwidth=610,natheight=642]{pics/fwd_carriage.pdf}}}
\end{picture} 
\caption{(Color Online) View of the FTOF counters for CLAS12 highlighting the location of the panel-1 and
panel-2 counters. The panel-1b counter arrays are shown in blue and the panel-2 counter arrays, mounted
around the perimeter of the Forward Carriage, are shown in orange. The panel-1a counter arrays, mounted
just downstream of the panel-1b arrays, are not visible in this picture. The Forward Carriage is roughly
10~m in diameter.} 
\label{fwd_car}
\end{figure}
%%%%%%%%%%%%%%%%%%%%%%%%%%%%%%%%%%%%%%%%%%%%%%%%%%%%%%%%%

The FTOF counters in the angular range from 5$^\circ$ to 35$^\circ$ consist of two sets of six 
triangular arrays. Just upstream of the electromagnetic calorimeter (ECAL) detectors, the panel-1a
arrays are mounted. These detector sets are the refurbished panel-1 TOF counters from the
decommissioned CLAS spectrometer~\cite{tof-nim}. Upstream of the panel-1a arrays the new panel-1b
arrays are mounted. In the event reconstruction the hit times for panel-1a and panel-1b are combined
together to determine the charged particle hit time (see Section~\ref{cluster}). In the angular range
from 35$^\circ$ to 45$^\circ$ the panel-2 arrays are mounted. These counters are refurbished panel-2
counters of the CLAS TOF system. A detailed summary of the FTOF technical parameters is given in
Table~\ref{spec-table}. 

%%%%%%%%%%%%%%%%%%%%%%%%%%%%%%%%%%%%%%%%%%%%%%%%%%%%%%%%%
\begin{table}[t]
\begin{center}
\begin{tabular} {c|l} \hline
~~Parameter~~ &~~~~~~~~~~~~~~~~~~~~~~ Design Value ~~~~~~~~~~\\ \hline \hline
\multicolumn{2}{l} {\bf Panel-1a} \\ \hline
Angular Coverage      & $\theta = 5^\circ \to 35^\circ$, $\phi: 50\% {\rm ~at~} 5^\circ \to 85\% {\rm ~at~} 
35^\circ$ \\ \hline
Counter Dimensions   & $L = 32.3$~cm $\to$ 376.1~cm, $w \times h$ = 15~cm $\times$ 5~cm   \\ \hline
Scintillation Material & BC-408   \\ \hline
PMTs                         & EMI 9954A, Philips XP2262 \\ \hline
Design Resolution     & 90~ps $\to$ 160~ps   \\ \hline \hline
\multicolumn{2}{l} {\bf Panel-1b} \\ \hline
Angular Coverage      & $\theta = 5^\circ \to 35^\circ$, $\phi: 50\% {\rm ~at~} 5^\circ \to 85\% {\rm ~at~} 
35^\circ$ \\ \hline
Counter Dimensions   & $L = 17.3$~cm $\to$ 407.9~cm, $w \times h$ = 6~cm $\times$ 6~cm   \\ \hline
Scintillation Material & BC-404 (\#1 $\to$ \#31), BC-408 (\#32 $\to$ \#62)  \\ \hline
PMTs                         & Hamamatsu R9779 \\ \hline
Design Resolution     & 60~ps $\to$ 110~ps   \\ \hline \hline
\multicolumn{2}{l} {\bf Panel-2} \\ \hline
Angular Coverage      & $\theta = 35^\circ \to 45^\circ$, $\phi: 85\% {\rm ~at~} 35^\circ \to 95\% {\rm ~at~} 
45^\circ$ \\ \hline
Counter Dimensions   & $L = 371.3$~cm $\to$ 426.1~cm, $w \times h$ = 22~cm $\times$ 5~cm   \\ \hline
Scintillation Material & BC-408   \\ \hline
PMTs                         & Photonis XP4312B, EMI 4312KB \\ \hline
Design Resolution     & 145~ps $\to$ 160~ps   \\ \hline
\end{tabular}
\caption{Parameters for the scintillators, PMTs, and counters for the FTOF panel-1a, panel-1b, 
and panel-2 arrays.}
\label{spec-table}
\end{center}
\end{table}
%%%%%%%%%%%%%%%%%%%%%%%%%%%%%%%%%%%%%%%%%%%%%%%%%%%%%%%%%

The panel-1 arrays consist of the old CLAS panel-1 TOF counters (called panel-1a) and a new set of panel-1
counters (called panel-1b).  The panel-1a arrays consist of 23 scintillators, each measuring 5.08-cm thick
and 15-cm wide.  The lengths of these counters range from 32~cm at the smallest scattering angles to
376~cm at the largest scattering angles.  The scintillators are constructed from Bicron~\cite{bicron}
BC-408 and are coupled to short acrylic light guides read out with 2-in Thorn EMI-9954A PMTs.  The new
panel-1b arrays consist of 62 scintillators constructed from Bicron BC-404 scintillator for the shorter
counters (first half of the arrays) and BC-408 for the longer counters (second half of the arrays), each
6-cm wide and 6-cm thick. The lengths of these counters range from 17~cm at the smallest scattering
angles to 408~cm at the largest scattering angles. The scintillators are read out by 2-in Hamamatsu R9779
PMTs~\cite{hamamatsu} coupled directly to the scintillation bars. This new panel is mounted to the Forward
Carriage in front of the panel-1a counters. The detailed design and bench testing results for these counters
is described in detail in Ref.~\cite{nim-p1b}.

The panel-2 arrays consist of selected counters from the old CLAS panel-2 TOF counters, and include 5
22-cm wide, 5.08-cm thick scintillators in each sector.  The length of these counters ranges from roughly
370~cm to 440~cm.  These scintillators are constructed from Bicron BC-408 and are read out through
curved acrylic light guides coupled to 3-in Philips XP4312B PMTs. These scintillators are included to give
complete acceptance for outbending charged particles incident upon the CLAS12 drift chambers.  

\section{FTOF System Design}
\label{sec:design}

In order to meet the performance and mechanical requirements for the FTOF system, the major
considerations in its design included the system geometry and areal coverage, the counter and system
time resolutions, the system components, and the design and materials associated with its mechanical
support structure in the active area of the spectrometer. These system design elements are described
in the following subsections. In addition, this section also includes information on the readout electronics
and the high voltage system used for the FTOF.

\subsection{Geometry}
\label{ftof-geometry}

The projected space behind the coils of the CLAS12 torus as defined by straight lines projecting from
the center of the nominal target position radially outward and is referred to downstream of the torus as
its shadow region. This region is inactive and defines the space available for locating the light guides,
PMTs, voltage dividers, and signal and high voltage cables. The remaining area in the forward direction is
the sensitive fiducial region of the detector and must be covered by scintillation counters. The design
specification for FTOF called for a minimum of 50\% azimuthal acceptance at 5$^\circ$ increasing to
95\% at 45$^\circ$.

%%%%%%%%%%%%%%%%%%%%%%%%%%%%%%%%%%%%%%%%%%%%%%%%%%%%%%%%%
\begin{figure}[htbp]
\vspace{4.7cm}
\begin{picture}(50,50) 
\put(0,-25)
{\hbox{\includegraphics[width=0.45\textwidth,natwidth=610,natheight=642]{pics/fwd_shadow1.pdf}}}
\put(180,180)
{\hbox{\includegraphics[width=0.45\textwidth,natwidth=610,natheight=642,angle=-90]{pics/fwd_shadow2.pdf}}}
\end{picture} 
\caption{(Color Online) (Left) View of the shadows created by the main torus cryostats and drift
chamber endplates as projected on the face of the FTOF system.  (Right) The defined active area
between the shadow projections through the three regions of drift chambers projected onto the
face of the FTOF in a representative Forward Carriage sector.}
\label{shadow}
\end{figure}
%%%%%%%%%%%%%%%%%%%%%%%%%%%%%%%%%%%%%%%%%%%%%%%%%%%%%%%%%

Figure~\ref{shadow} provides an illustration of the shadow region projected onto the Forward Carriage
created primarily by the torus cryostats and the drift chamber endplates as projected onto the face of
the FTOF system. Figure~\ref{shadow}(left) shows a picture of the shadow bands on the Forward
Carriage that defines a uniform gap of $\sim$40~cm between each sector. Figure~\ref{shadow}(right)
shows the defined active region in one sector of CLAS12 on the face of the FTOF. The azimuthal width
of this area at the position of the Forward Carriage in Hall~B essentially defined the length of the
scintillation counters. The final limits of the shadow region at the location of FTOF are actually defined
by the endplates of the drift chamber system located within the torus coils. The drift chamber systems
upstream of the torus and downstream of the torus have their endplates, on-board electronics, and readout
cables located mainly in the shadow of the torus cryostats.

The panel-1a and panel-1b FTOF arrays in each sector are triangular in shape with the shortest counters
located closest to the beamline and the longest counters furthest from the beamline. The length of each
counter for a given counter number $N_{counter}$ is as follows:

\begin{itemize}
\item Panel-1a: $L$ (cm) = $15.85 \times N_{counter} + 16.43$ ~~($N_{counter} = 1 \to 5$), 
\item Panel-1a: $L$ (cm) = $15.85 \times N_{counter} + 11.45$ ~~($N_{counter} = 6 \to 23$), 
\item Panel-1b: $L$ (cm) =$ 6.40 \times N_{counter} + 10.84$ ~~($N_{counter} = 1 \to 62$).
\end{itemize}

The panel-1a and panel-1b arrays are tilted toward the target at an angle of 25$^{\circ}$ consistent with
the other subsystems in the CLAS12 Forward Detector (DC, LTCC, RICH, ECAL). The panel-1a counters
are located at a radial distance from the target in the range from $R$=724.21~cm for $N_{counter} = 1$ to
$R$=691.74~cm for $N_{counter} =23$. The panel-1b counters are located at a radial distance from the
target in the range from $R$=716.15~cm for $N_{counter} = 1$ to $R$=677.97~cm for $N_{counter}=62$.
The gap between the coplanar panel-1b and panel-1a arrays in each sector is 10.72~cm. The minimum angle
covered by panel-1a based on a straight line from the target is 5.453$^\circ$. The corresponding minimum
angle covered by panel-1b is 3.667$^\circ$. Each of the panel-1a arrays covers an area of 7.0~m$^2$ and
each of the panel-1b arrays covers an area of 7.9~m$^2$. Figure~\ref{side-view} shows a two-dimensional
schematic of the layout and positioning of the arrays defining the key geometry parameters, which are
listed in Table~\ref{geom-parms}. See Ref.~\cite{ftof-geom} for more information.

The panel-2 arrays are mounted radially outward of the panel-1a and panel-1b arrays as shown in
Fig.~\ref{side-view}. The length of each counter for a given counter number $N_{counter}$ is as follows:

\begin{itemize}
\item Panel-2: $L$ (cm) = $13.73 \times N_{counter} + 357.55$~~ ~~($N_{counter} = 1 \to 5$).
\end{itemize}

The panel-2 arrays are tilted toward the target at an angle of 58.11$^\circ$. The minimum angle covered
by panel-2 based on a straight line from the target is 34.698$^\circ$. Each of the six panel-2 arrays
covers an area of 4.4~m$^2$. Note that the panel-2 arrays are not within in the CLAS12 acceptance as
seen by straight lines from the center of the target. However, due to the presence of the toroidal
magnetic field, they provide additional acceptance for low momentum tracks  and for tracks associated
with strange particles that decay in-flight after emerging from the target (e.g. $\Lambda \to N \pi$
with $c \tau = 7.89$~cm).

%%%%%%%%%%%%%%%%%%%%%%%%%%%%%%%%%%%%%%%%%%%%%%%%%%%%%%%%%
\begin{figure}[htbp]
\vspace{5.5cm}
\begin{picture}(30,50) 
\put(0,-50)
{\hbox{\includegraphics[width=0.75\textwidth,natwidth=610,natheight=642]{pics/side-view.pdf}}}
\end{picture} 
\caption{View of the FTOF scintillators for panel-1a, panel-1b, and panel-2 in the sector mid-plane for one
representative sector of the CLAS12 Forward Detector with the key parameters indicated.}
\label{side-view}
\end{figure}
%%%%%%%%%%%%%%%%%%%%%%%%%%%%%%%%%%%%%%%%%%%%%%%%%%%%%%%%%

%%%%%%%%%%%%%%%%%%%%%%%%%%%%%%%%%%%%%%%%%%%%%%%%%%%%%%%%%
\begin{table}[htbp]
\begin{center}
\begin{tabular} {c|c|c|c} \hline
Parameter & Panel-1a &  Panel-1b & Panel-2 \\ \hline
R\_min      & 726.689~cm & 717.236~cm & 659.71~cm \\ \hline
th\_min    & 5.453$^\circ$ & 3.667$^\circ$ & 34.698$^\circ$ \\ \hline
th\_tilt    & 25.00$^\circ$ & 25.00$^\circ$ & 58.11$^\circ$ \\ \hline
thick        & 5.08~cm           & 6.00~cm         & 5.08~cm \\ \hline
width       & 15.00~cm         & 6.00~cm         & 22.00~cm \\ \hline
gap\_1a\_1b & \multicolumn{2}{c|}{10.717~cm} &  -- \\ \hline
\end{tabular}
\caption{Nominal geometry parameters for the CLAS12 FTOF detector system.}
\label{geom-parms}
\end{center}
\end{table}
%%%%%%%%%%%%%%%%%%%%%%%%%%%%%%%%%%%%%%%%%%%%%%%%%%%%%%%%%%

Given the active area coverage requirements for the FTOF system within each CLAS12 sector on the
Forward Carriage, another key aspect of the geometry associated with the FTOF system design is
the width of the individual scintillation counters. An essential optimization was made to select the
counter width to reduce the number of readout channels, while considering the overall count rates per
bar at the nominal luminosity associated with incident charged and neutral particles including photons.
These rates must allow for reasonable PMT anode currents that do not affect the stability of the PMT
response in terms of pulse shape or saturation effects, nor lead to unreasonably short PMT lifetimes.
In addition, the width of the scintillation bars determines the granularity of the scattering angle definition
in the trigger and its matching to the projected tracks from the drift chambers to the electromagnetic
calorimeters. Note that the 15-cm widths of panel-1a and the 22-cm widths of panel-2 of the existing
refurbished counters of the CLAS TOF system were optimized for nominal beam-target luminosities a
factor of 10 lower than for CLAS12. The 6-cm widths of the newly constructed panel-1b counters were
optimized for the higher rate operating conditions of CLAS12.

\subsection{Time Resolution}
\label{res-sec}

Time-of-flight detectors are designed to provide an output signal for the data acquisition system that
reflects the time a charged particle passed through the scintillation counter. As the particle passes
through the scintillation material, it causes ionization that subsequently generates scintillation light. The
photons that are created travel on various paths inside of the scintillator and light guide (if present), and
may get absorbed or reflected (internally or on outer wrapping materials) before they ultimately impinge
on the photocathode of the PMT. This interaction produces a photoelectron signal that is amplified within
the stages of the PMT and the generated pulse is then input into the readout electronics, which includes a
discriminator and a time-to-digital (TDC) converter. The net effect of these different processes accounts
for the time resolution of the counter $\sigma_{TOF}$.

The contributions to the time resolution of TOF systems have been parameterized in Ref.~\cite{kuhlen}
by:

\begin{equation}
\label{timing-func}
\sigma_{TOF} = \sqrt{\sigma_0^2 + \frac{\sigma_1^2 + (\sigma_2 L/2)^2} {N_{pe}}}.
\end{equation}

Here $\sigma_0$ represents the intrinsic electronic resolution of the measurement system. This
represents a floor-term contribution that is independent of the light level. The remaining terms
$\sigma_1$ and $\sigma_2$ are directly dependent on the photo-statistics at the PMT photocathode
$N_{pe}$ (see Eq.(\ref{nphe-eq})). The term $\sigma_1$ models the jitter in the combined single
photoelectron response of the scintillation counter and its PMTs and the term $\sigma_2$ accounts
for path length variations in the light collection.  These path length variations in the scintillator scale
with the distance from the source to the PMT, which we take to be half the length of the counter
($L/2$), since the scintillators are read out at either side.  The statistical behavior of the last two
terms is indicated by scaling the single-photoelectron responses by $\sqrt{N_{pe}}$. For scintillators
that are several meters long, the dominant contribution comes from transit time variations of photon
paths in the scintillator to the PMT.

The values of the parameters $\sigma_0$, $\sigma_1$, and $\sigma_2$ for the panel-1a and panel-2
counters are given in Ref.~\cite{tof-nim}, where the above functional form with the parameters listed
in Table~\ref{timing-parms} was found to describe the measured data. A direct extension of these
parameters is assumed to be reasonable for estimating the time resolution for the panel-1b counters.
A summary of all parameters employed are listed in Table~\ref{timing-parms}. Note that due to
improvements in the resolution of the readout electronics for the CLAS12 FTOF system compared to
the CLAS TOF system, the floor-term $\sigma_0$ has been reduced from 62~ps to 40~ps. 

%%%%%%%%%%%%%%%%%%%%%%%%%%%%%%%%%%%%%%%%%%%%%%%%%%%%%%%%%
\begin{table}[htbp]
\begin{center}
\begin{tabular} {c|c} \hline
Parameter    & Nominal Value\\ \hline
$\sigma_0$ & 0.062~ns [CLAS TOF]; 0.040~ns [CLAS12 FTOF] \\ \hline
$\sigma_1$  & 2.1~ns [panel-1a/1b]; 2.0~ns [panel-2] \\ \hline
$\sigma_2$  & 2.0~ns \\ \hline
$N_{pe}^0$   & 918 \\ \hline
$\lambda$   & $0.358\cdot L + 81.725$~cm \\ \hline
\end{tabular}
\caption{Parameters determined for the CLAS TOF panel-1a and panel-2 counters in Ref.~\cite{tof-nim}
and used for a parameterization of the CLAS12 FTOF counters using the functional form for $\sigma_{TOF}$
in Eq.(\ref{timing-func}) and for $N_{pe}$ in Eq.(\ref{nphe-eq}).}
\label{timing-parms}
\end{center}
\end{table}
%%%%%%%%%%%%%%%%%%%%%%%%%%%%%%%%%%%%%%%%%%%%%%%%%%%%%%%%%%

The number of photoelectrons $N_{pe}$ in Eq.(\ref{timing-func}) for panel-1a and panel-2 at the PMT
photocathode was determined in Ref.~\cite{tof-nim} by:

\begin{equation}
\label{nphe-eq}
N_{pe} = N_{pe}^0 {\rm exp} \left( \frac{L_0}{2 \lambda_0} - \frac{L}{2 \lambda} \right) \cdot F,
\end{equation}

\noindent
where $N_{pe}$ for all counters was referenced to the average value measured for the response of
the shortest panel-1a counter $N_{pe}^0$ of length $L_0=32$~cm with attenuation length $\lambda_0$.
For the panel-2 counters, $N_{pe}$ is further scaled by the factor $F = 0.9$ to account for light
collection efficiencies at the end of the larger panel-2 counters with their 3-in PMTs and longer light
guides compared to the smaller panel-1a PMTs with their relatively short light guides~\cite{tof-nim}. For
the panel-1b counters, $N_{pe}$ is determined as for panel-1a using Eq.(\ref{nphe-eq}) by scaling by the
ratio of the cross sectional areas of the scintillation bars (15~cm $\times$ 5~cm vs. 6~cm $\times$ 6~cm). 

Figure~\ref{sigma_tof} shows the parameterized resolution for the counters in panel-1a, panel-1b,
and panel-2 as a function of counter length. The Forward Detector event reconstruction and particle
identification uses time information from both panel-1a and panel-1b. For tracks that pass through both
arrays the combined time information (described in Section~\ref{cluster}) is used and results in a 20\%
improvement compared to using the hit information from panel-1b alone.

%%%%%%%%%%%%%%%%%%%%%%%%%%%%%%%%%%%%%%%%%%%%%%%%%%%%%%%%%
\begin{figure}[htbp]
\vspace{4.4cm}
\begin{picture}(50,50) 
\put(25,-60)
{\hbox{\includegraphics[width=1.0\textwidth,natwidth=610,natheight=642]{pics/resolution.pdf}}}
\end{picture} 
\caption{(Color Online) Parameterized expectation of the counter hit time resolution for the FTOF panel-1a
(dotted), panel-1b (dashed), and panel-2 (dot-dashed) counters as a function of length. The solid (red) line
indicates the final expected resolution in the forward direction by combining the hit time information from
the panel-1a and panel-1b counters. The horizontal line indicates the 80-ps average time resolution
specification for the FTOF system.}
\label{sigma_tof}
\end{figure}
%%%%%%%%%%%%%%%%%%%%%%%%%%%%%%%%%%%%%%%%%%%%%%%%%%%%%%%%%

\subsection{System Components}

\subsubsection{Scintillator Material}

To optimize the time resolution over the full volume of the FTOF counters, scintillation materials with
fast time response and long attenuation length are essential. For the panel-1a and panel-2 FTOF counters
that were refurbished from the older CLAS TOF system, Bicron BC-408 was selected. For the panel-1b
counters constructed for the new CLAS12 FTOF system, a different design approach was considered
that optimized the overall system time resolution. For counters less than 2~m in length, the overall
performance is improved by the use of a faster scintillator with a small decay time $\tau_{decay}$, whereas
for the longer counters, a material with a longer attenuation length is the better choice. The final decision
for the panel-1b counters was to use BC-404 for counters 1 $\to$ 31 (lengths from 17.3~cm to 209.4~cm)
and BC-408 for counters 32 $\to$ 62 (lengths from 215.8~cm to 407.9~cm). Table~\ref{scint-specs}
lists the properties of the FTOF scintillation materials.

%%%%%%%%%%%%%%%%%%%%%%%%%%%%%%%%%%%%%%%%%%%%%%%%%%%%%%%%%
\begin{table}[htbp]
\begin{center}
\begin{tabular}{c|c|c} \hline
Property                                    & BC-404     & BC-408  \\ \hline    
Light Output, \% Anthracene  & 68             &  64    \\ \hline
Rise Time (ns)                           & 0.7             & 0.9    \\ \hline
Decay Time (ns)                         & 1.8             & 2.1     \\ \hline
Pulse Width, FWHM (ns)         & 2.2              & 2.5    \\ \hline
Wavelength of maximum emission (nm) & 408    & 425 \\ \hline
Light attenuation length (cm)   & 140             & 210   \\ \hline
Bulk attenuation length (cm)     & 160             & 380  \\ \hline
Polymer base                             & \multicolumn{2}{c}{Polyvinyltoluene} \\ \hline
Refractive index                       & \multicolumn{2}{c}{1.58}                    \\ \hline 
Density (g/cm$^3$)                  & \multicolumn{2}{c}{1.023}                    \\ \hline 
\end{tabular}
\end{center}
\caption{Properties of the plastic scintillation material BC-404 and BC-408 employed for the counters
of the FTOF system~\cite{scint-mat-ref}.}
\label{scint-specs}
\end{table}
%%%%%%%%%%%%%%%%%%%%%%%%%%%%%%%%%%%%%%%%%%%%%%%%%%%%%%%%%

The bulk attenuation length of the scintillation material is stated by its manufacturer to be 160~cm
for BC-404 and 380~cm for BC-408. However, the practical attenuation length of the prepared bars
is smaller than the nominal bulk value as the actual path length of photons from the charged particle
intersection point to the ends of the bar is increased and reflections occur due to the finite geometry
of the bar. For optimal response, this practical attenuation length should typically be longer than the
bar to ensure sufficient photon statistics. Measurements of the practical attenuation length of the
FTOF counters are given in Section~\ref{sec:attlen}.

\subsubsection{Photomultiplier Tubes and Voltage Dividers}

The panel-1a counters are read out at either end through 2-in diameter Thorn EMI 9954A PMTs (later
manufactured by ElectronTubes)~\cite{et-ref}. The PMTs are coupled to the scintillation bars using
12-cm-long acrylic light guides that match the 15~cm $\times$ 5~cm scintillator on one end and the
2-in diameter PMT on the other end.  For the panel-2 counters, 3-in diameter Philips XP4312B/D1 3-in
PMTs (later manufactured by Photonis~\cite{photonis}) are employed. The PMTs are coupled to the
scintillators through acrylic light guides that match the 22~cm $\times$ 5~cm scintillator on one end
and the 3-in diameter PMT on the other end. Both the 9954A and XP4312B/D1 PMTs have 12
linear-focused dynode stages. For both the panel-1a and panel-2 counters the PMTs are glued on the
light guides using BC-600 optical glue. See Ref.~\cite{tof-nim} for full details on the PMT selection
criteria and the light guide designs. The performance specifications for these PMTs are listed in
Table~\ref{pmt-specs}.

The voltage dividers employed for the panel-1a and panel-2 readout are custom units built specifically
for the CLAS TOF project~\cite{tof-nim}. The dividers use high voltage field-effect transistors to fix
the PMT gain by stabilizing the voltage and to protect the PMT against high light levels by shutting down
the circuit in case of over-current. The grid voltage for both types of dividers followed the
manufacturer's specifications.

The photomultipliers employed for the panel-1b counters are 2-in diameter Hamamatsu R9779 PMTs with
a high gain selection (minimum $0.5 \times 10^5$, average $1.0 \times 10^6$) that have been integrated
with a voltage divider to form the R9779-20MOD assembly. These PMTs include 8 linear-focused dynode
stages. This high time resolution PMT was selected due to its particularly compact overall assembly length
of 113~mm. The length restriction was necessary to fit within the defined shadow region of the torus
cryostats at the location of the PMTs (see Section~\ref{ftof-geometry} for details). These PMTs are
coupled directly to the ends of the scintillation bars using BC-600 optical glue. The performance
specifications for all FTOF PMTs are listed in Table~\ref{pmt-specs}.

%%%%%%%%%%%%%%%%%%%%%%%%%%%%%%%%%%%%%%%%%%%%%%%%%%%%%%%%%
\begin{table}[htbp]
\begin{center}
\begin{tabular}{c|c|c|c} \hline
                                                           & 9954A                    & R9779                     & XP4312B/D1 \\ \hline
Property                                            & Panel-1a                   & Panel-1b                  & Panel-2 \\ \hline
Diameter                                           & 2~in                         & 2~in                        & 3~in \\ \hline
Photocathode area                            & 16.6~cm$^2$           & 16.6~cm$^2$        & 36.3~cm$^2$ \\ \hline
Dynode stages                                  & 12                            & 8                              & 12 \\ \hline
Spectral response                            & 290 $\to$ 680~nm & 300 $\to$ 650~nm & 290 $\to$ 650~nm \\ \hline
Max. wavelength emission               & 400~nm & 420~nm & 420~nm \\ \hline
Gain                                                    & 1.8$\times$10$^7$ & 5.0$\times$10$^5$ & 3$\times$10$^7$ \\ \hline
Quantum eff. @ $\lambda_{max}$   & 28\% & 28\% & 28\% \\ \hline
Max. anode current rating               & 100~$\mu$A & 100~$\mu$A & 100~$\mu$A \\ \hline
Anode dark current                         & 2~nA & 10~nA & 10~nA \\ \hline
Anode pulse rise time                      & 2~ns & 1.8 ns & 2.1 ns \\ \hline
Electron transit time                        & 41 ns & 20 ns & 31 ns \\ \hline
Transition time spread                     & 0.4~ns & 0.25 ns & 0.4 ns \\ \hline
\end{tabular}
\end{center}
\caption{Properties of the PMTs used for the readout of the FTOF panel-1a, panel-1b, and panel-2
  counters. All of these PMTs have a borosilicate glass window and employ green-sensitive bialkali
  photocathodes.}
\label{pmt-specs}
\end{table}
%%%%%%%%%%%%%%%%%%%%%%%%%%%%%%%%%%%%%%%%%%%%%%%%%%%%%%%%%

\subsubsection{PMT Magnetic Shielding}

The FTOF PMTs are located in a range from 6.2~m to 7.2~m from the target in a region of where the
stray magnetic field from the torus is computed to be less than 30~G when the torus is operated at full
field. Custom magnetic shields for the PMTs are included to reduce both the axial and transverse
components of the field along the full accelerating structure of the PMT to a level of less than 0.2~G.
For the panel-1a and panel-2 counters, the PMT magnetic shields consist of a 0.020-in thick $\mu$-metal
cylinder, with the shield extending 2~in beyond the front face of the PMT.  For the panel-1b counters, the
magnetic shields are composed of 2-mm thick $\mu$-metal that extends 2~in beyond the front face of
the PMT. The shield has a rectangular box design that fits over the edge of the scintillation bar. The face
of the shield opposite the PMT side has a small penetration to allow the signal and HV cables and connectors
to pass through. Further details on the FTOF magnetic shielding and the field tests that were conducted
are included in Refs.~\cite{nim-p1b,ftof-shields}.

\subsubsection{Counter Assembly and Support}

Each scintillation counter is individually wrapped first with a reflective layer and then an opaque outer layer.
For panel-1a and panel-2 the scintillation counter wrapping materials include:

\begin{itemize}
\item 1 layer of 0.0094-in (0.02388~cm) thick black Kapton,
\item 2 layers of 0.001-in (0.00254~cm) thick aluminum foil.
\end{itemize}

\noindent
For panel-1b, the scintillation counter wrapping materials include:

\begin{itemize}
\item 3 layers of 0.0015-in (0.00381~cm) thick Tedlar,
\item 1 layer of 0.0003-in (0.00076~cm) thick aluminized polyester film.
\end{itemize}

After wrapping, each of the FTOF scintillation counters was attached to a support structure that consists
of a composite sandwich structure of stainless steel skins on structural foam that is attached to the
detector frame only at the two ends. The composite structure, which mounts on the scintillator side facing
away from the target, provides uniform material thickness to the detected particles.  The support was
undersized across the counter width so they could be placed as close together as allowed by the wrapping
material and the scintillation bar manufacturing tolerances. 

Each panel-1a counter is mounted on a 1-in thick supports to minimize the thickness of the package from
the standpoint of Coulomb multiple scattering and energy loss considerations.  The maximum deflection
for the installed scintillators is 4.4~mm, as estimated from deflection tests and the compound angle of
each detector, which relieves the overall support requirements. The space behind the panel-2 counters
allowed for 3-in thick sandwich supports, which are mechanically much stiffer and result in no appreciable
deflection. Again, each panel-2 counter is mounted to its own support structure. For the panel-1b counters,
the backing structures are 2-in thick and designed to support two panel-1b counters. The maximum
deflection for the installed scintillators is less than 5~mm, which occurs at the middle of the longest
counters.

The support structures onto which the scintillator counters are attached are bolted to box-beam support
frames (steel for panel-1a and panel-2, aluminum for panel-1b) that reside in the torus shadow regions.
The support frames are triangular in shape for panel-1a and panel-1b, and form a rhombus shape for
panel-2. The panel-1a frames are bolted directly to the upstream faces of the electromagnetic
calorimeters in each sector of the Forward Carriage. The panel-1b frames are bolted directly to the
panel-1a frames. The panel-2 frames are attached to the steel super-structure of the Forward Carriage.

\subsection{Electronics}
\label{sec-elec}

The outputs from the FTOF PMTs include both an anode and a dynode signal. The anode signals are
sent first to a discriminator and then to a TDC. The dynode signals are sent to a flash ADC. A block
diagram of the electronics layout for each FTOF counter is shown in Fig.~\ref{elec-block}.

For the FTOF PMTs the anode signal is roughly three times larger in amplitude than the dynode signal.
For the panel-1a and panel-2 PMTs, the dynode signals are bipolar with a negative polarity primary pulse
with a long tail that overshoots the baseline. This tail is not included in the determination of the pulse
charge. For the panel-1b PMTs, the anode signal has negative polarity and the dynode signal has positive
polarity. To ensure compatibility with the negative polarity input requirements of the FADC, the dynode
signal is inverted before the readout electronics using an inline Phillips Scientific 460 IT inverting
transformer.

The output from the panel-1b FADCs is also used as part of the CLAS12 level-1 trigger to select charged
hadrons. Signals in panel-1b above the FADC threshold are geometrically matched to hits in the
electromagnetic calorimeter and to found tracks in the drift chambers. Signals from the FTOF system
are also used to provide an effective charged particle veto for the detection of neutrals in the
electromagnetic calorimeters. While high resolution time measurements are the primary role of the
FTOF system for charged particle identification in the forward direction of CLAS12, the pulse height
information from the FADCs is also employed for energy loss measurements to provide an independent
means for identification of slow particles. In addition, pulse fitting techniques are employed using the
FADC pulse shape to determine the hit time of the track that can be compared to the TDC time to better
ensure matching of the ADC and TDC information in the high rate operating environment of CLAS12 (see
Section~\ref{cluster}).

%%%%%%%%%%%%%%%%%%%%%%%%%%%%%%%%%%%%%%%%%%%%%%%%%%%%%%%%%
\begin{figure}[htbp]
\vspace{5.5cm}
\begin{picture}(50,50) 
\put(35,-30)
{\hbox{\includegraphics[width=0.65\textwidth,natwidth=610,natheight=642]{pics/ftof-electronics-block.pdf}}}
\end{picture} 
\caption{Schematic of the electronics for each counter in the CLAS12 FTOF system.}
\label{elec-block}
\end{figure}
%%%%%%%%%%%%%%%%%%%%%%%%%%%%%%%%%%%%%%%%%%%%%%%%%%%%%%%%%

The intrinsic resolution of the electronics system ($\sigma_0$) must be optimized to ensure that it does
not become a limitation to the effective counter timing resolution. There are several contributions to this
term and each electronic component was studied to understand its effect.  From our measurements on the
bench and from CLAS12, a reasonable approximation for the floor term in the counter hit time resolution
is $\sigma_0=40$~ps (see Section~\ref{res-sec}). The PMT anode outputs are connected to JLab-designed
VME leading-edge discriminators. A leading-edge rather than a constant-fraction discriminator was chosen
for the FTOF system. Although a constant-fraction discriminator delivers better timing initially, off-line
time-walk corrections to leading-edge times give comparable results at a significantly lower cost since the
off-line analysis can use the measured charge. Time walk is an instrumental shift in the measured hit time
that arises due to the finite rise time of the analog pulse. For a given event time, pulses of different
amplitude cross the leading-edge discriminator threshold at slightly different time. The time-walk correction
algorithm is described in Section~\ref{sec-tw}. The discriminator threshold was set at -25~mV, significantly
above the 2~mV noise level. This threshold corresponds to 1~MeV of deposited energy. The discriminator
signal output width was set to 35~ns in order to prevent multiple outputs from the same input pulse.

The output of the discriminator goes to a CAEN VME TDC. Both high resolution TDCs (25~ps LSB CAEN
VX1290A) and lower resolution TDCs (100~ps LSB CAEN V1190A) are employed~\cite{tdc-manual}, where
the lower resolution TDCs are associated with the longer counters at large polar angles for panel-1a
($N_{counter}=17 \to 23$), panel-1b ($N_{counter}=49 \to 62$), and panel-2 ($N_{counter}=1 \to 5$). These
multi-hit pipeline TDCs were chosen in order to allow for readout capability in the operating luminosity of
$10^{35}$~cm$^{-2}$s$^{-1}$. The TDC readout window was set to 250~ns to ensure the full dynamic range
of the data was safely in time with the trigger. The key performance specifications of these TDC units are
given in Table~\ref{tdcadc-specs}.

The integral non-linearity (INL) of the TDCs represents the accumulated error of the input-output
characteristic of the TDC with respect to the ideal response and is defined by:

\begin{equation}
D(t) = \int \frac{t}{LSB},
\end{equation}

\noindent
where $D$ is the output data, $t$ is the input time, and $LSB$ (least significant bit) is the intrinsic bin
size. The compensation tables for the CAEN V1190A and VX1290A TDCs are stored as tables in the unit
SRAM memory. Initial tables are measured at the factory and come preloaded on the modules. These
tables are reasonably accurate when operating the module using its internal 40~MHz/25~ns period clock.
However, in CLAS12, the modules are strobed with an external clock of a slightly larger frequency of
41.67~MHz. This difference in the frequency has a non-negligible affect on the INL tables. For our
purposes we use a high frequency pulser to populate the full dynamic range of the TDC within the CLAS12
readout clock. The measured INL tables that were derived from this calibration were written into the
TDC memory to replace the factory-loaded values. Details on the procedure and the residual non-linearity
affects are given in Ref.~\cite{inl-tables}.

The PMT dynode outputs are connected to the FADCs for the pulse charge measurement. The readout
employs JLab-designed FADC250 16-channel VME 250-MHz flash ADCs~\cite{fadc-manual}. These units
can be operated in several readout modes. For standard data acquisition operation the FTOF counters are
readout in a mode where the pedestal is subtracted event-by-event. Figure~\ref{fadc-pulse} shows a raw
ADC pulse from a representative FTOF PMT where the pedestal has not been subtracted. Our procedure
determines the pedestal over the first 15 channels. This average is subtracted from our pulse signal region,
which lies between channels 35 and 65. A pulse fitting algorithm, which fits the leading edge of the pulse
down to the baseline, is used to determine the hit time from the FADC signal. The readout window for the
FTOF FADCs is set to 192 samples (48~ns). The applied readout threshold is set to 1~MeV to ensure that
the hit cluster energy can be determined with a reasonable accuracy. Details on the hit clustering for FTOF
are described in Section~\ref{cluster}. The key performance specifications of these FADC units are given
in Table~\ref{tdcadc-specs}.

The signal cables used for the FTOF system to connect from the PMT anodes and dynodes to the Forward
Carriage patch panels are RG-58C/U fire-retardant coaxial cables. This type of cable is appropriate for
moderate length cable runs for fast signals with low signal-distortion requirements. The cable runs vary
from 47~ft to 59~ft. The connections from the patch panels to the readout electronics are made with a
final 5~ft run of low loss RG-174 coaxial cable. The inline signal inverting transformers for the panel-1b
dynodes (see Section~\ref{sec-elec}) were attached directly to the Forward Carriage patch panels.

%%%%%%%%%%%%%%%%%%%%%%%%%%%%%%%%%%%%%%%%%%%%%%%%%%%%%%%%
\begin{table}[htbp]
\begin{center}
\begin{tabular}{c|c} \hline
TDC Specs (V1190A/VX1290A) & ADC Specs \\ \hline
No. Channels: 128/32             & 16               \\ \hline
RMS resolution 100~ps/25~ps          & Sampling 250 MHz \\ \hline 
Resolution: 19~bit/21~bit                  & Resolution: 12-bit \\ \hline
Inter-channel isolation $\le$ 3 LSB & Clock jitter 350~fs \\ \hline
Double-hit resolution 5~ns          & Data memory 8~$\mu$s \\ \hline    
Full-scale range 52~$\mu$s          & Trigger/Data latency 8~$\mu$s / 32~ns \\ \hline  
\multicolumn{2}{c}{Integral/Differential non-linearity} \\
$<$ 2.5 LSB / $<$ 3 LSB             & $\pm$0.5 LSB / $\pm$0.8 LSB \\ \hline
Inter-channel isolation $<$ 3 LSB   & SNR 56.8~dB @ 100~MHz input \\ \hline
\end{tabular}
\end{center}
\caption{Key performance specifications of the FTOF CAEN V1190A and VX1290A pipeline
TDCs and the JLab FADC250 flash ADCs.}
\label{tdcadc-specs}
\end{table}
%%%%%%%%%%%%%%%%%%%%%%%%%%%%%%%%%%%%%%%%%%%%%%%%%%%%%%%%%

%%%%%%%%%%%%%%%%%%%%%%%%%%%%%%%%%%%%%%%%%%%%%%%%%%%%%%%%%
\begin{figure}[htbp]
\vspace{5.0cm}
\begin{picture}(50,50) 
\put(5,-45)
{\hbox{\includegraphics[width=0.65\textwidth,natwidth=610,natheight=642]{pics/fadc-pulse.pdf}}}
\end{picture} 
\caption{Typical FADC pulse for a representative FTOF counter from the JLab FADC250. The ``PED''
region is used to determine the average pedestal in the ``SIG'' region shown about the PMT pulse. This
plot shows signal amplitude vs. FADC channel.}
\label{fadc-pulse}
\end{figure}
%%%%%%%%%%%%%%%%%%%%%%%%%%%%%%%%%%%%%%%%%%%%%%%%%%%%%%%%%

\subsubsection{High-Voltage Supplies}

The PMTs for the FTOF counters typically operate at about 2000~V with negative polarity. The typical
dark current drawn by the PMTs is $<20$~nA. The system is powered by a single high voltage mainframe
for each sector. These mainframes are either CAEN 1527LC or CAEN 4527 units outfitted with negative
polarity 24-channel A1535N modules that can supply up to 3.5~kV per channel with a maximum current of
3~mA. The power supply has a voltage ripple specification of $<$20~mV peak-to-peak. Each channel
consumes less than 1~W during counter operation with typical supply currents per channel between
300~$\mu$A to 500~$\mu$A.

The mainframe is controlled remotely through the Hall~B Slow Controls system. A graphical user interface
using EPICS~\cite{epics} running on a UNIX system communicates with the mainframe via Ethernet. The
mainframe settings enable basic protection of the PMTs in terms of maximum voltage and current settings,
and channel ramp rates.

The high voltage cables for each PMT are fire-retardant RG-59 coaxial cables that run from the PMT
voltage divider to a local disconnect HV distribution box located behind the panel-2 arrays in each
sector. There are four 48-channel HV distribution boxes for each sector, two for the left PMTs and two
for the right PMTs. The output of each HV distribution box is a pair of 35-ft long multi-conductor cables,
each containing 24-channels, with a Radiall connector to mate with the HV A1535N board input connector.
Each multi-conductor high voltage cable contains individual conductors wrapped in Tefzel insulation, an outer
wire shield, and a PVC insulation wrap. Each conductor is rated at 5~kV.

\section{FTOF Performance}
\label{sec:performance}

This section highlights the performance of the FTOF system both on the test bench and in Hall~B
during the first beam runs for CLAS12. The bench test timing performance is important to
understand to ensure that both the refurbished counters that make up the panel-1a and panel-2
array from the CLAS TOF system still meet their original performance specifications as detailed
in Table~\ref{spec-table} and Ref.~\cite{tof-nim}. These bench performance studies are even more
important for the newly constructed panel-1b FTOF arrays for CLAS12 as they are primarily
responsible for the limits of the particle identification separation for CLAS12 in the forward direction.
Full details on the bench test performance results for the panel-1a and panel-2 counters are provided
in Ref.~\cite{dsc-cn2013-001} and for the panel-1b counters in Ref.~\cite{nim-p1b}.

In this section the essential performance results from the bench testing studies are presented in terms
of the counter photoelectron statistics and benchmark time calibrations. Then the calibration algorithms
are introduced to provide details on how the in-beam FTOF time resolution performance was quantified.
Finally, this section provides the current status of the particle identification capabilities of the FTOF
system in relation to the design specifications.

\subsection{Bench Measurements}

\subsubsection{Counter Photoelectron Statistics}
\label{sec:npe}

The primary approach to determine the average number of photoelectrons at the photocathode of a PMT
generated by minimum-ionizing particles passing through the scintillation bars~(see Ref.~\cite{Gi86} for
further details) employs the ratio of the integral of the signal pulse to the integral of the pulse for a single
photoelectron. For these measurements we used the pulse integration feature of an Agilent Technologies
MSO-X 3034A 350~MHz (4 GSa/s) oscilloscope and averaged over 1000 pulses. The minimum-ionizing
particle signals were analyzed by connecting the scope to a PMT mounted on one of the shortest FTOF
panel-1a counters. For the single photoelectron signal, we took data using just a bare PMT on the bench
using the same HV setting. For both measurements the oscilloscope threshold was adjusted appropriately.
For the minimum-ionizing peak analysis the threshold had to be set high enough ($>$200~mV) to eliminate
tracks that did not pass through the full thickness of the bar. For the single photoelectron peak the
threshold had to be set low enough (1~mV) to pick out the single-electron emission noise pulses from the 
photocathode that are the dominant source of the PMT intrinsic dark current. This measurement scheme
yielded $N_{pe}$=1000$\pm$100, a value consistent with the one found during the initial characterization
of these counters for the CLAS TOF system nearly 25 years ago~\cite{tof-nim}. Based on the
parameterization given in Eq.(\ref{nphe-eq}), Fig.~\ref{nphe-plot} shows the number of photoelectrons at
the PMT photocathodes for the different FTOF counters.

%%%%%%%%%%%%%%%%%%%%%%%%%%%%%%%%%%%%%%%%%%%%%%%%%%%%%%%%%
\begin{figure}[htbp]
\vspace{3.8cm}
\begin{picture}(50,50) 
\put(33,-60)
{\hbox{\includegraphics[width=1.0\textwidth,natwidth=610,natheight=642]{pics/nphe.pdf}}}
\end{picture} 
\caption{Parameterized distribution of the number of photoelectrons vs. counter length (cm) for the
panel-1a (dotted), panel-1b (dashed), and panel-2 (dot-dashed) counters based on direct measurements
with the shortest panel-1a counter.}
\label{nphe-plot}
\end{figure}
%%%%%%%%%%%%%%%%%%%%%%%%%%%%%%%%%%%%%%%%%%%%%%%%%%%%%%%%%

A second method to estimate the number of photoelectrons produced at the PMT photocathode,
which accounts for the quantum efficiency at the photocathode, can be estimated from cosmic
ray data using the form of Ref.~\cite{kajino}:

\begin{equation}
\label{nphe}
\langle N_{pe} \rangle = \left( \frac{M_{ADC}}{\sigma_{ADC}} \right)^2,
\end{equation}

\noindent
where $M_{ADC}$ is the ADC mean for the minimum-ionizing peak in the ADC spectrum and $\sigma_{ADC}$
is the width of the ADC distribution. The form of Eq.(\ref{nphe}) assumes that a finite $\sigma_{ADC}$
arises solely due to statistical variations in the number of photoelectrons created at the photocathode for
an event sample with a fixed energy loss per track, which we can assume to be a good approximation for
perpendicularly incident minimum-ionizing tracks. From the measured data averaged across the counters in
panel-1a and panel-1b it was found that $N_{pe}^{1a} = 373 \pm 39$ and $N_{pe}^{1b} = 1158 \pm 77$
\cite{pmt-currents}. These results, while roughly a factor of two below the parameterized estimates, also
show the same factor of three difference in the expected number of photoelectrons for panel-1b relative
to panel-1a. The estimates from the first approach are considered to be more reliable not only because they
are connected to a more direct measurement of the number of photoelectrons, but also because this
parameterization for $N_{pe}$ used in Eq.(\ref{timing-func}) agrees reasonably well with the measured
counter resolutions shown in Section~\ref{tres-beam}.

\subsubsection{Bench Time Resolution Performance}
\label{sec-bench}

The basic algorithm used on the test bench for the panel-1a and panel-2 counters to determine the time
resolution of a given reference counter was to use incident cosmic ray tracks to compare the measured
time for a reference counter to the time measured by two other identical counters in a triplet counter
configuration (see Fig.~\ref{triplet}). For a triplet measurement, where the track passes through all
three counters with double-sided readout, six times are measured ($t_1 \to t_6$). Each time measurement
actually represents the difference between the discriminated PMT signal (TDC start) and the trigger
time (TDC stop) generated from the six-fold PMT coincidence. These time measurements are then
translated into three counter hit times $t_{t,m,b} = \frac{1}{2}(t_{1,3,5} + t_{2,4,6})$.

%%%%%%%%%%%%%%%%%%%%%%%%%%%%%%%%%%%%%%%%%%%%%%%%%%%%%%%%%
\begin{figure}[htbp]
\vspace{4.2cm}
\begin{picture}(50,50) 
\put(75,5)
{\hbox{\includegraphics[width=0.80\textwidth,natwidth=610,natheight=642]{pics/triplet-alt.pdf}}}
\end{picture} 
\caption{Schematic representation of a triplet of counters (labeled top - $t$, middle - $m$, and bottom
- $b$) with a cosmic ray track traversing the stack. The geometry of the triplet was configured such that
the separation between the top and middle counters matches that of the middle and bottom counters.}
\label{triplet}
\end{figure}
%%%%%%%%%%%%%%%%%%%%%%%%%%%%%%%%%%%%%%%%%%%%%%%%%%%%%%%%%

For incident tracks that pass fully through each counter of the triplet with measured times $t_t$, $t_m$,
and $t_b$, we can define a time residual $t_r = t_m - \frac{1}{2}(t_t + t_b)$, where the time $t_m$ of the
middle scintillator hit should be the average of the measured times $t_t$ and $t_b$ for the top and bottom
scintillator hits, respectively. Thus the measured residual $t_r$ should nominally be zero. However, due to
the smearing of the measured times $t_t$, $t_m$, and $t_b$ due to the finite time resolution of the
measurements, the residual time $t_r$ will also be smeared. While we still expect the mean of $t_r$ to 
be zero, the width of the $t_r$ distribution can be used to determine the average time resolution of the
counters in the triplet.

The average time resolution of each of the identical counters was computed from the variance $\delta t_r$
in the measured time residual $t_r$. Assuming the average time resolution for each PMT in the triplet
($\Delta t_i$, $i = 1 \to 6$) is identical and taking into account that each counter is read out using two
PMTs, we can write the final expression for the average counter time resolution as:

\begin{equation}
\label{sig-counter}
\sigma_{counter} = \frac{2}{\sqrt{6}} \delta t_r.
\end{equation}

\noindent
Thus a measure of the variance of the time residual distribution provides a measure of the average
resolution of a counter in the triplet. 

Figure~\ref{final-resolution} shows the average time resolution measured in the triplet configurations
for the panel-1a and panel-2 FTOF counters. For these measurements the fully assembled counter arrays
were stacked one above the other in the cosmic ray test stand. The triplets $t$, $m$, and $b$ were
formed from the counters for Sectors~1, 6, and 5 and separately for the counters for Sectors~2, 4, and
3. This analysis included a minimum PMT ADC cut to remove events that did not pass through the full
thickness of the counter (``corner-clippers'') and also included a coordinate cut of $\pm$10~cm about
the center of the scintillation bar. Due to the use of leading edge discriminators, the measured PMT times
were corrected with a time-walk function of the form:

\begin{equation}
\label{walk-function}
t_{walk}^{L,R} = \frac{A_0}{1 + A_1 \sqrt{(ADC - PED)_{L,R}}}.
\end{equation}

\noindent
Here, $ADC - PED$ is the pedestal-subtracted ADC value for each PMT. The time-walk functional in
these measurements best describe the data using $A_0 = 50.0$ and $A_1 = 0.852$.

The average time resolutions for the counters in the panel-1a S1-S6-S5 and S2-S4-S3 triplets were
found to be slightly better ($\sim$15\%) than those achieved for the baseline measurements over
20~years ago. For panel-2, the average time resolution was found to be slightly worse ($\sim$15\%)
than these same baseline measurements.

%%%%%%%%%%%%%%%%%%%%%%%%%%%%%%%%%%%%%%%%%%%%%%%%%%%%%%%%%%%%%%
\begin{figure}[htbp]
\vspace{3.0cm}
\begin{picture}(50,50) 
\put(0,-60)
{\hbox{\includegraphics[width=0.60\textwidth,natwidth=610,natheight=642]{pics/p1a-tres.pdf}}}
\end{picture} 
\begin{picture}(50,50) 
\put(135,-60)
{\hbox{\includegraphics[width=0.60\textwidth,natwidth=610,natheight=642]{pics/p2-tres.pdf}}}
\end{picture} 
\caption{(Color Online) Average bench measurement resolutions (ps) vs. counter length (cm) using cosmic rays
for the refurbished FTOF panel-1a (left) and panel-2 (right) counters. The different sets of data points
correspond to the two different cosmic ray test stands used for calibration. The data points for the counters
in sectors 1, 5, and 6 were averaged together, as well as those for the counters in sectors 2, 3, and 4.}
\label{final-resolution}
\end{figure}
%%%%%%%%%%%%%%%%%%%%%%%%%%%%%%%%%%%%%%%%%%%%%%%%%%%%%%%%%%%%%%%%%

The bench measurements for the panel-1b FTOF counters were carried out using a stack of six
equidistant counters of a given counter number from $N_{counter} = 1 \to 62$. Accounting for the
necessary path length corrections to relate the individual counter times to each other, six simultaneous
triplet counter measurements were analyzed and the resulting time-residual system of equations was then
solved for the individual counter time resolution. Also in distinction to the simple time-walk correction
employed for the panel-1a and panel-2 measurements shown in Eq.(\ref{walk-function}), a more sophisticated
position-dependent time-walk correction was employed that generalizes the simpler position-independent form.
Precision measurements of the time-walk amplitude ($A_0$ in Eq.(\ref{walk-function})) vs. the distance from
the PMT showed a nearly linear fall-off of the amplitude with increasing distance from the PMT. On average
the time-walk amplitude is $\sim$30\% larger at the PMT compared to the far end of the bar, although this
near to far end ratio of the amplitude decreases linearly with the length of the bar. Our measurements
showed this ratio varies between 20\% for the shortest bars to 40\% for the longest bars. After accounting
for this correction vs. hit position along the bar, a final baseline for the average panel-1b time resolutions was
extracted averaging over the six counters of a given number from $N_{counter} = 1 \to 62$. The resulting time
resolutions for the panel-1b counters are shown in Fig.~\ref{p1b-tres} and range from 30~ps for the shortest
counters  (17~cm long) to 80~ps for the longest counters (408~cm long). Full details describing the
measurements are provided in Ref.~\cite{nim-p1b}.

%%%%%%%%%%%%%%%%%%%%%%%%%%%%%%%%%%%%%%%%%%%%%%%%%%%%%%%%%%%%%%%%
\begin{figure}[htbp]
\vspace{4.2cm}
\begin{picture}(50,50) 
\put(50,-85)
{\hbox{\includegraphics[width=0.8\textwidth,natwidth=610,natheight=642]{pics/p1b-tres.pdf}}}
\end{picture} 
\caption{Measurements of the time resolution (ps) vs. counter length (cm) achieved for the FTOF panel-1b
system averaged over the six counters of a given length belonging to each CLAS12 Forward Carriage sector.
These data were acquired on the bench using cosmic rays. Full details are included in Ref.~\cite{nim-p1b}.}
\label{p1b-tres}
\end{figure}
%%%%%%%%%%%%%%%%%%%%%%%%%%%%%%%%%%%%%%%%%%%%%%%%%%%%%%%%%%%%%%%%

For our purposes in quoting counter time resolution values, it is essential to distinguish between two
different quantities. The first is the {\em intrinsic} counter time resolution that reflects the resolution
parameterized in Eq.(\ref{timing-func}). This includes the resolution contributions that mainly depend on
the photon statistics at the PMT photocathode and hence the counter geometry, surface quality, scintillation
material and bulk quality, wrapping preparations, etc., the transit time spread of the PMT, and the readout
electronics noise (the floor term of the resolution). However, when calibrating the time resolution in situ in
Hall B with beam interactions in the experimental target, an {\em effective} time resolution is extracted that
includes not only the intrinsic resolution contributions, but also contributions from the angle-dependent
uncertainty in the path length determined by the CLAS12 forward tracking system and the resolution spread
in the accelerator RF signal that is used as a comparison reference time. The results quoted in this section
represent the intrinsic time resolutions of the counters. The effective in situ time resolutions are discussed
in detail and presented in Section~\ref{tres-beam}.

\subsection{FTOF Beam-Data Calibrations}
\label{beam-data-calib}

In the nominal data taking mode for CLAS12, whenever the FTOF is involved in an event that triggers the
spectrometer readout, the ADCs and TDCs for all PMTs with a signal above the discriminator threshold are
recorded. For the FADCs, the charge of the pulse is integrated over the extent of the pulse region and the
pedestal is subtracted event by event as discussed in Section~\ref{sec-elec}. For the TDCs the time
recorded is relative to the trigger. To determine the flight time of the charged track from the target to
the FTOF, the TDC time must be correlated with the time of the electron beam bunch initiating the trigger
that is defined by the accelerator radio frequency (RF) pulse. The RF signal from the accelerator has a period
of 2.004~ns. The RF bunch length itself corresponds to a few picoseconds. Although the signal timing is very
accurate (with a resolution of $<$20~ps), the determination of which beam bunch produced a given interaction
must be determined by the experiment.

The full calibration of each of the FTOF counters involves a number of discrete steps that are carried out
separately for a given data run (where a run typically lasts for about two hours). The calibration constants
for each run are stored in the CLAS12 calibration database. The calibrations are redone whenever there is
a response shift outside of our allowed timing or energy tolerances. The steps to the FTOF calibration are
carried out in a particular sequence~\cite{ftof-calib}.

\begin{enumerate}
\item Left/right PMT time offsets: This time offset accounts for the difference in the time recorded
between the left side and right side PMTs in a given counter due mainly to the different PMT transit times.
These time offsets are determined from the centroid of the difference between the left/right TDC time
difference and the left and right TDC time computed using the counter hit point from the drift chamber
system divided by the effective speed of light in the counter. These time shifts range between $\pm$5~ns.

\item ADC Calibration: Determine the ADC channel to energy deposition calibration factor for each counter
using minimum-ionizing events, see Section~\ref{gain-matching}.

\item Attenuation Length Calibration: This property of the counter quantifies the light absorption length and
is determined by relating the measured ADC as a function of hit coordinate along the bar, see
Section~\ref{sec:attlen}.

\item Effective Velocity Calibration: Determine the effective speed of light propagation along the counter,
see Section~\ref{sec:veff}.

\item Time-Walk Amplitude Calibration: Compare the measured hit time with respect to the measured ADC
to determine the time-walk correction, see Section~\ref{sec-tw}.
  
\item Counter-to-Counter Time Offset Calibration: In order to measure the absolute flight time of a charged
particle from the target to the FTOF counter and to be able to reconstruct exclusive events when hits are
associated with multiple FTOF counters, the relative time offsets of each counter relative to all of the other
counters in the system need to be determined. This is done in two steps. The first step is to correlate each
counter hit time to the RF time, which amounts to a precision time alignment in bins of the TDC LSB. The
second step is a coarse alignment of each counter hit time in bins of the RF period $T_{RF}$. See
Section~\ref{sec-talign}. During this step the effective counter time resolutions are extracted, see
Section~\ref{tres-beam}.

\item TDC Calibration: After calibrating the integral non-linearities of each TDC channel in the system (see
Section~\ref{sec-elec}), the TDC channel to time calibration is completed using beam events, see
Section~\ref{sec-tdccal}.

\end{enumerate}

To calibrate the panel-1a and panel-1b counters an event filter is chosen to select charged leptons and pions.
For the panel-2 calibration it is necessary to select charged pions and protons. Due to the increased multiple
energy loss and Coulomb multiple scattering of the proton sample, the effective counter time resolutions
derived for the panel-2 counters are noticeably worse than for the bench test results.

\subsubsection{PMT Gain Matching}
\label{gain-matching}

One of the purposes of gain matching the FTOF PMTs is to equalize the detector response to tracks that
cross the FTOF arrays such that two counters are involved. This is a necessary procedure because each
counter must contribute equally to the trigger for a common-threshold discriminator level. Gain matching,
so that the minimum-ionizing particle response is such that the peak location appears in the same ADC
channel for all counters, also allows for easier data monitoring during online and offline analyses.

The FTOF PMT high voltage settings were determined using calibration runs employing minimum-ionizing
tracks. These minimum-ionizing tracks deposit roughly 10~MeV (12~MeV) as they pass through the 5-cm
(6-cm) thick FTOF scintillation bars, as $dE/\rho dx = 2$~MeV/g/cm$^2$ for minimum-ionizing particles.
The high voltage settings were initially based on runs using cosmic rays with the readout based on a
calorimeter pixel trigger that effectively select tracks approximately perpendicular to the face of the
FTOF counters in panel-1a and panel-1b. Currently the calibrations are carried out using minimum-ionizing
tracks from beam data coming from the target. In this case the ADC charge information is scaled by a
scintillation bar path length correction given by $t/P$, where $t$ is the counter thickness and $P$ is the
path-length of the track in the counter as determined by extrapolation of the drift chamber track to the
location of the FTOF counter. The energy deposited in the scintillation bars is recorded by the ADCs, which
show Landau peaks above the pedestal. Minimum-ionizing tracks that do not pass through the full counter
thickness and more heavily ionizing tracks give rise to a background beneath the Landau peak.

For the HV calibrations, to avoid issues with the attenuation of light for tracks that pass near the ends of
the bars and with unbalanced light entering the left and right PMTs, we combine the information from the
left and right PMTs to produce an average ADC spectrum for the counter through the quantity known
as the geometric ADC mean:

\begin{equation}
\label{adc}
\overline{ADC} = \sqrt{ (ADC - PED)_L \cdot (ADC - PED)_R}.
\end{equation}

Given the finite dynamic range of the ADC, we have chosen to position the minimum-ionizing peak in a
particular ADC channel that is different for the panel-1a, panel-1b, and panel-2 counters. For all counters
this channel is selected so that it is safely above the pedestal, but leaves sufficient range for the more
highly ionizing charged tracks of our typical physics events. To minimize PMT aging effects that result in
loss of PMT gain with time correlated with the total charge collected at the anode of the PMT, the gains
are set as low as possible.

The position of the minimum-ionizing peak in the ADC spectrum is set by the PMT HV values. For a given
scintillation bar, a typical $\overline{ADC}$ spectrum is shown in Fig.~\ref{gmean}. Given that the same
channel is chosen to position the minimum-ionizing peak for all counters in either panel-1a or panel-1b, the
required PMT gain increases linearly from the short to the long bars to compensate for the attenuation
losses in the longer bars.

%%%%%%%%%%%%%%%%%%%%%%%%%%%%%%%%%%%%%%%%%%%%%%%%%%%%%%%%%
\begin{figure}[htbp]
\vspace{4.5cm}
\begin{picture}(30,50) 
\put(95,-50)
{\hbox{\includegraphics[width=0.53\textwidth,natwidth=610,natheight=642]{pics/gmean.pdf}}}
\end{picture} 
\caption{(Color Online) Geometric ADC mean spectrum for one representative FTOF counter from beam
data. The recorded events are pedestal subtracted. The red curve is a fit function that includes a Landau
shape for the peak and an exponential for the background.}
\label{gmean}
\end{figure}
%%%%%%%%%%%%%%%%%%%%%%%%%%%%%%%%%%%%%%%%%%%%%%%%%%%%%%%%%

The PMT gains depend exponentially on the applied voltage. Expressed in a slightly different way, we 
can relate the PMT gain $G_1$ at a given voltage $V_1$ to the gain $G_2$ at a different voltage $V_2$ 
via:

\begin{equation}
\label{power-law}
\frac{G_1}{G_2} = \left( \frac{V_1}{V_2} \right) ^\alpha.
\end{equation}

\noindent
This is a basic power law form with $\alpha$ representing the power law factor. Eq.(\ref{power-law}) can
be rewritten in a slightly different form as:

\begin{equation}
\label{delta}
\frac{\Delta G}{G} = \alpha \frac{\Delta V}{V}.
\end{equation}

It is this expression that is the basis for relating the position of the minimum-ionizing peak in the 
$\overline{ADC}$ spectrum (see Eq.(\ref{adc})) to the PMT HV setting. The gain-matching procedure
then amounts to adjusting the HV settings of all PMTs to the values required to position the
minimum-ionizing peak for each counter in the desired ADC location. At the same time the algorithm
uses the individual left and right PMT ADC spectra for a given counter to ensure that these PMT
gains are balanced.

The power law factor $\alpha$ in Eq.(\ref{power-law}) for each PMT type can be determined by looking
at data with two different high voltage settings. In this manner the average $\alpha$ factors for the
FTOF PMTs were determined to be 13.4 for panel-1a, 4.7 for panel-1b, and 8.6 for panel-2. With these
values the calibrations converge within just a few iterations such that all of the minimum-ionizing particle
peak locations are within $\pm$25 ADC channels of their set targets.

The energy loss in a counter for a passing charged particle track is determined after the
minimum-ionizing peak centroids are aligned. The energy loss in each counter is computed from each
PMT as:

\begin{equation}
E_{L,R} = ADC_{L,R} \cdot \left [ \frac{\left( \frac{dE}{dx} \right)_{MIP} \cdot t}{ADC_{MIP}}\right ]
\exp\left(\frac{d_{L,R}}{\lambda}\right),
\end{equation}

\noindent
where $ADC_{MIP}$ is the centroid of the minimum-ionizing peak in the geometric mean distribution,
$\left( \frac{dE}{dx} \right)_{MIP}$ is the energy loss for minimum-ionizing particles in the scintillation
bars (1.956~MeV/cm), $t$ is the counter thickness ($t$=5~cm for panel-1a and panel-2, and $t$=6~cm
for panel-1b), $d$ is the distance along the bar from the track hit position to the PMT and $\lambda$ is
the counter attenuation length. The energy loss used in the event reconstruction is the geometric mean of
the separate measures $E_{L,R}$.

Figure~\ref{ftof-dedx} shows the reconstructed energy loss normalized by the track path length through
the bar for different panels from a data run with a 10.6-GeV electron beam incident upon a liquid-hydrogen
target. The data allow the separation of minimum-ionizing particles from more heavily ionizing particles.
The minimum-ionizing particles lose a constant amount of energy as a function of path length. At low-momentum
the more heavily ionizing particles have energy loss that increases linearly with distance until they can pass
through the counter. At that point their energy loss follows the Bethe-Bloch formula.

%%%%%%%%%%%%%%%%%%%%%%%%%%%%%%%%%%%%%%%%%%%%%%%%%%%%%%%%%
\begin{figure}[htbp]
\vspace{5.0cm}
\begin{picture}(50,50) 
\put(-2,270)
{\hbox{\includegraphics[width=0.9\textwidth,,height=0.50\textheight,natwidth=610,natheight=642,angle=-90]
{pics/ftof-dedx.pdf}}}
\end{picture} 
\caption{(Color Online) Measured FTOF counter energy loss for positively charged particles from
10.6-GeV electrons incident upon a liquid-hydrogen target normalized by the extrapolated path length
from the projection of the forward track through the counter array. The normalized $dE/dx$ (MeV/cm)
spectrum shows the separation of minimum-ionizing particles from more heavily ionizing particles summed
over the counters in panel-1a (left), panel-1b (middle), and panel-2 (left). The top row of plots show the
normalized $dE/dx$ and the bottom row of plots show $dE/dx$ vs. track momentum (GeV).}
\label{ftof-dedx}
\end{figure}
%%%%%%%%%%%%%%%%%%%%%%%%%%%%%%%%%%%%%%%%%%%%%%%%%%%%%%%%

\subsubsection{Attenuation Length Measurements}
\label{sec:attlen}

The attenuation length of the scintillation bars represents the distance $\lambda$ into the material
where the probability that the photon has been absorbed is $1/e$. The measured ADC values for each
PMT can be written in terms of the attenuation length as:

\begin{equation}
\label{al-adc}
ADC = A_0 e^{-d/\lambda},
\end{equation}

\noindent
where $A_0$ is a constant, $d$ is the distance along the counter with respect to the PMT location, and
$\lambda$ is the counter attenuation length. Forming the ratio of the ADC values from the right over
the left ends of the counter,

\begin{equation}
\label{linear}
\log \left( \frac{(ADC-PED)_R}{(ADC-PED)_L} \right ) = C + \frac{2d}{\lambda},
\end{equation}

\noindent
a linear fit of the log ADC ratio vs. coordinate can be used to extract the effective counter attenuation
length. 

Figure~\ref{atten-len} shows the measured attenuation lengths for the FTOF counters in one sector of
the CLAS12 Forward Detector extracted from data with a 10.6~GeV electron beam incident upon a
liquid-hydrogen target.

%%%%%%%%%%%%%%%%%%%%%%%%%%%%%%%%%%%%%%%%%%%%%%%%%%%%%%%%%
\begin{figure}[htbp]
\vspace{3.7cm}
\begin{picture}(30,50) 
\put(-5,-70)
{\hbox{\includegraphics[width=1.2\textwidth,natwidth=610,natheight=642]{pics/atten-r4013.pdf}}}
\end{picture} 
\caption{Counter attenuation lengths (cm) vs. counter number for all FTOF counters in one sector of the
CLAS12 Forward Detector determined from beam data.}
\label{atten-len}
\end{figure}
%%%%%%%%%%%%%%%%%%%%%%%%%%%%%%%%%%%%%%%%%%%%%%%%%%%%%%%%%

\subsubsection{Effective Velocity Determination}
\label{sec:veff}

The effective velocity of light in each counter employs a calculation based on the comparison of the
reconstructed coordinate information along the scintillation bar from the time information with
the track hit coordinate determined from the extrapolation of the track beyond the drift chambers
to the location of the FTOF counters. Figure~\ref{veff} shows the measured effective velocity for
each counter in one sector of the CLAS12 Forward Detector using data with a 10.6-GeV electron
beam incident on a liquid-hydrogen target. 

As the counter length increases, so does the effective velocity because light rays at large angles with
longer actual trajectories to the PMT are systematically lost owing to attenuation. These quantities are
used in FTOF analysis to determine the hit time for each event from the measured TDC times. The
intrinsic position resolution is given by $v_{eff} \times \sigma(t_L - t_R)$ for each counter, which is
most relevant for the interactions of neutral particles. The position for charged particles can be
measured more precisely with the drift chambers.

%%%%%%%%%%%%%%%%%%%%%%%%%%%%%%%%%%%%%%%%%%%%%%%%%%%%%%%%%
\begin{figure}[htbp]
\vspace{3.7cm}
\begin{picture}(30,50) 
\put(-10,-70)
{\hbox{\includegraphics[width=1.2\textwidth,natwidth=610,natheight=642]{pics/veff-r4013.pdf}}}
\end{picture} 
\caption{Counter effective velocities (cm/ns) vs. counter number for all FTOF counters in one sector of
the CLAS12 Forward Carriage determined from beam data.}
\label{veff}
\end{figure}
%%%%%%%%%%%%%%%%%%%%%%%%%%%%%%%%%%%%%%%%%%%%%%%%%%%%%%%%%

\subsubsection{Time-Walk Corrections}
\label{sec-tw}

The approach that we adopted to correct the FTOF TDC times for time-walk effects is different from
the one employed for our bench test studies of the counters in their cosmic ray test stands described in
Section~\ref{sec-bench}. We ultimately settled on an approach that first accounts for the average hit
position independent correction with a time-walk functional of the form:

\begin{equation}
\label{tw-func}
t^{corr}_{L,R} = t_{L,R} - \frac{tw0_{L,R}}{\sqrt{(ADC - PED)_{L,R}}}.
\end{equation}

The time-walk parameters $tw0_{L,R}$ are determined by defining the following vertex time residuals:

\begin{equation}
\label{tres}
t_{L,R}^{res} = \left(t_{L,R} - \frac{d_{L,R}^{DC}}{v_{eff}} - \frac{L}{\beta c} \right) 
- \left( t_{RF} + \frac{z_{vert}}{\beta_e c} \right),
\end{equation}

\noindent
for each PMT, where $t_{L,R}$ are the measured TDC times after the left/right PMT time offset
correction, $d^{DC}/v_{eff}$ corrects the time measured at the PMT to the time at the track hit
point on the counter determined from DC tracking information, and $L/(\beta c)$ is the track
flight time from the reaction vertex to the FTOF. The track path length $L$ is defined via DC
tracking and $\beta$ is determined by the DC momentum and the particle identification determined
from the Event Builder. The term $z_{vert}/(\beta_e c)$ corrects the RF time $t_{RF}$ for the actual
electron beam event vertex location along the $z$-axis of the extended target. In this expression,
$d_{L,R}^{DC}$ are the distances along the bar from the track hit point to the left and right PMTs.
This vertex time residual represents the FTOF hit time from a single PMT traced back to the reaction
vertex and compared to a precise time reference $t_{RF}$ given by the RF signal from the accelerator.
As $t_{RF}$ represents a reference time for the arrival of the electron beam bunch at a fixed
position along the beamline in Hall~B assigned as the center of the target, the time must be corrected
for the displacement of the reaction vertex along the extended length of the target.

As the beam bucket that was associated with the event is not determined at this point, the 
time walk for each PMT is actually determined using the modulus of $t^{res}_{L,R}$ with the RF
beam bucket period $T_{RF}=$1/(RF frequency) by fitting:

\begin{equation}
t'_{L,R} = mod (t^{res}_{L,R}, T_{RF})~{\rm vs.}~(ADC-PED)_{L,R}.
\end{equation}

Figure~\ref{twalk-plot} shows the $t'$ vs. $ADC$ distribution for a representative FTOF PMT in
panel-1b from beam data using a 10.6-GeV electron beam incident upon a 5-cm long liquid-hydrogen
target. Note that all distributions that employ $t'_{L,R}$ are sorted in 25~ps bins (consistent with
the TDC LSB). The overall scale of the time-walk effects spanning the full dynamic range of the
ADC is 2~ns.

%%%%%%%%%%%%%%%%%%%%%%%%%%%%%%%%%%%%%%%%%%%%%%%%%%%%%%%%%%
\begin{figure}[htbp]
\vspace{4.0cm}
\begin{picture}(50,50) 
\put(55,180)
{\hbox{\includegraphics[width=0.52\textwidth,natwidth=610,natheight=642,angle=-90]{pics/twalk-plot.pdf}}}
\end{picture} 
\caption{(Color Online) Plot of $t'$ (ns) vs. $ADC$ for one representative PMT from panel-1b from beam
data. As $t'$ is defined using the modulus of $T_{RF}$, its limits span $\pm T_{RF}/2$.}
\label{twalk-plot}
\end{figure}
%%%%%%%%%%%%%%%%%%%%%%%%%%%%%%%%%%%%%%%%%%%%%%%%%%%%%%%%%%

The second part of the full time-walk correction accounts for additional position-dependent effects as
discussed in Section~\ref{sec-bench}. To simplify the algorithm the approach we take is to determine
first a position-independent correction for the left-side and right-side PMTs. In a second step we then
fit a second-order polynomial to the counter hit time vs. hit position along the counter.
Figure~\ref{twalk-pos} shows the distribution before and after the second correction. The time employed
for this step is the track hit time at the vertex (relative to the vertex-corrected RF time) averaging the
left and right PMT hit times. The before distribution where only the position-independent time-walk
correction is applied, reveals a characteristic ``smile'' pattern, which reflects the unaccounted for
position-dependent time-walk effects on the measured times incorporating both the left and right PMTs
each with their own linearly falling time-walk parameters when moving away from each PMT as discussed in
Section~\ref{sec-bench}. Effectively this approach actually accounts for all remaining position dependences
in the calibration parameters. Specifically it also takes care of the effective velocity changes with position
along the bar moving away from the PMT. However, the dominant position-dependence accounted for here is
related to the time-walk.

%%%%%%%%%%%%%%%%%%%%%%%%%%%%%%%%%%%%%%%%%%%%%%%%%%%%%%%%%%
\begin{figure}[htbp]
\vspace{2.5cm}
\begin{picture}(50,50) 
\put(0,-75)
{\hbox{\includegraphics[width=0.5\textwidth,natwidth=610,natheight=642]{pics/p1b-posdep1.pdf}}}
\put(200,-75)
{\hbox{\includegraphics[width=0.5\textwidth,natwidth=610,natheight=642]{pics/p1b-posdep2.pdf}}}
\end{picture} 
\caption{(Color Online) Plot of $t'$ (ns) vs. hit position (cm) along the bar from beam data after the
position-independent time-walk correction (left) and after the ad hoc second-order polynomial correction
to remove the residual coordinate dependence (right).}
\label{twalk-pos}
\end{figure}
%%%%%%%%%%%%%%%%%%%%%%%%%%%%%%%%%%%%%%%%%%%%%%%%%%%%%%%%%%

\subsubsection{Counter-to-Counter Time Alignment}
\label{sec-talign}

The flight time of a charged particle from the reaction vertex to an FTOF counter is given by:

\begin{equation}
TOF = \overline{t}_{hit} - t_{ST},
\end{equation}

\noindent
where $\overline{t}_{hit}$ is the average FTOF counter hit time and $t_{ST}$ is the event start time.
The event start time is associated with the RF but needs to be synchronized with the particular RF
beam bucket associated with the event. The beam bunch width within the RF beam bucket is only
about 2~ps and, therefore, represents a precise time marker. However, as the RF time signal has a
period of $T_{RF}$, it is not a priori known which RF beam bucket is the one associated with the
event that led to the hit in the FTOF counter.

The determination of the absolute flight time of charged particle tracks from the reaction vertex
to an FTOF counter is performed in two steps. In the first step, fine timing offsets (binned in the
25~ps TDC LSB) are determined to align the FTOF hit times traced back to the vertex for each
counter within the RF time window. In the second step, coarse timing offsets binned in units of the
RF period $T_{RF}$ are determined to select the specific RF beam bucket associated with the event.

The fine timing alignment algorithm uses the FTOF hit time traced back to the event vertex relative
to the RF to align the vertex times of all FTOF hits (modulo $T_{RF}$). However, instead of using the
separate left and right PMT hit times as in Eq.(\ref{tres}), this algorithm uses the average counter
hit times, 

\begin{equation}
t_{res}' = mod \left[ \left( \left(\overline{t}_{hit} - \frac{L}{\beta c} \right) -
\left(t_{RF} + \frac{z_{vert}}{\beta_e c} \right) \right), T_{RF} \right].
\end{equation}

Figure~\ref{rfp-plot} shows the $t_{res}'$ distribution for one representative FTOF counter. The
centroid of the Gaussian fit gives the fine timing offset. The width of the Gaussian fit is a measure
of the effective time resolution of the counter. To display the full $t_{res}'$ distribution avoiding any
wrap-around effects near the edges of the $T_{RF}$ range, the algorithm plots the $t_{res}'$ distribution
in a range of $\pm T_{RF}/2$ about the peak channel in the distribution.

%%%%%%%%%%%%%%%%%%%%%%%%%%%%%%%%%%%%%%%%%%%%%%%%%%%%%%%%%%
\begin{figure}[htbp]
\vspace{4.0cm}
\begin{picture}(50,50) 
\put(35,202)
{\hbox{\includegraphics[width=0.65\textwidth,natwidth=610,natheight=642,angle=-90]{pics/rfp-plot.pdf}}}
\end{picture} 
\caption{(Color Online) Distribution of the FTOF hit times from beam data traced back to the vertex relative
to the RF (ns) for one representative FTOF panel-1b counter with the Gaussian plus background fit overlaid
to determine the counter RF offset and the effective counter time resolution. As $t'$ is defined using the
modulus of $T_{RF}$ this distribution is limited to span $\pm T_{RF}/2$.}
\label{rfp-plot}
\end{figure}
%%%%%%%%%%%%%%%%%%%%%%%%%%%%%%%%%%%%%%%%%%%%%%%%%%%%%%%%%%

After the fine timing offset calibration, the counter timing is precisely aligned modulo $T_{RF}$. The next
step in the FTOF timing calibration is to fix the measured hit times for all counters to the specific RF bunch
associated with the event.  This is carried out using coincidences of charged particle tracks to link the hit
times of all counters across the full FTOF system.

The coarse timing offset algorithm (called P2P for paddle-to-paddle) selects events with two forward-going
charged tracks and computes the vertex time difference between any given FTOF counter relative to hits in
all of the other FTOF counters,

\begin{equation}
t_{P2P} = t_{vert}^1 - t_{vert}^2,
\end{equation}

\noindent
where,

\begin{equation}
t_{vert}^i = \overline{t}_{hit}^i - \frac{L}{\beta c}.
\end{equation}

At this point the counter timing has already been aligned to within a multiple of $T_{RF}$. Note that
particle identification of each track is given by the Event Builder, and as both tracks are assumed to
originate from the same reaction vertex, no vertex time corrections are necessary. The algorithm
adjusts the vertex time differences between all counters to zero. The coarse time offsets represents
a single parameter for each counter that is restricted to values of $n \cdot T_{RF}$, with
$n = 0, \pm 1, \pm 2, ...$.

Figure~\ref{p2p-plot} shows the $t_{P2P}$ distribution for one representative FTOF counter before and
after the coarse timing alignment. As expected, the histogram is dominated by events in a single channel
(of width $T_{RF}$) centered at $T_{RF} = 0$. As these constants are predominantly determined by the
fixed system cable lengths, of which there are four different lengths used to connect the panel-1a and the
panel-1b counters, the constants primarily reflect the differences in the signal propagation times along
the signal cables. Note that the algorithm specifically identifies two track events for the calibration and
does not consider hits in panel-1b and panel-1a associated with the same track.

%%%%%%%%%%%%%%%%%%%%%%%%%%%%%%%%%%%%%%%%%%%%%%%%%%%%%%%%%%
\begin{figure}[htbp]
\vspace{2.7cm}
\begin{picture}(50,50) 
\put(-2,-70)
{\hbox{\includegraphics[width=0.50\textwidth,natwidth=610,natheight=642]{pics/p2p-plot1.pdf}}}
\put(197,-70)
{\hbox{\includegraphics[width=0.50\textwidth,natwidth=610,natheight=642]{pics/p2p-plot2.pdf}}}
\end{picture} 
\caption{Distribution of the vertex time differences (ns) for tracks in a single representative FTOF
counter compared to tracks in all other FTOF counters using event samples with two forward-going
charged particle tracks. (Left) Before P2P corrections and (right) after P2P corrections. The histogram
is sorted in bins of $T_{RF}$.}
\label{p2p-plot}
\end{figure}
%%%%%%%%%%%%%%%%%%%%%%%%%%%%%%%%%%%%%%%%%%%%%%%%%%%%%%%%%%

\subsubsection{TDC Calibration}
\label{sec-tdccal}

The final calibration step is the calibration of the TDCs. This calibration is a single constant for each
TDC channel in the system that converts the measured TDC channel bin into time. The nominal TDC
LSB is 25~ps for the CAEN VX1290A and V1190A TDC units employed for the FTOF readout (see
Section~\ref{sec-elec}).

The calibration is completed by fitting the PMT time residuals of Eq.(\ref{tres}) vs. TDC channel using
a linear function. The TDC calibration is the value that fixes the slope of $t_{res}'$ to be zero.
Figure~\ref{tdc-plot} shows the distribution of $t_{res}'$ vs. TDC for a representative FTOF counter.
Any bin-to-bin $\Delta t$ variations reflect remaining integral non-linearities in the measured TDC
compensation tables (see Section~\ref{sec-elec}). At the present time a single conversion constant
of 23.45~ps/channel is employed for the FTOF system TDCs.

%%%%%%%%%%%%%%%%%%%%%%%%%%%%%%%%%%%%%%%%%%%%%%%%%%%%%%%%%%
\begin{figure}[htbp]
\vspace{3.7cm}
\begin{picture}(50,50) 
\put(70,168)
{\hbox{\includegraphics[width=0.50\textwidth,natwidth=610,natheight=642,angle=-90]{pics/tdc-plot.pdf}}}
\end{picture} 
\caption{(Color Online) Distribution of $t_{res}'$ (ns) vs. TDC channel for one representative FTOF
panel-1b PMT. The TDC conversion constant for each channel is that which forces the slope of the
linear fit to be zero.}
\label{tdc-plot}
\end{figure}
%%%%%%%%%%%%%%%%%%%%%%%%%%%%%%%%%%%%%%%%%%%%%%%%%%%%%%%%%%

The CAEN TDCs used for the FTOF are readout with a 24~ns clock strobed so that all TDC times are
referenced to an edge of this clock. The CLAS12 trigger comes on the edge of a clock with a period of
4~ns and the TDC stop will not occur until the next 24~ns clock edge. The use of these two different
clocks introduces a delay between the trigger and the TDC stop given by $n \cdot 4$~ns, with
$n = 0 \to 5$ (referred to as the six-fold TDC cycle ambiguity) where $n$ is the phase. A TDC jitter
correction is made to define the value of the phase $n$ that is valid for the entire data run.

The average hit time resolution for the FTOF from the TDCs is about 80~ps and that from the FTOF
FADCs, given the rapid fall time of the fast PMT signals that provide for only 2-3 samples on the falling
edge is only about 1~ns. A matching requirement of 10~ns between the TDC time and the FADC time is
employed during event reconstruction. While this matching requirement still needs to be tuned further,
it is already reasonably efficient at allowing the FADC hits to be matched with the TDC hits. This is
important as due to the different thresholds on the discriminators and the FADCs, the number of
entries in the hits lists can be up to a factor of two different. The matching criteria is also essential in
order to assign the correct ADC information to the hit not only for the time-walk correction that directly
uses the measured ADC but also for the energy loss computation.

\subsubsection{Counter Time Resolutions}
\label{tres-beam}

The effective time resolutions for each counter determined after complete calibrations of the FTOF
system are shown in Fig.~\ref{eff-tres}. These measurements are from a beam data run with 10.6-GeV
electrons incident on a 5-cm long liquid-hydrogen target. These time resolutions represents the current
quality of the overall CLAS12 calibrations. The results are based on calibration procedures that are not
yet fully optimized, as well as uncertainties in the reconstructed momentum and path length from the
forward track reconstruction. Note that the time resolution floor-term $\sigma_0$ discussed in
Section~\ref{res-sec} and Eq.(\ref{timing-func}) does not include the contributions from the
reconstructed path length uncertainties. These uncertainties are polar and azimuthal angle dependent.
Near the torus coils the true magnetic field has different variations than accounted for in our conductor
model used to generate the field for the event reconstruction. Furthermore, the path length uncertainties
grow strongly for high momentum tracks at small angles, which represent the dominant part of our kinematic
phase space at 10.6~GeV. It is also important to mention that studies of the CLAS12 subsystem detector
alignment based on survey data and based on zero-field straight track data are in progress. Misalignments
of the detector affect the quality and accuracy of the reconstruction. When all of these uncertainties and
misalignments are accounted for their contribution to the floor-term of the resolution function will be
reduced.

Nevertheless the time resolutions already achieved meet the system design specifications in the forward
direction outlined in Section~\ref{sec:overview} and shown in Table~\ref{spec-table}. For the panel-2
counters the time resolutions are 200-250~ps, but the calibrations are presently limited by statistics and by
the use of low-momentum protons in the calibration sample (as discussed in Section~\ref{beam-data-calib}). 
With these resolutions, the quality of the particle identification in the Forward Detector of CLAS12 allows
the experimental program in Hall~B to reach its goals. As further operating experience with CLAS12 is
gained, we expect to realize further modest but important improvements in the FTOF time resolution that
will allow $\pi/K$, $\pi/p$, and $K/p$ separation in the Forward Detector of CLAS12 to be pushed to higher
momenta than currently seen.

%%%%%%%%%%%%%%%%%%%%%%%%%%%%%%%%%%%%%%%%%%%%%%%%%%%%%%%%%
\begin{figure}[htbp]
\vspace{3.6cm}
\begin{picture}(50,50) 
\put(-5,-70)
{\hbox{\includegraphics[width=1.2\textwidth,natwidth=610,natheight=642]{pics/res-r5038.pdf}}}
\end{picture} 
\caption{The measured effective time resolution (ps) vs. counter number for each of the FTOF counters
in sector~1 as determined using final state leptons and pions for panel-1a and panel-1b and pions and
protons for panel-2.}
\label{eff-tres}
\end{figure}
%%%%%%%%%%%%%%%%%%%%%%%%%%%%%%%%%%%%%%%%%%%%%%%%%%%%%%%%%

\subsubsection{Hit Clustering and Hit Times}
\label{cluster}

The reconstructed FTOF counter hit times need to account for the time delays along the readout
path that include the PMT and voltage divider signal transit times, the signal propagation times along
the signal cables and the electronics, and the time-walk effects associated with the use of leading edge
discriminators for the readout. Full details on the FTOF reconstruction in terms of reconstructed hit
times and energy, the reconstruction algorithm, the time, energy, and coordinate uncertainties, and
the hit clustering and matching algorithms are provided in Ref.~\cite{ftof-recon}.

The track hit times reconstructed from the readout of the left and right PMTs are given by:

\begin{equation}
t_{L,R} = (C_{TDC} \cdot TDC_{L,R}) - t_{L,R}^{walk} \mp \frac{C_{LR}}{2} + C_{RF} + C_{p2p},
\end{equation}

\noindent
where $C_{TDC}$ is the TDC channel to time conversion factor, $TDC$ is the measured TDC value relative
to the trigger signal, $t^{walk}$ is the time-walk correction, $C_{LR}$ is the time shift to center the TDC
difference distribution relative to the track coordinate at zero, and $C_{RF}$ and $C_{p2p}$ are the time
shifts to align all of the counter times with respect to the RF and to each other, respectively.

The hit times of the passing charged particle relative to the trigger signal can be determined separately 
from the times $t_L$ and $t_R$ measured by the left and right PMTs, respectively, and are given by:

\begin{equation}
t_{hit}^{L,R} = t_{L,R} - \frac{d^{DC}_{L,R}}{v_{eff}}.
\end{equation}

\noindent
where $d^{DC}_{L,R}$ are the distances from the track hit point along the bar relative to each end of the
bar as determined by the drift chamber tracking information. The average FTOF hit time is then given by:

\begin{equation}
\bar{t}_{hit} = \frac{1}{2} ( t_{hit}^L + t_{hit}^R ) = \frac{1}{2} \left[ t_L + t_R - \frac{L}{v_{eff}} \right].
\end{equation}

Particle trajectories from the target passing through the plane of a given counter array (i.e. panel-1a,
panel-1b, or panel-2) can pass through up to two adjacent counters. To determine the full deposited
energy in a given counter layer, a clustering algorithm is used to match the FTOF counter hits with
the track trajectory at the location of the FTOF array. Neighboring FTOF hits that match to the
trajectory are assigned to be part of a cluster. The assigned cluster energy is then the sum of the
deposited energies in both counters,

\begin{equation}
  E_{cluster} = \sum_{i=1}^{2} E_{dep}^i.
\end{equation}

\noindent
The relevant path length through the layer is then determined from ray tracing of the drift chamber
track. This path length through the layer is then used to compute $dE/dx$.

The hit time associated with the cluster in a given layer is currently based on assigning the hit time as
the counter hit with the closest geometrical match to the charged track extrapolated from the drift
chambers to the FTOF system. The algorithm will be further studied and optimized in the future.

In the forward direction where there is a defined hit or a defined cluster in both panel-1b and panel-1a,
a cluster matching algorithm (using timing and hit-coordinate information) is applied to determine if the
hit or cluster in panel-1b and the hit or cluster in panel-1a are associated with the same incident track.
Once the clusters in panel-1b and panel-1a arrays are matched together as associated with the same
incident track, the corrected FTOF time based on the panel-1a and panel-1b cluster times is computed using
a time resolution weighting according to the counter in each cluster with the largest energy deposition.

\begin{equation}
  t_{corr} = \frac{\displaystyle \frac{\displaystyle t_{1b}^{cluster}}{\displaystyle \delta_{1b}} +
    \frac{\displaystyle (t_{1a}^{cluster} - \Delta r/\beta c)}{\displaystyle \delta_{1a}}}
  {\displaystyle \left( \frac{\displaystyle 1}{\displaystyle \delta_{1b}} +
    \frac{\displaystyle 1}{\displaystyle \delta_{1a}} \right)}.
\end{equation}

\noindent
Here $\delta_{1a,1b}$ are the effective time resolutions measured for the counters determined during
the fine timing algorithm step (see Section~\ref{sec-talign}), and $t_{1a,1b}^{cluster}$ are the cluster
times defined with respect to the RF time. The term $\Delta r/(\beta c)$ accounts for the path length
difference between the panel-1b cluster hit coordinate and the panel-1a cluster hit coordinate and comes
from forward tracking information. As $\beta$ depends on the FTOF time, it is assumed that it is based
on the panel-1b time information.

Given the effective FTOF counter resolutions shown in Fig.~\ref{eff-tres}, the overall FTOF hit
time resolution is improved by 15-20\% when combining the times from panel-1b and panel-1a in this
manner. Of course, if the track interacts with only panel-1a or with only panel-1b due to the slightly
different solid angles of coverage of the arrays, then only the single plane hit time is used in the
event reconstruction. This is also accounted for in the CLAS12 Monte Carlo simulation code.
  
\subsection{Beam Performance}  
\label{sec:beam}

The first in-beam characterization of the FTOF system took place during the Dec. 2017 to Feb. 2018
CLAS12 Engineering Run and subsequently during the first physics production running periods that took
place from Mar. - May 2018 and Sep. - Dec. 2018. During these periods the performance of the FTOF
system was tested at different beam energies (2.2, 6.4, 10.6~GeV), different torus and solenoid magnetic
field strengths and polarities (from 0 field to full field for both magnets), and over a range of beam-target
luminosities up to twice the nominal planned CLAS12 luminosity of $1 \times 10^{35}$~cm$^{-2}$s$^{-1}$.
In this section the measured scaler rates and PMT currents as a function of beam current are presented,
as well as the reconstruction results and particle identification capabilities relative to the system
specifications based on the current system calibrations.

\subsubsection{FTOF Rates and PMT Currents}

The count rates during beam operations can be viewed during data taking using the scalers associated
with the discriminators or with the FADCs. The threshold applied for these scalers are set at 1~MeV.
During a beam current scan from 5~nA to 70~nA with a 10.6-GeV electron beam incident upon the 5-cm long
liquid-hydrogen target (where 70~nA corresponds to the roughly to the nominal design luminosity for CLAS12
$1 \times 10^{35}$~cm$^{-2}$s$^{-1}$) the average count rate in the different FTOF counters was studied.
Averaged over the three different arrays, the results shown in Fig.~\ref{ftof-rates}, display a reasonably
linear behavior. The rates in panel-1a are about a factor of two larger than those for panel-1b. This is in
agreement with the fact that the panel-1a arrays are 2.5 times wider than the panel-1b arrays. However, some
portion of the incident radiation is absorbed in the panel-1b counters reducing the flux seen in panel-1a. At the
nominal luminosity of CLAS12 the average measured rates in the panel-1b counters are about 500~kHz and
those in panel-1a are about 1~MHz.

%%%%%%%%%%%%%%%%%%%%%%%%%%%%%%%%%%%%%%%%%%%%%%%%%%%%%%%%%
\begin{figure}[htbp]
\vspace{5.2cm}
\begin{picture}(50,50) 
\put(-10,-70)
{\hbox{\includegraphics[width=1.2\textwidth,natwidth=610,natheight=642]{pics/ftof-rates.pdf}}}
\end{picture} 
\caption{(Color Online) Measured FTOF counter rates (kHz) for 10.6-GeV electrons on a liquid-hydrogen
target as a function of beam current (nA). The nominal operating luminosity of CLAS12 of $1 \times 10^{35}$
cm$^{-2}$s$^{-1}$ corresponds to a beam current of $\sim$70~nA. The red circles correspond to the
average panel-1a counter rates, the blue squares to the average panel-1b counter rates, and the green
triangles to the average panel-2 counter rates.}
\label{ftof-rates}
\end{figure}
%%%%%%%%%%%%%%%%%%%%%%%%%%%%%%%%%%%%%%%%%%%%%%%%%%%%%%%%%

Based on a detailed simulation of the full CLAS12 detector and beamline based on our GEANT-4 Monte
Carlo~\cite{clas12-gemc}, the response of the FTOF with an 11~GeV electron beam incident upon a 5-cm
liquid-hydrogen target has been studied. Shown in Fig.~\ref{ftof-gemc} is a comparison of the contributions
to the overall rate associated with hits above the readout threshold separated into contributions from
photons, neutrons, and charged particles. By far the dominant contribution to the overall measured FTOF
rate is associated with low energy photons, whose energy deposition in the counters is significantly less
than the contribution from minimum-ionizing hadrons.

%%%%%%%%%%%%%%%%%%%%%%%%%%%%%%%%%%%%%%%%%%%%%%%%%%%%%%%%%
\begin{figure}[ht]
\vspace{3.8cm}
\begin{picture}(50,50) 
\put(25,-70)
{\hbox{\includegraphics[width=0.7\textwidth,natwidth=610,natheight=642]{pics/mc-rates.pdf}}}
\end{picture} 
\caption{(Color Online) Simulation results for the FTOF counter rates (kHz) for each sector for
11~GeV electrons on a liquid-hydrogen target at the nominal CLAS12 design luminosity of
$1 \times 10^{35}$~cm$^{-2}$s$^{-1}$. The rates are calculated for a 1-MeV deposited-energy
threshold and expressed in kHz. Here the left-most group of 62 paddles corresponds to panel-1b,
the middle group of 23 paddles corresponds to panel-1a, and the right-most group of 5 paddles
corresponds to panel-2.}
\label{ftof-gemc}
\end{figure}
%%%%%%%%%%%%%%%%%%%%%%%%%%%%%%%%%%%%%%%%%%%%%%%%%%%%%%%%%

The average PMT current is directly proportional to the average number of photoelectrons
$\langle N_{pe} \rangle$ created at the photocathode by the scintillation light and the average incident
charged particle event rate $\langle R \rangle$. This current can be expressed as:

\begin{equation}
\langle i_{PMT} \rangle = \langle N_{pe} \rangle \cdot Q_e \cdot G \cdot \langle R \rangle,
\end{equation}

\noindent
where $Q_e = 1.6 \times 10^{-19}$ C/e is the electron charge and $G$ is the PMT gain (assumed to
be $1 \times 10^6$). Using the photoelectron statistics estimated in Section~\ref{res-sec},
Fig.~\ref{mc-pmt-currents} shows the predictions of the PMT anode currents for all of the FTOF
counters at nominal operating luminosity at 11~GeV from our detailed GEANT-4 Monte Carlo
studies~\cite{gemc-cn2017}. These predictions show typical PMT anode currents in the panel-1a and
panel-1b PMTs at the level of 5~$\mu$A to 10~$\mu$A increasing linearly with counter length. Direct
in-beam measurements of the PMT anode currents are shown in Fig.~\ref{pmt-currents}. The in-beam
measurements are in good accord with the simulation expectations.

%%%%%%%%%%%%%%%%%%%%%%%%%%%%%%%%%%%%%%%%%%%%%%%%%%%%%%%%%
\begin{figure}[htbp]
\vspace{3.8cm}
\begin{picture}(50,50) 
\put(20,-70)
{\hbox{\includegraphics[width=0.7\textwidth,natwidth=610,natheight=642]{pics/mc-currents.pdf}}}
\end{picture} 
\caption{(Color Online) Computed PMT currents ($\mu$A) from Monte Carlo studies for each sector
at the nominal operating luminosity of CLAS12 of $1 \times 10^{35}$~cm$^{-2}$s$^{-1}$ with 11~GeV
electrons incident upon a liquid-hydrogen target. Here the left-most group of 62 paddles corresponds
to panel-1b, the middle group of 23 paddles corresponds to panel-1a, and the right-most group of 5
paddles corresponds to panel-2.}
\label{mc-pmt-currents}
\end{figure}
%%%%%%%%%%%%%%%%%%%%%%%%%%%%%%%%%%%%%%%%%%%%%%%%%%%%%%%%%

%%%%%%%%%%%%%%%%%%%%%%%%%%%%%%%%%%%%%%%%%%%%%%%%%%%%%%%%%
\begin{figure}[htbp]
\vspace{4.2cm}
\begin{picture}(50,50) 
\put(60,-70)
{\hbox{\includegraphics[width=0.8\textwidth,natwidth=610,natheight=642]{pics/full-ftof.pdf}}}
\end{picture} 
\caption{(Color Online) Measurements of the PMT anode current for representative panel-1b left-side
PMTs ($N_{counter}=10, 20, 40$) as a function of beam current with a 10.6-GeV electron beam incident
upon a 5-cm long liquid-hydrogen target.}
\label{pmt-currents}
\end{figure}
%%%%%%%%%%%%%%%%%%%%%%%%%%%%%%%%%%%%%%%%%%%%%%%%%%%%%%%%%

\subsubsection{Reconstruction Results}

The particle identification in the Forward Detector of CLAS12 relies heavily on the combination of
measured charged particle momentum and the flight time from the target to the respective FTOF
counter. The vertex time is determined with respect to the accelerator RF, modulo the RF period
$T_{RF}$. The beam bunch for each event is identified using the flight time of scattered electrons
or high-momentum pions, traced back to the interaction point. The FTOF resolution of $< 200$~ps
allows clear selection of the correct beam bucket. In Fig.~\ref{fig:masses} we show the distribution
of masses for all reconstructed positively charged hadrons without any kinematic cuts other than
those imposed by the detector acceptance for the data taken with a 10.6-GeV electron beam incident
upon a 5-cm long liquid-hydrogen target and after initial calibrations of the FTOF system. A clear
separation of pions and protons can be seen from these data.

%%%%%%%%%%%%%%%%%%%%%%%%%%%%%%%%%%%%%%%%%%%%%%%%%%%%%%%%%
\begin{figure}[htbp]
\vspace{3.2cm}
\begin{picture}(50,50) 
\put(0,-185)
{\hbox{\includegraphics[width=1.0\textwidth,natwidth=610,natheight=642]
{pics/ftof-mass.pdf}}}
\end{picture} 
\caption{Reconstructed mass squared (GeV$^2$) for positively charged particles for all counters in
panel-1a (left), panel-1b (middle), and panel-2 (right) from beam data with a 10.6-GeV electron beam
incident on a liquid-hydrogen target.}
\label{fig:masses}
\end{figure}
%%%%%%%%%%%%%%%%%%%%%%%%%%%%%%%%%%%%%%%%%%%%%%%%%%%%%%%%

A plot of velocity versus momentum is shown in Fig. \ref{fig:betavsp} for positively charged particles,
displaying the overall particle identification possible with this detector through the separation of the
different particle species. Here the distributions are presented separately for the counters in panel-1a,
panel-1b, and panel-2 summed over all CLAS12 Forward Detector sectors. These distributions qualitatively
show the particle separation for $\pi/K$, $\pi/p$, and $K/p$ vs. momentum as required by the system
specifications in Section~\ref{sec:overview} and Table~\ref{spec-table}.

%%%%%%%%%%%%%%%%%%%%%%%%%%%%%%%%%%%%%%%%%%%%%%%%%%%%%%%%%
\begin{figure}[htbp]
\vspace{4.0cm}
\begin{picture}(50,50) 
\put(-15,-200)
{\hbox{\includegraphics[width=1.1\textwidth,natwidth=610,natheight=642]{pics/bvsp.pdf}}}
\end{picture} 
\caption{(Color Online) Velocity of positive hadrons ($\beta$) versus momentum (GeV) for all counters
in panel-1a (left), panel-1b (middle), and panel-2 (right) from beam data with a 10.6~GeV electron beam
incident on a liquid-hydrogen target.}
\label{fig:betavsp}
\end{figure}
%%%%%%%%%%%%%%%%%%%%%%%%%%%%%%%%%%%%%%%%%%%%%%%%%%%%%%%%%%

\section{Summary}
\label{sec:summary}

We have designed and assembled a time-of-flight system for the Forward Detector of the new CLAS12
Spectrometer in Hall~B at Jefferson Lab known as the Forward Time-of-Flight or FTOF detector. This
system consists of 90 scintillation bars in each of the six forward sectors of the CLAS12 Forward Detector
for a total of 540 counters. This design is based on rectangular counters varying in length from 17~cm to
426~cm in three different counter arrays in each sector. In the polar angle range from 5$^\circ$ to
35$^\circ$ the FTOF system consists of two layers of counters referred to as panel-1a and panel-1b.
The panel-1a counters were refurbished from the forward TOF counters that were part of the original
CLAS spectrometer. The panel-1b counters were newly constructed for CLAS12. Together the timing
measurements from these counters currently provide effective time resolutions from 50~ps at small
angles to 100~ps at large angles. In the polar angle range from 35$^\circ$ to 45$^\circ$ the FTOF
system consists of the panel-2 counters that provide effective time resolutions of about 250~ps. With
these time resolutions the FTOF system can separate $\pi/K$ to 2.8~GeV, $K/p$ to 4.8~GeV, and $\pi/p$
to 5.4~GeV with 4$\sigma$ separation with up to an order of magnitude difference in the relative yields.
The specifications are sufficient to meet the design particle identification requirements in the forward
direction for the full CLAS12 physics program. The performance of the FTOF system was verified in
extensive bench studies in our cosmic ray test stands, as well as after installation in the first beam runs
with the CLAS12 system in the period from Dec. 2017 to Dec. 2018. 

\ack

We benefited greatly from useful discussions with and assistance from Sergey Boyarinov, Chris Cuevas,
Haiyan Lu, Cole Smith, Elton Smith, and Veronique Ziegler. The panel-1b counters were designed and built
at the University of South Carolina with the help of Jesse Anderson, Gleb Fedotov, Robert Hedrick, Gary
Hollis, Evan Phelps, Robert Steinman, Felician Stratman, Ye Tian, Arjun Trivedi, Nick Tyler, and a dedicated
team of graduate and undergraduate students. We also thank the Hall~B technical crew for their efforts
during counter installation and cabling. This work was supported in part by DOE Contract
DE-AC05-84ER40150 and NSF Grant PHY-1205782.

\begin{thebibliography}{99}

\bibitem{clas12-nim}
V.D. Burkert {\it et al.}, to be published in Nucl. Inst. and Meth. A, (2020). (see this issue)
  
\bibitem{clas-nim}
B.A. Mecking {\it et al.}, Nucl. Inst. and Meth. {\bf A503}, 513 (2003).

\bibitem{tof-nim}
E.S. Smith {\it et al.}, Nucl. Inst. and Meth. {\bf A432}, 265 (1999).

\bibitem{bicron}
Bicron, 12345 Kinsman Road, Newbury, OH 44065, http://www.bicron.com.

\bibitem{hamamatsu}
Hamamatsu Photonics, http://www.hamamatsu.com.
  
\bibitem{nim-p1b}
R. Gothe {\it et al.}, {\it A New Time-of-Flight System for CLAS12}, to be submitted to Nucl. Inst.
and Meth A, (2020).

\bibitem{ftof-geom}
D.S. Carman, {\it Forward Time-of-Flight Geometry for CLAS12}, CLAS12-Note 2014-005.
https://misportal.jlab.org/mis/physics/clas12/viewFile.cfm/2014-005.pdf?documentId=13

\bibitem{kuhlen}
M. Kuhlen, M. Moszynski, R. Stroynowski, E. Wicklund, and B. Milliken, Nucl.  Instr. and Meth.
{\bf A301}, 223 (1991).

\bibitem{scint-mat-ref}
https://www.crystals.saint-gobain.com/sites/imdf.crystals.com/files/documents/bc400-404-408-412-416-data-sheet.pdf

\bibitem{et-ref}
ADIT Electron Tubes, 300 Crane St, Sweetwater, TX 79556
  
\bibitem{photonis}
Photonis Imaging Sensors, http://www.photonis.com

\bibitem{ftof-shields}
D.S. Carman and V. Baturin, {\it CLAS12 FTOF Studies: Rate and Magnetic Shielding Effects on PMT Timing
Resolutions}, CLAS-Note 2011-018.\\
https://misportal.jlab.org/ul/Physics/Hall-B/clas/viewFile.cfm/2011-018.pdf?documentId=653
  
\bibitem{tdc-manual}
http://www.caen.it/servlet/checkCaenManualFile?Id=12125
  
\bibitem{inl-tables}
E. Jastremski, https://www.jlab.org/Hall-B/ftof/manuals/caen-inl-notes.pdf

\bibitem{fadc-manual}
JLab 250~MHz FADC manual, https://www.jlab.org/Hall-B/ftof/manuals/FADC250UsersManual.pdf
  
\bibitem{epics}
EPICS (Experimental Physics and Industrial Control System),\\ http://www.aps.anl.gov/epics

\bibitem{dsc-cn2013-001}
D.S. Carman, {\it CLAS12 FTOF Panel-1a and Panel-2 Refurbishment and Baseline Test Results}, CLAS12-Note 2013-001.\\
https://misportal.jlab.org/mis/physics/clas12/viewFile.cfm/2013-001.pdf?documentId=1

\bibitem{Gi86} 
R.T. Giles, F.M. Pipkin, and J.P. Wolinski, Nucl. Instr. and Meth. A {\bf 252}, 41 (1986).

\bibitem{kajino}
T. Yamaoka, F. Kajino, I. Tada, S. Hayashi, {\it Absolute Number Calibration of Photoelectrons of
Photomultiplier Tubes Using the Nature of Statistical Distribution}, Proceedings of the 28th International
Cosmic Ray Conference, July 31 - August 7, 2003, editors: T. Kajita, Y. Asaoka, A. Kawachi, Y. Matsubara, and
M. Sasaki, p. 2871 (2003).

\bibitem{pmt-currents}
D.S. Carman, {\it Forward Time-of-Flight PMT Currents}, FTOF internal note, (2014).
https://www.jlab.org/Hall-B/ftof/notes/ftof\_currents.pdf

\bibitem{ftof-calib}
D.S. Carman, {\it Description of the Calibration Algorithms for the CLAS12 Forward Time-of-Flight System},\\
https://www.jlab.org/Hall-B/ftof/notes/ftof\_calib.pdf

\bibitem{ftof-recon}
D.S. Carman, {\it Forward Time-of-Flight Reconstruction for CLAS12},\\
https://www.jlab.org/Hall-B/ftof/notes/ftof-recon.pdf

\bibitem{clas12-gemc}
CLAS12 GEANT4 Monte Carlo suite {\it gemc}, http://gemc.jlab.org.

\bibitem{gemc-cn2017}
R. De Vita, D.S. Carman, C. Smith, S. Stepanyan, and M. Ungaro, {\it Study of the Electromagnetic Background
Rates in CLAS12}, CLAS12-Note 2017-016.\\
https://misportal.jlab.org/mis/physics/clas12/viewFile.cfm/2017-016.pdf?documentId=52

%\bibitem{scint-spec}
%Eljen Technologies scintillator specifications,\\
%http://www.eljentechnology.com/images/technical\_library/Physical\_Constants\_Plastic.pdf

\end{thebibliography}

\end{document}
