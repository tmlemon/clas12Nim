\section{Performance}
\label{Performance}

This section discusses preliminary performance of the ECAL using physics data production runs from Fall 2018.
This run period used beam energies of 10.6, 7.5, and 6.5~GeV with the CLAS12 torus magnet polarity set for both
inbending and outbending electrons. The data presented here used the standard ECAL FPGA-based electron
trigger~\cite{nim:trig} with a 15~kHz trigger rate and beam currents of 40~nA on a 5~cm LH$_2$ target.

\subsection{Electron Response}

The response of the PCAL and EC components of a single ECAL module to high energy electrons is shown in
Fig.~\ref{fig:S10_1_000}. The beam energy was 10.6~GeV and outbending scattered electrons were selected by
the Event Builder (EB) service by matching a negative charge forward track to ECAL clusters in PCAL, ECIN, and
ECOU and requiring an activated High Threshold Cherenkov Counter (HTCC)~\cite{nim:htcc} mirror segment that
matches the same track. The plot clearly shows the correlations between energy reconstructed in the forward
and rear sections of the ECAL, while the logarithmic $z$-scale emphasizes the range of fluctuations. The diagonal
lines are the total reconstructed  energy with the location of the total energy trigger threshold shown at PCAL+EC =
300~MeV and the other line showing the maximum deposited energy consistent with the scattering angle cutoff at
$\theta_{elec}=6^\circ$. Contributions from pions that exceed the 4.7~GeV threshold cutoff of the HTCC were
rejected in the hardware trigger using a PCAL energy threshold of 60~MeV.

The sampling fraction for electromagnetic showers is defined as the ratio of the total sampled energy
(PCAL+EC) to the incident particle energy. The measured sampling fraction for electrons in the momentum range 1-10~GeV is shown in
Fig.~\ref{fig:S10_1_0} as a function of the electron track momentum determined by the Forward Tracking 
drift chambers.  The measurement is compared to GEMC simulations of electrons impacting the central portion of the ECAL,
which show a dependence on the measured energy ranging from 0.22 to 0.255 over the expected range of electron
momentum. The comparison shows there are a few percent systematic deviations due possibly to residual calibration errors.

\begin{figure}[t]
\centering
\includegraphics[width=0.8\columnwidth,keepaspectratio]{img/S10_1_000.png}
\caption[]{Reconstructed shower energy in the PCAL and EC modules in response to electrons with a momentum
  range of 1-10~GeV and a polar scattering angle range of 6$^\circ$-34$^\circ$. The energy is not corrected for
  sampling fraction. The diagonal lines show the limits for the total reconstructed energy PCAL+EC.}
\label{fig:S10_1_000}
\end{figure}

\begin{figure}[t]
\centering
\includegraphics[width=1.0\columnwidth,keepaspectratio]{img/S10_1_0.png}
\caption[]{Top: Distribution of the ratio $E/P$ of the total reconstructed ECAL energy to the momentum of
  outbending forward tracks linked to clusters in PCAL, ECIN, and ECOU that were identified as electrons by the
  Event Builder service. Bottom: $E/P$ vs. ECAL measured energy (black points) compared to the GEMC prediction (black line). The inset
  shows a Gaussian fit to the overall $E/P$ distribution.}
\label{fig:S10_1_0}
\end{figure}

The calorimeter energy resolution can be expanded as a function of energy with the usual parameterization of
contributions \cite{ps1981}:
\begin{equation}
\biggl[\frac{\sigma(E)}{E}\biggr]^2 = \frac{\sigma^2_0}{E^2} + \frac{\sigma^2_1}{E} +\sigma^2_2 + \sigma^2_3 E 
\label{eq:sferror}
\end{equation}

\begin{itemize}
\item $\sigma_0$ - Pedestal noise, cross-talk;
\item $\sigma_1$ - Poisson statistics (sampling, PMT);
\item $\sigma_2$ - Calibration errors (PMT gains);
\item $\sigma_3$ - Shower leakage fluctuations.
\end{itemize}

Typically $\sigma_0$ is ignored, although for MIP-based calibration analysis the contribution may be non-negligible.
To minimize this contribution all CLAS12 PMT data are taken with FADC pedestals measured and subtracted
event-by-event. Shower leakage contributions to $\sigma_3$ are most important for inbending electrons impacting
the ECAL at the forward-most angles, where there is incomplete overlap of PCAL and EC and vanishing acceptance, and are not expected to be significant for the oubending data considered here.

Estimates of the contributions from $\sigma_1$ and $\sigma_2$ were performed using fits to the expected linear
dependence of the total relative variance on the inverse of the electron energy as shown in Fig.~\ref{fig:S10_1_1}.
The summary of these fits shows an average energy resolution of $\sigma_1 = 0.09$~GeV$^{1/2}$, which is consistent
with expectations from GEMC. The fit results for $\sigma_2$ of around $4\%$ is typical of the present instabilities
of PMT gain matching.
 
\begin{figure}[t]
\centering
\includegraphics[width=1.0\columnwidth,keepaspectratio]{img/S10_1_1.png}
\includegraphics[width=0.5\columnwidth,keepaspectratio]{img/S10_1_2.png}
\caption[]{Relative variance of the measured sampling fraction plotted vs. the inverse electron energy. The
  linear fits used to obtain the resolution and calibration variance terms in Eq.~\ref{eq:sferror} are summarized
  at the bottom, where $\sigma_1 (\blacksquare)$ and $\sigma_2 (\blacktriangle)$ are plotted vs. sector.}
\label{fig:S10_1_1}
\end{figure}

\begin{figure}[t]
\centering
\includegraphics[width=1.0\columnwidth,keepaspectratio]{img/S10_1_3.png}
\caption[]{Top: Histograms show the distribution of residuals in Sector~3 between the projected hit position of
  forward tracks for electrons and the reconstructed PCAL cluster position for the radial (left) and transverse
  (right) coordinates. Bottom: Sector dependence of radial (black) and transverse (red) residuals for PCAL (left)
  and ECIN (right). The error bars show the sigma of Gaussian fits to the residual distributions.}
\label{fig:S10_1_3}
\end{figure}

\subsection{Position Resolution}

The scintillator alignment and position resolution were estimated from comparison of the reconstructed position of
shower clusters with the extrapolation of the forward tracking trajectory state vector of electron tracks from the
target to the tracking planes of the PCAL and ECIN. Fig.~\ref{fig:S10_1_3} summarizes the electron track-cluster
matching residuals in the tilted local sector frame. Systematic offsets for the PCAL of $\approx 1$~cm and
$\approx 0.5$~cm are seen in the radial (perpendicular to the U strips) and transverse (along the direction of the
U strips) directions, respectively. These results are consistent with the present uncertainties in the scintillator
location. The ECIN residuals have not changed from the CLAS era, although the fitted sigmas of the residual
distributions are smaller, possibly due to the improved angular resolution of the CLAS12 forward tracking and
smaller multiple scattering at 10~GeV. The PCAL residual fits imply an angular resolution of $\approx 1.2$~mrad
for showers in the Forward Detector. The measured offsets, together with survey data, will be used to further
optimize the track matching cuts used for electron identification and background rejection. 

\begin{figure}[t]
\centering
\includegraphics[width=0.7\columnwidth,keepaspectratio]{img/opa.png}
\caption[]{Opening angle of the $\pi^0 \rightarrow \gamma \gamma$ decay vs. the energy asymmetry $X$ of
  the two photons, shown for the range of $\pi^0$ energies measured in CLAS12.}
\label{fig:opa}
\end{figure}

\subsection{Reconstruction of $\pi^0\rightarrow \gamma \gamma$ Decays}

Detection of the neutral $\pi$ meson requires reconstruction of the invariant mass $M$ of the
$\pi^0 \rightarrow \gamma \gamma$ decay, using:
\begin{equation}
M^2 = 2 E_1 E_2 (1-\cos \theta_{12}),
\label{eq:ivm}
\end{equation}
where $E_1$ and $E_2$ are the energies of the two photon showers in the ECAL and $\theta_{12}$ is their opening
angle. The $\pi^0$ energy can be determined from these same measurements in
terms of the photon energy asymmetry $X$:
\begin{equation}
E^2_{\pi} = \frac{m^2_{\pi}}{(1-\cos \theta_{12})(1-X^2)}  \\
X=\frac{E_1-E_2}{E_1+E_2}.
\label{eq:epi}
\end{equation}
Here $m_\pi$ corresponds to the physical $\pi^0$ mass of 134.98 MeV.
The reaction kinematics following from Eqs.~\ref{eq:ivm} and \ref{eq:epi} are shown in Fig.~\ref{fig:opa}. Clearly
both the invariant mass and energy reconstruction are dominated by the ECAL angular resolution and accuracy
at the highest energies, while at lower energies both the energy and angle resolutions contribute. Therefore this
measurement can be used to reveal and refine the systematics and consistency of the energy calibration, geometrical
alignment and cluster reconstruction of the ECAL.

\begin{figure}[h]
\centering
\includegraphics[width=1.0\columnwidth,keepaspectratio]{img/fx-pi0-fits.png}
\caption[]{Empirical fits to both the combinatorial background and the invariant mass peaks reconstructed from two
  photons detected in the same sector of the ECAL. These data represent symmetric
  $\pi^0 \rightarrow \gamma \gamma$ decays where both photons are within the energy bins indicated in the plot.
  Fits shown are for photon energies of $E_{\gamma}$ = 0.25 (left), 0.75 (middle), and 2.75~GeV (right).}
\label{fig:fx-pi0-fits}
\end{figure}

From Eq.~\ref{eq:ivm} the following expression for the invariant mass resolution can be derived, expressed in terms of the uncertainties of the measured quantities:
\begin{equation}
  \sigma_m  = \frac{m_{\pi}}{2}\left[\left(\frac{\sigma(E_1)}{E_1}\right)^2 + \left(\frac{\sigma(E_2)}{E_2}\right)^2
    + \sigma^2_{\theta_{12}}\frac{\sin^2 \theta_{12}}{(1-\cos \theta_{12})^2}\right]^{1/2}.
\label{eq:sigmam1}
\end{equation}
For symmetric decays, where $E_1 \approx E_2$ or $X\sim 0$, it follows from Eq.~\ref{eq:epi} that the invariant mass resolution $\sigma_m$
reduces to a dependence solely on the pion energy $E_{\pi}$:

\begin{equation}
  \sigma_m = \frac{m_{\pi}}{2}\left[\frac{4 \sigma^2_1}{E_{\pi}} + 2 \sigma^2_2 +
    \sigma^2_{\theta_{12}}\left(\frac{4 E^2_{\pi}}{m^2_{\pi}}-1\right)^2\right]^{1/2}.
\label{eq:sigma2}
\end{equation}

From Eq.~\ref{eq:sigma2} it is possible to identify the contributions to $\sigma_m$ due to energy resolution
$\sigma_1$, calibration error $\sigma_2$, and opening angle resolution $\sigma_{\theta_{12}}$ by measuring the
invariant mass as a function of pion energy. This study was performed using a single 10.6~GeV run to accumulate
a high statistics sample of symmetric $\pi^{0}$ decays. Gaussian fits similar to those shown in
Fig.~\ref{fig:fx-pi0-fits} were used to extract the mean and sigma of the invariant mass peak over a photon energy
range of 0.2-4.0~GeV, corresponding to a pion energy range of 0.5-8.0~GeV. Results from the fits are shown in
Fig.~\ref{fig:fx-study-summary}, where a sector-based analysis (open symbols) and a global analysis summed over
sectors (solid symbols) are plotted.  

The analysis of symmetric decays, which are kinematically over-constrained, also permits an estimate for the photon
energy correction~\cite{2006015}, under the assumption that unphysical values of the invariant mass arise solely
from this dependence. The GEMC prediction of the photon sampling fraction for the ECAL is shown by the solid line
in Fig.~\ref{fig:fx-study-summary-2}. The calculated energy dependence and absolute magnitude depend largely on
the lead-scintillator design and details of the inert material between the PCAL and EC and pre-radiation in the
materials in front of the ECAL. The fitted photon energy correction from the symmetric decay analysis is seen to
follow the GEMC parameterization, with some systematic deviations that are being studied. Our current physics
analysis uses the GEMC result.

\begin{figure}[h]
\centering
\includegraphics[width=1.0\columnwidth,keepaspectratio]{img/fx-study-summary.png}
\caption[]{Summary of fits to distributions of invariant mass from $\pi^0 \rightarrow \gamma \gamma$ symmetric
  decays showing the mass resolution $\sigma_m$ vs. $\pi^0$ energy. A model of the mass resolution using
  Eq.~\ref{eq:sigma2} (black line) was fitted to the energy dependence of $\sigma_m$ to extract estimates of the photon
  energy (blue) and opening angle (green) resolution, as well as the overall calibration uncertainty (red).}
\label{fig:fx-study-summary}
\end{figure}

\begin{figure}[h]
\centering
\includegraphics[width=1.0\columnwidth,keepaspectratio]{img/fx-study-summary-2.png}
\caption[]{Summary of fits to invariant mass peaks from $\pi^0 \rightarrow \gamma \gamma$ symmetric decays
  used to determine the calorimeter sampling fraction for photons. The solid line shows the GEMC calculated
  photon sampling fraction.}
\label{fig:fx-study-summary-2}
\end{figure}

\subsection{Neutron Detection}

Clusters in the ECAL not associated with any reconstructed forward track are designated neutrals by the Event
Builder service. Association of neutral PCAL clusters with ECIN and ECOU clusters are based on proximity to
straight line trajectories from the scattering target centerline to the PCAL cluster. Photons and neutrons are
distinguished on the basis of the timing response, with neutrons defined as having velocity $\beta < 0.9$.
Experiments that require the exclusive measurement of neutrons in the ECAL therefore demand both adequate timing
resolution and sufficient detection efficiency. 

\begin{figure}[h]
\centering
\includegraphics[width=1.0\columnwidth,keepaspectratio]{img/S10_4_0.png}
\caption[]{Missing mass $M_X$ distribution from detection of a $e^-~\pi^+$ final state using a proton target. The
  beam energy was 7.5~GeV. Single neutrons are selected with the cut $M_X^2 < 1.2$~GeV.}
\label{fig:S10_4_0}
\end{figure}

\begin{figure}[h]
\centering
\includegraphics[width=1.0\columnwidth,keepaspectratio]{img/S10_4_1.png}
\caption[]{Differences in direction cosines between the missing momentum of the tagged neutron and the detected
  neutral cluster in the ECAL. The vertical lines show the cuts used to minimize the background from uncorrelated
  photons.}
\label{fig:S10_4_1}
\end{figure}

\begin{figure}[h]
\centering
\includegraphics[width=1.0\columnwidth,keepaspectratio]{img/S10_4_2.png}
\caption[]{Correlation between the measured velocity $\beta$ of neutral clusters in the ECAL and the missing
  momentum of the tagged neutrals subject to the $\Delta cx,\Delta cy$ cuts shown in Fig.~\ref{fig:S10_4_1}.
  The black line shows the expected correlation for neutrons.}
\label{fig:S10_4_2}
\end{figure}

Neutron reconstruction in the ECAL was studied using the $p\,(\,e,e'\,\pi^+\,)\,n$ reaction, with a cut on missing mass 
to provide a source of single tagged neutrons (see Fig.~\ref{fig:S10_4_0}.) Data were taken with a 7.5~GeV beam energy. 
Events with missing momentum pointing into the fiducial region of the ECAL were selected. Candidate neutron hits were identified by
requiring the direction of the missing momentum to coincide with the direction of a measured neutral cluster within
the expected angular resolution, assuming the target center as origin. Typical cuts used for these angular
residuals are shown in Fig.~\ref{fig:S10_4_1} for the direction cosines $cx$ and $cy$ of the neutron momentum
vector with respect to the beamline. Events that satisfy the angle cuts are plotted in Fig.~\ref{fig:S10_4_2}. The
data show a clear correlation between the measured velocity $\beta_{neutral}$ and the tagged neutron momentum. The agreement with 
the expected correlation (black line) for a particle with the neutron mass indicates the timing calibration is reasonable.  
Some photon events are also visible, possibly from beam related electromagnetic background from shielding structures.  The resulting mass 
squared $M^2$ distribution calculated from the measured $\beta_{neutral}$ and missing momentum is shown 
in Fig.~\ref{fig:S10_4_3}, where the photon and neutron peaks are clearly separated. 

\begin{figure}[h]
\centering
\includegraphics[width=1.0\columnwidth,keepaspectratio]{img/S10_4_3.png}
\caption[]{Mass distribution of neutrals detected in the PCAL and EC calculated from the measured $\beta$
  and missing momentum of tagged spectator neutrals. Both photon and neutron peaks are visible.}
\label{fig:S10_4_3}
\end{figure}
