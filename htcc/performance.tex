\section{Performance}The HTCC is one of the major CLAS12 systems used in experiments with the electron beam. The most important aspects of the HTCC performance are that it provides good timing, high electron detection efficiency, high signal strength, and high rejection factor of charged pions. All these parameters are critical for the quality of the data obtained in experiments since the detector, in combination with the forward calorimeter [ref. to ECAL], provides a fast trigger signal for CLAS12. As shown in section 6 the MC prediction for the HTCC for electrons is $\approx$100\%. Fig.~\ref{fig:RAFO_2GeV}. shows the experimentally measured electron detection efficiency for elastically scattered electrons at 2 GeV. The corresponding thresholds applied were approximately 2.5 photoelectrons. Measurements were performed using a special procedure with a random trigger that was not correlated with the HTCC. There were observed 27 events not detected by the HTCC due to the applied threshold. As shown, the electron detection efficiency is $\eta$ = (99$\pm$0.2)\%, which is in good agreement with the MC estimate. This result can be considered as a conservative estimate due to relatively high threshold used in measurements. Moreover, electrons travel a longer distance in the radiator gas (10 \% to 30 \% difference depending on a scattering angle). For these electrons the signal strength is higher, and therefore the detection efficiency is higher also as compared with the aforementioned efficiency for the elastically scattered electrons.   

\begin{figure}[!ht]
    \centering
    \includegraphics[width=1.0\linewidth,trim={0.0cm 0.0cm 0.0cm 0.0cm},clip]{images/RAFO_2GeV.jpg}
    \caption{Electron detection efficiency for elastically scattered electrons at 2 GeV. Data are obtained with the random trigger not correlated with the HTCC or other detector components of CLAS12.}
    \label{fig:RAFO_2GeV}
\end{figure}

Fig.~\ref{fig:positivePNPEC6595} shows the response of the detector in a wide range of particle momentum. The increase of number of events at high momenta is due to registration of charged pions (above threshold of their registration in the HTCC) and this is clearly illustrated.

\begin{figure}[!ht]
    \centering
    \includegraphics[width=1.0\linewidth,trim={0.0cm 0.0cm 0.0cm 0.0cm},clip]{images/positivePNPEC6595.png}
    \caption{Distributions of the HTCC response in a wide momentum range, including the region beyond the threshold of charged pion registration. Data obtained for positrons and $\pi^{+}$-mesons.}
    \label{fig:positivePNPEC6595}
\end{figure}

\begin{comment}
As shown bellow in the Fig.~\ref{fig:avgNPE_Theta_Phi_Dev_Build-2_NO_HOLES} the signal strength goes up for the utmost mirrors (large electron scattering angles). This is because electrons travel a longer distance in the radiator gas (10\% to 30\% difference depending on angle.) In other words the electron detection efficiency obtained for elastically scattered electrons at 2 GeV can be considered as as a conservative estimate for the efficiency of electron detection at larger angles.
\end{comment} 

The signal strength in the HTCC depends on the actual properties of the mirror facets, such as their final shape and reflectance. The accuracy of the combined mirror assembly and the alignment of the HTCC components (mirror, PMTs, Winston Cones), and the composition of the radiator gas all influence the final results. The FADC histogram of  the typical signal strength distribution obtained in one half-sector \#1 and \#2 of Sector 1 is shown in Fig.~\ref{fig:Signal_S1_HS1_HS2_R1_R2}. The signal strength for scattered electrons averaged over all HTCC channels is shown in Fig.~\ref{fig:Average_HTCC_Signal}. The experimentally measured mean value of 16.3 phe is close to Monte-Carlo simulation results, (see Fig.~\ref{fig:10cm_Targ_5T_Field_Phi}).

\begin{figure}[!ht]
    \centering
    \includegraphics[width=1.0\linewidth,trim={0.0cm 0.0cm 0.0cm 0.0cm},clip]{images/Signal_S1_HS1_HS2_R1_R2.jpg}
    \caption{Typical distributions of the signal strength in channels covering polar angles in the range of $5^\circ$ to $12.5^\circ$ (Ring 1) and $12.5^\circ$ to $20.0^\circ$ (Ring 2) within azimuthal interval of $60^\circ$.}
    \label{fig:Signal_S1_HS1_HS2_R1_R2}
\end{figure}

\begin{figure}[!ht]
    \centering
    \includegraphics[width=1.0\linewidth,trim={0.0cm 0.0cm 0.0cm 0.0cm},clip]{images/Average_HTCC_Signal.jpg}
    \caption{The HTCC average signal strength for electrons from beam data.}
    \label{fig:Average_HTCC_Signal}
\end{figure}

Fig.~\ref{fig:HTCC_Response_run4013} shows the HTCC response for different electron momenta. Fig.~\ref{fig:avgNPE_Theta_Phi_Dev_Build-2_NO_HOLES}  shows the distribution of the HTCC response over the entire face of the mirror in the $x-y$-plane. Similar distribution is shown in Fig.~\ref{fig:avgNPE_XY_Dev_Build_02npe} obtained at the lower electron detection threshold of 0.2 photoelectrons. At the large electron scattering angles in range of 27.5$^\circ$ to 35$^\circ$, the statistics is lower. Fig.~\ref{fig:statistics_Theta_Phi_Dev_Build_NO_HOLES} shows the distribution of statistics in all 6 sectors. The data shows that the integrated signal strength is about 16.5 photoelectrons.

\begin{figure}[!ht]
    \centering
    \includegraphics[width=1.0\linewidth,trim={0.0cm 0.0cm 0.0cm 1.73cm},clip]{images/HTCC_Response_run4013.png}
    \caption{The HTCC response for electrons: signal strength vs. momentum at 10 GeV energy.}
    \label{fig:HTCC_Response_run4013}
\end{figure}

\begin{figure}[!ht]
    \centering
    \includegraphics[width=1.0\linewidth,trim={0.0cm 0.0cm 0.0cm 1.67cm},clip]{images/avgNPE_Theta_Phi_Dev_Build-2_NO_HOLES.png}
    \caption{The HTCC response (in $N_{phe}$) for electrons in $x-y$-plane of the mirror.}
    \label{fig:avgNPE_Theta_Phi_Dev_Build-2_NO_HOLES}
\end{figure}

\begin{figure}[!ht]
    \centering
    \includegraphics[width=1.0\linewidth,trim={0.0cm 0.0cm 0.0cm 1.67cm},clip]{images/avgNPE_XY_Dev_Build_02npe.png}
    \caption{The HTCC response (in $N_{phe}$) for electrons in $x-y$-plane of the mirror at the electron detection threshold of 0.2 photoelectrons.}
    \label{fig:avgNPE_XY_Dev_Build_02npe}
\end{figure}

\begin{figure}[!ht]
    \centering
    \includegraphics[width=1.0\linewidth,trim={0.0cm 0.0cm 0.0cm 1.67cm},clip]{images/statistics_Theta_Phi_Dev_Build_NO_HOLES.png}
    \caption{Distribution of statistics in all 6 sectors.}
    \label{fig:statistics_Theta_Phi_Dev_Build_NO_HOLES}
\end{figure}

We also note that in cases when the electrons cross the mirror close to its edges (at approximately at 5$^\circ$ and 35$^\circ$) one should expect unavoidable losses in the signal strength: some part of the Cherenkov light just passes by the mirror. As far as the internal borders between adjacent mirrors are concerned, there are similar losses that take place and are finally partially compensated due to the complete azimuthal symmetry of the detector, see Fig.~\ref{fig:avgNPE_Theta_Phi_Dev_Build-2_NO_HOLES}. The width of that area along internal boundaries that is deformed in the direction normal to the mirror face due to the shrinkage of the glue is estimated between $\sim$5 to $\sim$10 mm. This area includes the technological zone ~0.5 mm of width that is not reflecting the light at all. As a result these regions (width up to $\sim$10 mm) along internal boundaries between mirror facets defuse the light impinging the area, and therefore the signal strength is reduced. this edge effect is normal for the given design of the detector.

