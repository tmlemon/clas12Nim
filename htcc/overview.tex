\section{Overview}

The CLAS12 spectrometer has been designed and built for comprehensive experimental studies of matter,
using primarily a high-energy electron beam~\cite{clas12-nim}. For these experiments this spectrometer
must be capable of detecting scattered electrons within the entirety of its forward acceptance range
and at the highest possible efficiency with low background. The High Threshold Cerenkov Counter (HTCC) (see
Fig.~\ref{fig:Picture1}) in CLAS12 exists to fulfill this goal---to detect scattered electrons in conjunction with
other detector systems to generate a fast trigger signal. 

\begin{figure}[ht]
    \centering
    \includegraphics[width=0.75\linewidth]{images/Picture1.jpg}
    \caption{Fully assembled High Threshold Cherenkov Counter.}
    \label{fig:Picture1}
\end{figure}
The distinguishing features of the detector were influenced by its location in front of the drift chambers
(DC)~\cite{dc-nim}, which required that the HTCC incorporate a minimum amount of material in the active
area in front of the tracking detectors. Because the HTCC is a single module system, it occupies very limited
space within CLAS12. Consequently, the construction requirements---including transportation to the hall and
installation into the nominal location of the detector---were important for its structural design. 
