\section{Acknowledgements}

\indent We appreciate the contribution and valuable help of D. McKay, C. Apeldoorn and M. Powers in development and implementation of a new technology for the HTCC construction. Only due to their direct involvement in the project and tight control of results it was possible to solve successfully so many difficult problems, and consequently completely avoid any compromises that would otherwise lead to reduced overall performance of the detector.

\section{Conclusion} The High Threshold Cherenkov Counter has been designed and build to meet all requirements that were defined mostly by its location and available space in front of the forward tracking system, drift chambers, of the CLAS12 spectrometer. A new technology of building light-weight multifocal ellipsoidal mirror was developed and successfully used. The detector introduces a small amount of material in the CLAS12 acceptance that is only the radiator gas and the mirror itself that has thickness less than the total radiation length of the $CO_{2}$ radiator. There is no elements within the acceptance that support the HTCC mirror. The detector provides a full azimuthal coverage and very efficient light collection: the Cherenkov light is detected after one reflection in 80\% of events and after two reflections in remaining 20\% of events. Experiments with electron beam have confirmed all designed working parameters of the detector. Performance of the HTCC is adequate, reliable and meets all expectations for CLAS12 experiments.



