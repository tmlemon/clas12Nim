\section{Beamline}

The CLAS12 beamline is made up of several pieces, each discussed below. The positioning and composition of the beamline
depends on the run configuration, that can be:

\begin{itemize}
	\item FTOn: Forward Tagger present and operational. The shelding is moved downstream to accommodate the FT acceptance between
          $2.5$ and $5$ degrees.
	\item FTOff: Forward Tagger is present except the tracker, and turned off.
\end{itemize}


\subsection{Vacuum pipe}

A stainless steel vacuum pipes that contain the eletron beam. The pipe starts downstream of the target at $z=cm$
changes radius inside the torus and downstream of the torus.

\subsection{Moeller Shielding}


\subsection{Torus Shielding}
shielding around the vacuum pipe inside the torus hub in the form of tungsten bricks

\subsection{Donwstream shielding}
shielding downstream of the torus in the form of a connecting tungsten nose and a long lead pipe.


There are two

\subsection{Geometry}

The beamline geometry is entirely imported from the engineering CAD model.


\subsubsection{Geometry Git Location}

