\subsection{Central Neutron Detector (CND)}

\subsection{Geometry}

The CND geometry is implemented through the native GEMC geometry API.
The paddles are Geant4 generic trapezoids, see \F{cndGeometry}. The u-turn light guides are G4Polycons (volumes with cylindrical symmetry
with varying radius along one axis).
The paddles are assigned the scintillator material and associated with the CND hit process routine.

\begin{figure}
	\centering
	\includegraphics[width=0.99\columnwidth,keepaspectratio]{img/cndGeometry.png}
	\includegraphics[width=0.99\columnwidth,keepaspectratio]{img/cndDetail.png}
	\caption{Top: overall view of the CND detector. Three layers of scintillators are placed at increasing ``z''. Pairs of scintillators
            are connected through a scintillator u-turn junction. Bottom: enlarged view of the junctions. }
	\label{fig:cndGeometry}
\end{figure}


\subsubsection{Geometry Location on GitHub}
The Github location of the GEMC perl API script is \url{https://github.com/gemc/detectors/tree/master/clas12/cnd}.


\subsection{Digitization}


The energy deposited is reduced based on the position in the paddle using the calibrated attenuation length.
Two signals are then propagated, one directly to the PMT and one through the scintillator junction.
Layer-dependent factors, applied to the two  signals, account for the light loss in the u-turn and in the neighboring paddle.
These factors were determined during cosmic-ray tests.

The corrected energy is converted to the theoretical number of photons $N_{th}$ using the constant 1210 $\gamma$ / MeV, which accounts for light
propagation in the 1.4 m long light guides, for losses at the junctions and for the quantum efficiency of the photomultiplier.
A Poissonian is used to
calculate the actual number of photons $N_{actual}$ and the resulting ``smeared'' energy is the converted to ADC using the FADC conversion factor.


The absolute hit time is corrected by:

\begin{itemize}
	\item the effective velocity (from CCDB)
	\item a left/right time offset factor (from CCDB)
	\item the Birks-attenuation factor
	\item the position and corresponding paddle length of the indirect hit
\end{itemize}

The time is then smeared by a resolution read from CCDB using a Gaussian function and then digitized using a TDC conversion factor.
The Birks factor, reducing the deposited
energy depending on the particle type, enters in the timing calculation as
follows: the direct and indirect times are smeared with a Gaussian
function having a width directly proportional to an empirically determined
constant, and inversely proportional to the square root of the measured
light (which is, in turn, proportional to the attenuated energy).



The digitized output bank variables are summarized in Table \ref{tab:cndBank}.

\begin{table}[h]
	\begin{center}
		\begin{tabular}{| c | c | c |}
			\hline \hline
			Variable  &          Description     \\
			\hline
              sector  &        sector number     \\
               layer  &         layer number     \\
           component  &     component number     \\
                ADCL  &             ADC Left     \\
                ADCR  &            ADC Right     \\
                TDCL  &             TDC Left     \\
                TDCR  &            TDC Right     \\
                hitn  &           hit number     \\
			\hline \hline
		\end{tabular}
	\end{center}
	\caption{The digitized CND bank.}\label{tab:cndBank}
\end{table}


The time window  of the CND is set to 5 ns: all geant4 steps within the same paddle and time window will be collected on one hit.
The CND hit process routine location in git is \url{https://github.com/gemc/source/blob/master/hitprocess/clas12/cnd_hitprocess.cc}
