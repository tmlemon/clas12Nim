\subsection{Target}

The CLAS12 target components are imported from the engineering model. The STEP files are converted to tessellated STL files and imported
in the GEMC simulation \cite{targetCorrection}, \cite{targetStudy}. An example of the tessellation is shown in \F{targetScatteringChamber}.

Key elements of the STL import include the torlon tube to the target cell,
the target aluminum windows, the Kapton walls, and the scattering chamber, see \F{targetDesign}.
An overview of the target in Geant4 and the engineering model is shown in \F{targetOverview}.

\begin{figure}
	\centering
	\includegraphics[width=0.95\columnwidth,keepaspectratio]{img/targetDesign.png}
	\caption{The CLAS12 target design. Top left: the entry assembly schematic. Top right: the liquid hydrogen cell
            dimensions: the outer radius is tapered down from 15 mm at z=-2.5cm to 10mm at z=2.5mm.
            Bottom left: The cell implementation in GEMC from the CAD drawings. From left to right (beam direction):
            the black torlon tube, the upstream aluminum window, the target cell, the kapton cup and the
				downstream aluminum window. Bottom right: the GEMC implementation of the kapton cup.}
	\label{fig:targetDesign}
\end{figure}


\begin{figure}
	\centering
	\includegraphics[width=0.99\columnwidth,keepaspectratio]{img/targetOverview1.png}
	\includegraphics[width=0.99\columnwidth,keepaspectratio]{img/targetOverview2.png}
	\caption{Top: overview of the target implementation in GEMC includes the scattering chamber (cyan color), the
            downstream cup near the right of the figure and 50 $\mu$m aluminum window. Bottom: the torlon base
            tube starts at a radius of r=6.06 mm, and ends at r = 7.75 mm. It is 63.7 mm long.}
	\label{fig:targetOverview}
\end{figure}

The Github location of the GEMC perl API scripts and the STL files is \url{https://github.com/gemc/detectors/tree/master/clas12/targets}.





